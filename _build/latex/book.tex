%% Generated by Sphinx.
\def\sphinxdocclass{jupyterBook}
\documentclass[letterpaper,10pt,english]{jupyterBook}
\ifdefined\pdfpxdimen
   \let\sphinxpxdimen\pdfpxdimen\else\newdimen\sphinxpxdimen
\fi \sphinxpxdimen=.75bp\relax
\ifdefined\pdfimageresolution
    \pdfimageresolution= \numexpr \dimexpr1in\relax/\sphinxpxdimen\relax
\fi
%% let collapsible pdf bookmarks panel have high depth per default
\PassOptionsToPackage{bookmarksdepth=5}{hyperref}
%% turn off hyperref patch of \index as sphinx.xdy xindy module takes care of
%% suitable \hyperpage mark-up, working around hyperref-xindy incompatibility
\PassOptionsToPackage{hyperindex=false}{hyperref}
%% memoir class requires extra handling
\makeatletter\@ifclassloaded{memoir}
{\ifdefined\memhyperindexfalse\memhyperindexfalse\fi}{}\makeatother

\PassOptionsToPackage{warn}{textcomp}

\catcode`^^^^00a0\active\protected\def^^^^00a0{\leavevmode\nobreak\ }
\usepackage{cmap}
\usepackage{fontspec}
\defaultfontfeatures[\rmfamily,\sffamily,\ttfamily]{}
\usepackage{amsmath,amssymb,amstext}
\usepackage{polyglossia}
\setmainlanguage{english}



\setmainfont{FreeSerif}[
  Extension      = .otf,
  UprightFont    = *,
  ItalicFont     = *Italic,
  BoldFont       = *Bold,
  BoldItalicFont = *BoldItalic
]
\setsansfont{FreeSans}[
  Extension      = .otf,
  UprightFont    = *,
  ItalicFont     = *Oblique,
  BoldFont       = *Bold,
  BoldItalicFont = *BoldOblique,
]
\setmonofont{FreeMono}[
  Extension      = .otf,
  UprightFont    = *,
  ItalicFont     = *Oblique,
  BoldFont       = *Bold,
  BoldItalicFont = *BoldOblique,
]



\usepackage[Bjarne]{fncychap}
\usepackage[,numfigreset=1,mathnumfig]{sphinx}

\fvset{fontsize=\small}
\usepackage{geometry}


% Include hyperref last.
\usepackage{hyperref}
% Fix anchor placement for figures with captions.
\usepackage{hypcap}% it must be loaded after hyperref.
% Set up styles of URL: it should be placed after hyperref.
\urlstyle{same}

\addto\captionsenglish{\renewcommand{\contentsname}{Introducción}}

\usepackage{sphinxmessages}



        % Start of preamble defined in sphinx-jupyterbook-latex %
         \usepackage[Latin,Greek]{ucharclasses}
        \usepackage{unicode-math}
        % fixing title of the toc
        \addto\captionsenglish{\renewcommand{\contentsname}{Contents}}
        \hypersetup{
            pdfencoding=auto,
            psdextra
        }
        % End of preamble defined in sphinx-jupyterbook-latex %
        

\title{MEC501 - Manejo y conversión de energía solar térmica}
\date{Oct 07, 2022}
\release{}
\author{Francisco V.\@{} Ramirez\sphinxhyphen{}Cuevas}
\newcommand{\sphinxlogo}{\vbox{}}
\renewcommand{\releasename}{}
\makeindex
\begin{document}

\pagestyle{empty}
\sphinxmaketitle
\pagestyle{plain}
\sphinxtableofcontents
\pagestyle{normal}
\phantomsection\label{\detokenize{intro::doc}}


\sphinxAtStartPar
En este curso revisaremos los fundamentos y aplicaciones de tecnologías para el manejo y
conversión de la energía solar. La primera parte de este curso estará enfocada hacia los
fundamentos de la radiación solar. Analizaremos el fenómeno electromagnético de la radiación
y su interacción con la materia para explicar los mecanismos detrás de la respuesta óptica
en diversos escenarios. Revisando aspectos como refracción, dispersión (scattering) y
transporte radiativo de energía, responderemos preguntas como: ¿Por qué la luz se propaga
por el vacío?, ¿Por qué el cielo es azul?, ¿Por qué la leche es blanca?, entre otras
interrogantes. Finalizaremos esta primera parte con los fundamentos de la transferencia de
calor por radiación, y aspectos principales de la radiación solar. Esta parte del curso
permitirá establecer una base sólida en torno a la manipulación de la energía solar, así
como el rol de la nanofotónica en el desarrollo de futuras tecnologías.

\sphinxAtStartPar
En la segunda parte revisaremos las principales tecnologías de conversión y manejo de
energía solar. Comenzaremos analizando colectores solares aplicados a sistemas de agua
caliente en sectores residenciales. Debido a la simpleza en su diseño, el análisis de estos
sistemas servirá como plataforma para aplicar y reforzar los conceptos aprendidos en la
primera parte. Luego, revisaremos centrales termosolares, enfocándonos en los colectores
concentradores y sus diversas configuraciones. También veremos celdas fotovoltaicas, y
como mejorar la eficiencia mediante el manejo de la radiación térmica. Posteriormente,
estudiaremos el manejo de la radiación solar en edificaciones, donde revisaremos algunos
materiales de termorregulación solar y sus aplicaciones. Finalizaremos el curso revisando
otras tecnologías de conversión solar aún en desarrollo como desalinización y combustibles
solares, discutiendo su potencial y limitaciones.
\begin{itemize}
\item {} 
\sphinxAtStartPar
Introducción

\begin{itemize}
\item {} 
\sphinxAtStartPar
{\hyperref[\detokenize{0_introduccion/0_introduccion::doc}]{\sphinxcrossref{Introducción al curso}}}

\end{itemize}
\end{itemize}
\begin{itemize}
\item {} 
\sphinxAtStartPar
Unidades del curso

\begin{itemize}
\item {} 
\sphinxAtStartPar
{\hyperref[\detokenize{1_ondas_electromagneticas/1_ondas_electromagneticas::doc}]{\sphinxcrossref{La radiación como un fenómeno electromagnético}}}

\item {} 
\sphinxAtStartPar
{\hyperref[\detokenize{2_ondas_EM_en_la_materia/2_ondas_EM_en_la_materia::doc}]{\sphinxcrossref{Ondas electromagnéticas en la materia}}}

\item {} 
\sphinxAtStartPar
{\hyperref[\detokenize{3_Interacci_xf3n_materia-luz/3_Interacci_xf3n_materia-luz::doc}]{\sphinxcrossref{Interacción materia\sphinxhyphen{}luz}}}

\item {} 
\sphinxAtStartPar
{\hyperref[\detokenize{4_Scattering/4_Scattering::doc}]{\sphinxcrossref{Scattering electromagnético}}}

\item {} 
\sphinxAtStartPar
{\hyperref[\detokenize{5_TransporteRadiativo/5_TransporteRadiativo::doc}]{\sphinxcrossref{Transporte Radiativo}}}

\item {} 
\sphinxAtStartPar
{\hyperref[\detokenize{6_RadiacionTermica/6_RadiacionTermica::doc}]{\sphinxcrossref{Radiación Térmica}}}

\end{itemize}
\end{itemize}

\sphinxstepscope


\part{Introducción}

\sphinxstepscope

\sphinxAtStartPar
MEC501 \sphinxhyphen{} Manejo y Conversión de Energía Solar Térmica


\chapter{Introducción al curso}
\label{\detokenize{0_introduccion/0_introduccion:introduccion-al-curso}}\label{\detokenize{0_introduccion/0_introduccion::doc}}
\sphinxAtStartPar

Profesor: Francisco Ramírez Cuevas
Fecha: 1 de Agosto 2022


\section{Situación actual del sistema energético}
\label{\detokenize{0_introduccion/0_introduccion:situacion-actual-del-sistema-energetico}}
\begin{sphinxuseclass}{cell}\begin{sphinxVerbatimInput}

\begin{sphinxuseclass}{cell_input}
\begin{sphinxVerbatim}[commandchars=\\\{\}]
\PYG{k+kn}{from} \PYG{n+nn}{IPython}\PYG{n+nn}{.}\PYG{n+nn}{display} \PYG{k+kn}{import} \PYG{n}{display}\PYG{p}{,} \PYG{n}{HTML}\PYG{p}{,} \PYG{n}{IFrame}
\PYG{n}{display}\PYG{p}{(}\PYG{n}{IFrame}\PYG{p}{(}\PYG{l+s+s1}{\PYGZsq{}}\PYG{l+s+s1}{https://ourworldindata.org/grapher/energy\PYGZhy{}consumption\PYGZhy{}by\PYGZhy{}source\PYGZhy{}and\PYGZhy{}region?stackMode=absolute}\PYG{l+s+s1}{\PYGZsq{}}\PYG{p}{,} \PYG{l+s+s1}{\PYGZsq{}}\PYG{l+s+s1}{900px}\PYG{l+s+s1}{\PYGZsq{}}\PYG{p}{,} \PYG{l+s+s1}{\PYGZsq{}}\PYG{l+s+s1}{600px}\PYG{l+s+s1}{\PYGZsq{}}\PYG{p}{)}\PYG{p}{)}
\end{sphinxVerbatim}

\end{sphinxuseclass}\end{sphinxVerbatimInput}
\begin{sphinxVerbatimOutput}

\begin{sphinxuseclass}{cell_output}
\begin{sphinxVerbatim}[commandchars=\\\{\}]
\PYGZlt{}IPython.lib.display.IFrame at 0x7f95b04c3220\PYGZgt{}
\end{sphinxVerbatim}

\end{sphinxuseclass}\end{sphinxVerbatimOutput}

\end{sphinxuseclass}
\sphinxAtStartPar
\sphinxstylestrong{Datos relevantes}
\begin{itemize}
\item {} 
\sphinxAtStartPar
El consumo mundial de combustibles fósiles (petróleo, gas y carbón) a aumentado en casi 65\% desde el año 1990 hasta el año 2021.

\item {} 
\sphinxAtStartPar
El porcentaje de energía proveniente de combustibles fósiles, ha disminuido de casi un 87\% en 1990 a 83\% en el 2021 (en Chile, este porcentaje cayó desde 82\% a 76.5\%).

\item {} 
\sphinxAtStartPar
Actualmente en Chile, las energías renovables representan un 23.5\%, donde un 6.25\% proviene de energía solar.

\end{itemize}
\begin{quote}

\sphinxAtStartPar
Por otro lado, \sphinxstylestrong{la disponiblidad de combustilbes fósiles es limitada}. Segun estudios ´\sphinxhref{https://earthbuddies.net/when-will-we-run-out-of-fossil-fuel/}{las reservas de petroleo, gas y carbón se agotarán para los años 2052, 2060 y 2090, respectivamente.}
\end{quote}
\begin{quote}

\sphinxAtStartPar
Esto significa que los precios de los combustibles fósiles seguirán aumentando en las próximas décadas.
\end{quote}


\section{Problemas medioambientales asociados al sistema energético actual}
\label{\detokenize{0_introduccion/0_introduccion:problemas-medioambientales-asociados-al-sistema-energetico-actual}}
\sphinxAtStartPar
Como vimos, aunque el consumo de energías renovables ha ido en aumento, la quema de combustibles fósiles sigue siendo la principal fuente de energía en el mundo.

\sphinxAtStartPar
Además de los problemas asociados a la disponibilidad limitada y aumento de precio de los combustibles fósiles, existen consecuencias medioambientales que tienen asociado un costo indirecto, tales como:
\begin{itemize}
\item {} 
\sphinxAtStartPar
Lluvia ácida asociada a la emisión de SO\(_2\) y NO\(_x\)

\item {} 
\sphinxAtStartPar
Disminución de la capa de ozono por emisión de CFC y NO\(_x\). (\sphinxstyleemphasis{Aunque se han tomado una serie de medidas para reducir las emisiones de CFC, se estima que el daño en zonas como la Antartica seguirá presenta hasta, al menos, el año 2075.})

\item {} 
\sphinxAtStartPar
Cambio climático producto del aumento de gases de efecto invernadero

\end{itemize}


\section{Tecnologías de manejo y conversión de energía solar}
\label{\detokenize{0_introduccion/0_introduccion:tecnologias-de-manejo-y-conversion-de-energia-solar}}

\subsection{Aspectos generales de la energía solar}
\label{\detokenize{0_introduccion/0_introduccion:aspectos-generales-de-la-energia-solar}}\begin{itemize}
\item {} 
\sphinxAtStartPar
El sol es la única fuente externa de energía en la tierra

\item {} 
\sphinxAtStartPar
Todas la formas de energía disponibles tiene origen solar (combustibles fósiles, mareomotríz, eólica, etc)

\end{itemize}


\subsection{Disponibilidad de energía solar}
\label{\detokenize{0_introduccion/0_introduccion:disponibilidad-de-energia-solar}}
\begin{sphinxuseclass}{cell}\begin{sphinxVerbatimInput}

\begin{sphinxuseclass}{cell_input}
\begin{sphinxVerbatim}[commandchars=\\\{\}]
\PYG{n}{display}\PYG{p}{(}\PYG{n}{IFrame}\PYG{p}{(}\PYG{l+s+s1}{\PYGZsq{}}\PYG{l+s+s1}{https://globalsolaratlas.info/}\PYG{l+s+s1}{\PYGZsq{}}\PYG{p}{,}\PYG{l+s+s1}{\PYGZsq{}}\PYG{l+s+s1}{100}\PYG{l+s+s1}{\PYGZpc{}}\PYG{l+s+s1}{\PYGZsq{}}\PYG{p}{,}\PYG{l+s+s1}{\PYGZsq{}}\PYG{l+s+s1}{600px}\PYG{l+s+s1}{\PYGZsq{}}\PYG{p}{)}\PYG{p}{)}
\end{sphinxVerbatim}

\end{sphinxuseclass}\end{sphinxVerbatimInput}
\begin{sphinxVerbatimOutput}

\begin{sphinxuseclass}{cell_output}
\begin{sphinxVerbatim}[commandchars=\\\{\}]
\PYGZlt{}IPython.lib.display.IFrame at 0x7f95b049b1f0\PYGZgt{}
\end{sphinxVerbatim}

\end{sphinxuseclass}\end{sphinxVerbatimOutput}

\end{sphinxuseclass}
\sphinxAtStartPar
Chile es el país con mayores niveles de radiación en el mundo.

\sphinxAtStartPar
Por ejemplo, en base al atlas solar del \sphinxhref{https://globalsolaratlas.info/map?c=11.523088,8.261719,3}{World Bank Group}, si calculamos la energía generada por el área con mayores niveles de radiación considerando paneles con \sphinxhref{https://solarity.cz/blog/500-wp-solar-modules-era/}{potencia máxima de 500 Wp}

\begin{sphinxuseclass}{cell}\begin{sphinxVerbatimInput}

\begin{sphinxuseclass}{cell_input}
\begin{sphinxVerbatim}[commandchars=\\\{\}]
\PYG{n}{A}     \PYG{o}{=} \PYG{l+m+mf}{250060.15} \PYG{c+c1}{\PYGZsh{} Superficie total (km\PYGZca{}2)}
\PYG{n}{Pmax}  \PYG{o}{=} \PYG{l+m+mi}{500}       \PYG{c+c1}{\PYGZsh{} Potencia máxima por panel en condiciones estándard (Wp)}
\PYG{n}{PVOUT} \PYG{o}{=} \PYG{l+m+mf}{6.0}       \PYG{c+c1}{\PYGZsh{} Potencía específica suministrada (kWh/Wp)}

\PYG{c+c1}{\PYGZsh{} Energía total suministrada (TWh)}
\PYG{n}{Etot}  \PYG{o}{=} \PYG{n}{A}\PYG{o}{*}\PYG{l+m+mf}{1E3}\PYG{o}{*}\PYG{o}{*}\PYG{l+m+mi}{2}\PYG{o}{*}\PYG{n}{PVOUT}\PYG{o}{*}\PYG{n}{Pmax}\PYG{o}{/}\PYG{l+m+mf}{1E12}
\PYG{n+nb}{print}\PYG{p}{(}\PYG{l+s+s2}{\PYGZdq{}}\PYG{l+s+s2}{Energía eléctrica suministrada: }\PYG{l+s+si}{\PYGZpc{}.1f}\PYG{l+s+s2}{ TWh (Energía consumida en Chile 444 TWh)}\PYG{l+s+s2}{\PYGZdq{}} \PYG{o}{\PYGZpc{}} \PYG{n}{Etot}\PYG{p}{)}
\end{sphinxVerbatim}

\end{sphinxuseclass}\end{sphinxVerbatimInput}
\begin{sphinxVerbatimOutput}

\begin{sphinxuseclass}{cell_output}
\begin{sphinxVerbatim}[commandchars=\\\{\}]
Energía eléctrica suministrada: 750.2 TWh (Energía consumida en Chile 444 TWh)
\end{sphinxVerbatim}

\end{sphinxuseclass}\end{sphinxVerbatimOutput}

\end{sphinxuseclass}\begin{quote}

\sphinxAtStartPar
“\sphinxstyleemphasis{Para abastecer toda la energía que requiere Chile si tuviéramos almacenamiento suficiente necesitamos unos mil kilómetros cuadrados, algo menos que el 1\% del desierto y equivalente más o menos a la superficie de la comuna de Melipilla.}”  \sphinxhref{https://www.revistaei.cl/2019/09/04/con-todo-el-potencial-de-energia-solar-de-chile-se-podria-abastecer-60-veces-el-consumo-del-pais-y-el-20-del-mundo/}{Rodrigo Palma, director de SERC, 2022}
\end{quote}


\subsection{Descripción de tecnologías para energía solar}
\label{\detokenize{0_introduccion/0_introduccion:descripcion-de-tecnologias-para-energia-solar}}\begin{itemize}
\item {} 
\sphinxAtStartPar
Fotovoltaica
\sphinxincludegraphics[width=400\sphinxpxdimen]{{pv_cell}.jpg}

\end{itemize}
\begin{itemize}
\item {} 
\sphinxAtStartPar
Termosolar
\sphinxincludegraphics[width=400\sphinxpxdimen]{{thermo_solar_power_plant}.jpg}

\end{itemize}
\begin{itemize}
\item {} 
\sphinxAtStartPar
Termoregulación solar para edificaciones
\sphinxincludegraphics[width=400\sphinxpxdimen]{{smart_window}.jpg}

\end{itemize}
\begin{itemize}
\item {} 
\sphinxAtStartPar
Desalinización solar
\sphinxincludegraphics[width=400\sphinxpxdimen]{{solar_desalination}.png}

\end{itemize}
\begin{itemize}
\item {} 
\sphinxAtStartPar
Combustibles solares (ejemplos)

\end{itemize}

\sphinxAtStartPar
Generación fotoelectroquímica

\noindent\sphinxincludegraphics[width=400\sphinxpxdimen]{{artificial_photosynthesis}.png}

\sphinxAtStartPar
Generación termoquímica

\noindent\sphinxincludegraphics[width=400\sphinxpxdimen]{{thermochemical_solar_reactor}.png}


\subsection{Tecnologías en base a energía solar en Chile}
\label{\detokenize{0_introduccion/0_introduccion:tecnologias-en-base-a-energia-solar-en-chile}}
\sphinxAtStartPar
En Chile las principales tecnologías son la fotovoltaica y termosolar, \sphinxhref{https://www.pv-magazine-latam.com/2021/04/13/chile-alcanza-46-gw-fotovoltaicos-de-potencia-instalada/}{con una capacidad instalada de 4.6 GW}. La mayor parte de este simunistro corresponde a plantas solares fotovoltaicas. Respecto al suministro de energía termosolar, se destacan dos proyectos:

\sphinxAtStartPar
\sphinxhref{https://laderasur.com/articulo/recorriendo-cerro-dominador-la-unica-planta-termosolar-que-funciona-en-medio-del-desierto-de-atacama/}{Planta solar cerro dominador}

\noindent\sphinxincludegraphics[width=400\sphinxpxdimen]{{cerro_dominador}.jpg}

\sphinxAtStartPar
\sphinxhref{https://energia.gob.cl/noticias/valparaiso/primer-concentrador-solar-termico-en-valparaiso}{Concentrador solar parabolico en Valparaiso}

\noindent\sphinxincludegraphics[width=400\sphinxpxdimen]{{concentrador_solar_chile}.jpg}


\section{Descripción general de la asignatura}
\label{\detokenize{0_introduccion/0_introduccion:descripcion-general-de-la-asignatura}}
\sphinxAtStartPar
En este curso revisaremos los fundamentos y aplicaciones de tecnologías para el manejo y conversión de la energía solar. \sphinxstylestrong{La primera parte de este curso estará enfocada hacia los fundamentos de la radiación solar}. Analizaremos el fenómeno electromagnético de la radiación y su interacción con la materia para explicar los mecanismos detrás de la respuesta óptica en diversos escenarios.

\sphinxAtStartPar
En la segunda parte revisaremos las principales tecnologías de conversión y manejo de energía solar. \sphinxstylestrong{Comenzaremos analizando colectores solares aplicados a sistemas de agua caliente en sectores residenciales}. Debido a la simpleza en su diseño, el análisis de estos sistemas servirá como plataforma para aplicar y reforzar los conceptos aprendidos en la primera parte. Luego, revisaremos centrales termosolares, celdas fotovoltaicas,  manejo de la radiación solar en edificaciones, y otras tecnologías aún en desarrollo como desalinización y combustibles solares.


\subsection{Objetivos generales}
\label{\detokenize{0_introduccion/0_introduccion:objetivos-generales}}
\sphinxAtStartPar
Aprender los fundamentos del manejo de la radiación térmica y solar, así como sobre las principales tecnologías de conversión y manejo de energía solar


\subsection{Objetivos específicos}
\label{\detokenize{0_introduccion/0_introduccion:objetivos-especificos}}\begin{itemize}
\item {} 
\sphinxAtStartPar
Entender la radiación como un fenómeno electromagnético

\item {} 
\sphinxAtStartPar
Entender la interacción materia\sphinxhyphen{}luz y su relación con la respuesta óptica de los materiales

\item {} 
\sphinxAtStartPar
Comprender los régimenes de análisis para el transporte radiativo en función de la longitud de escala.

\item {} 
\sphinxAtStartPar
Familiarizar al estudiante con las herramientas de modelación de transporte radiativo.

\item {} 
\sphinxAtStartPar
Comprender los fundamentos de la transferencia de calor por radiación.

\item {} 
\sphinxAtStartPar
Entender las principales características de la radiación solar.

\item {} 
\sphinxAtStartPar
Capacitar al estudiante en el diseño de sistemas de agua caliente domiciliarios basados en colectores solares.

\item {} 
\sphinxAtStartPar
Familiarizar al estudiante con las principales tecnologías de manejo y conversión de energía solar.

\item {} 
\sphinxAtStartPar
Familiarizar al estudiante con las tecnologías de manejo y conversión de energía solar emergentes, y el rol de la nanofotónica en el futuro de estas tecnologías.

\end{itemize}


\section{Temario}
\label{\detokenize{0_introduccion/0_introduccion:temario}}

\subsection{Contenidos}
\label{\detokenize{0_introduccion/0_introduccion:contenidos}}\begin{enumerate}
\sphinxsetlistlabels{\arabic}{enumi}{enumii}{}{.}%
\item {} 
\sphinxAtStartPar
La radiación como un fenómeno electromagnético

\item {} 
\sphinxAtStartPar
Interacción materia\sphinxhyphen{}luz

\item {} 
\sphinxAtStartPar
Dispersión (scattering) de la luz y transporte radiativo

\item {} 
\sphinxAtStartPar
Fundamentos de la transferencia de calor por radiación

\item {} 
\sphinxAtStartPar
Radiación Solar

\item {} 
\sphinxAtStartPar
Colectores planos para sistemas de agua caliente termosolar

\item {} 
\sphinxAtStartPar
Plantas termosolares

\item {} 
\sphinxAtStartPar
Celdas fotovoltaicas

\item {} 
\sphinxAtStartPar
Tecnologías de termorregulación solar para edificaciones

\item {} 
\sphinxAtStartPar
Sistemas de desalinización Solar

\item {} 
\sphinxAtStartPar
Dispositivos de conversión de combustibles solares

\end{enumerate}


\section{Evaluación de la asignatura}
\label{\detokenize{0_introduccion/0_introduccion:evaluacion-de-la-asignatura}}\begin{itemize}
\item {} 
\sphinxAtStartPar
Tres (3) pruebas en formato de tareas (80\% de nota de presentación)

\item {} 
\sphinxAtStartPar
Cuestionarios al comienzo de cada clase (20\% de la nota de presentación)

\end{itemize}
\label{equation:0_introduccion/0_introduccion:bf79faf3-d899-49fc-9d89-e683610af0df}\begin{equation}
\mathrm{NP}  = 80\%\mathrm{Promedio Pruebas} + 20\%\mathrm{Promedio Cuestionarios}
\end{equation}\begin{quote}

\sphinxAtStartPar
*El promedio de cuestionarios considera 8 de las mejores notas.
\end{quote}

\sphinxAtStartPar
Aquellos alumnos con \(\mathrm{NP} \geq 5.0\) no requieren rendir examen. En caso de rendir el examen, \sphinxstylestrong{la calificación de este reemplaza la peor nota de las pruebas parciales (solo si la nota del exámen es superior a la peor nota).} La nota final se calcula mediante el promedio de las 3 mejores notas (incluido examen).

\sphinxstepscope


\part{Unidades del curso}

\sphinxstepscope

\sphinxAtStartPar
MEC501 \sphinxhyphen{} Manejo y Conversión de Energía Solar Térmica


\chapter{La radiación como un fenómeno electromagnético}
\label{\detokenize{1_ondas_electromagneticas/1_ondas_electromagneticas:la-radiacion-como-un-fenomeno-electromagnetico}}\label{\detokenize{1_ondas_electromagneticas/1_ondas_electromagneticas::doc}}
\sphinxAtStartPar

Profesor: Francisco Ramírez CueSvas
Fecha: 12 de Agosto 2022


\section{Repaso de cálculo vectorial}
\label{\detokenize{1_ondas_electromagneticas/1_ondas_electromagneticas:repaso-de-calculo-vectorial}}

\subsection{Campo escalar y vectorial}
\label{\detokenize{1_ondas_electromagneticas/1_ondas_electromagneticas:campo-escalar-y-vectorial}}\begin{itemize}
\item {} 
\sphinxAtStartPar
Un campo escalar representa la distribución espacial de una magnitud. Por ejemplo, distribución de densidad, temperatura o presión. En coordenadas cartesianes: \(f = f(x,y,z)\), donde \(f\) es un campo escalar.

\item {} 
\sphinxAtStartPar
Un campo vectorial representa la distribución espacial de una magnitud vectorial. Por ejemplo, distribución de velocidades, campo eléctrico o magnético. En coordenadas cartesianas: \(\vec{f} = \vec{f}(x,y,z)\), donde \(\vec{f}\) es un campo escalar.

\end{itemize}

\sphinxAtStartPar
Por ejemplo, consideremos la siguiente modelación de convección natural en cavidad cuadrada:

\noindent{\hspace*{\fill}\sphinxincludegraphics[width=500\sphinxpxdimen]{{natural_convection}.png}\hspace*{\fill}}

\sphinxAtStartPar
Aquí podemos visualizar la distribución espacial de temperaturas y velocidades de un fluido sometido a las condiciones indicadas en la figura.

\sphinxAtStartPar
De esta figura podemos identificar:
\begin{itemize}
\item {} 
\sphinxAtStartPar
Campo escalar: Distribución de temperaturas

\item {} 
\sphinxAtStartPar
Campo vectorial: Distribución de velocidades

\end{itemize}


\subsection{Operadores diferenciales}
\label{\detokenize{1_ondas_electromagneticas/1_ondas_electromagneticas:operadores-diferenciales}}
\sphinxAtStartPar
\sphinxstylestrong{Operador Del.}

\sphinxAtStartPar
Definimos el operador \(\nabla\) o “del”, como:
\label{equation:1_ondas_electromagneticas/1_ondas_electromagneticas:ce3e612c-f0dc-4c36-9f80-4e62cf464e4e}\begin{equation}
\nabla= \left( \hat{x}\frac{\partial }{\partial x} + \hat{y}\frac{\partial }{\partial y} + \hat{z}\frac{\partial }{\partial z} \right)
\end{equation}
\sphinxAtStartPar
\sphinxstylestrong{Operador Gradiente.}

\sphinxAtStartPar
Es equivalente a la derivada de una función, pero en múltiples dimenciones. Permite identificar zonas de crecimiento o decrecimiento de un campo escalar o vectorial. Se define como el operador Del multiplicado por el campo escalar.
\label{equation:1_ondas_electromagneticas/1_ondas_electromagneticas:3c03554e-673c-4e88-af57-c931aca7b06c}\begin{equation}
\nabla f= \frac{\partial f}{\partial x}\hat{x} + \frac{\partial f}{\partial y}\hat{y}+ \frac{\partial f}{\partial z}\hat{z}
\end{equation}\begin{quote}

\sphinxAtStartPar
El gradiente de un campo escalar \(f\), es un vector
\end{quote}

\sphinxAtStartPar
\sphinxstylestrong{Operador Divergente.}

\sphinxAtStartPar
Se aplica a campos vectoriales. Es una medida de cuanto un campo vectorial diverge o converge respecto de un punto en cuestión. Se define como el producto punto entre el operador Del y un campo vectorial:
\label{equation:1_ondas_electromagneticas/1_ondas_electromagneticas:3941915b-a559-49de-9a29-c448b9e74a1d}\begin{equation}
\nabla \cdot \vec{f}= \frac{\partial f_x}{\partial x} + \frac{\partial f_y}{\partial y} + \frac{\partial f_z}{\partial z}
\end{equation}
\sphinxAtStartPar
Por ejemplo:

\noindent{\hspace*{\fill}\sphinxincludegraphics[width=0.700\linewidth]{{divergence}.jpg}\hspace*{\fill}}

\sphinxAtStartPar
(a) \(\nabla\cdot\vec{f} \gt 0\)

\sphinxAtStartPar
(b) \(\nabla\cdot\vec{f} = 0\)

\sphinxAtStartPar
(c) \(\nabla\cdot\vec{f} \gt 0\)

\sphinxAtStartPar
\sphinxstylestrong{Operador Rotacional.}

\sphinxAtStartPar
Se aplica a campos vectoriales. Es una medida de cuanto un campo vectorial rota respecto de un punto en cuestión. Se define como el producto cruz entre el operador Del y un campo vectorial:
\label{equation:1_ondas_electromagneticas/1_ondas_electromagneticas:795b1438-50ba-4045-83c4-ae44b7fd71a4}\begin{equation}
\nabla \times \vec{f}= \left(\frac{\partial f_z}{\partial y}- \frac{\partial f_y}{\partial z}\right)\hat{x}+\left(\frac{\partial f_x}{\partial z}- \frac{\partial f_z}{\partial x}\right)\hat{y}+\left(\frac{\partial f_y}{\partial x}- \frac{\partial f_x}{\partial y}\right)\hat{z}
\end{equation}
\sphinxAtStartPar
Por ejemplo:

\noindent{\hspace*{\fill}\sphinxincludegraphics[width=600\sphinxpxdimen]{{curl}.jpg}\hspace*{\fill}}

\sphinxAtStartPar
(a) \(\nabla\times\vec{f} \gt 0\)

\sphinxAtStartPar
(b) \(\nabla\times\vec{f} \gt 0\)
\begin{quote}

\sphinxAtStartPar
En la figura anterior (divergente), \(\nabla\times\vec{f} = 0\) en todos los casos.
\end{quote}


\subsection{Ejemplos de uso de operadores diferenciales}
\label{\detokenize{1_ondas_electromagneticas/1_ondas_electromagneticas:ejemplos-de-uso-de-operadores-diferenciales}}
\sphinxAtStartPar
Los operadores diferenciales permiten una descripción más compacta en de las formuals basadas en ecuaciones diferenciales parciales.

\noindent{\hspace*{\fill}\sphinxincludegraphics[width=500\sphinxpxdimen]{{mass_conservation}.png}\hspace*{\fill}}

\sphinxAtStartPar
Un ejemplo conocido es el caso de la ecuación de conservación de masa en su forma diferencial.
\begin{equation*}
\begin{split}\frac{\partial \rho}{\partial t}+\frac{\partial (\rho u)}{\partial x} +\frac{\partial (\rho v)}{\partial y}+\frac{\partial (\rho w)}{\partial z}=0.\end{split}
\end{equation*}
\sphinxAtStartPar
Usando el operador Del,
\begin{equation*}
\begin{split}\frac{\partial \rho}{\partial t} + \nabla\cdot\left(\rho\vec{V}\right)
=0,\end{split}
\end{equation*}
\sphinxAtStartPar
o bien:
\begin{equation*}
\begin{split}\frac{\partial \rho}{\partial t} + \vec{V}\cdot\nabla\rho+ \rho\left(\nabla\cdot\vec{V}\right)
=0.\end{split}
\end{equation*}

\section{Ecuaciones de Maxwell}
\label{\detokenize{1_ondas_electromagneticas/1_ondas_electromagneticas:ecuaciones-de-maxwell}}

\subsection{Ley de Gauss}
\label{\detokenize{1_ondas_electromagneticas/1_ondas_electromagneticas:ley-de-gauss}}
\sphinxAtStartPar
\sphinxstyleemphasis{El flujo de campo electrico a través de una superficie cerrada es proporcional a la carga eléctrica, \(\rho\), contenida dentro de esta superficie.}

\sphinxAtStartPar
En su forma diferencial:
\label{equation:1_ondas_electromagneticas/1_ondas_electromagneticas:ffa31127-4d5c-48ff-9fa8-10c025ef116b}\begin{equation}
\nabla\cdot\left(\varepsilon_0\vec{E}\right) = \rho
\end{equation}
\sphinxAtStartPar
Donde:
\begin{itemize}
\item {} 
\sphinxAtStartPar
\(\vec{E}\), es el \sphinxstylestrong{campo eléctrico} (se mide en unidades de \(\mathrm{V/m}\)).

\item {} 
\sphinxAtStartPar
\(\varepsilon_0 = 8.854\times10^{-12}\) \(\mathrm{F/m}\), es la \sphinxstylestrong{permitividad} en el vacío.

\end{itemize}
\begin{quote}

\sphinxAtStartPar
Un campo eléctrico diveregente(convergente) es el resultado de una carga eléctrica positiva(negativa) que actúa como fuente(sumidero)
\end{quote}


\subsection{Ley de continuidad del campo magnético}
\label{\detokenize{1_ondas_electromagneticas/1_ondas_electromagneticas:ley-de-continuidad-del-campo-magnetico}}
\sphinxAtStartPar
\sphinxstyleemphasis{No existen cargas magnéticas que den lugar a un campo magnético}

\sphinxAtStartPar
En su forma diferencial:
\label{equation:1_ondas_electromagneticas/1_ondas_electromagneticas:da4a15d6-fc0c-4521-b458-5c4b70f5485c}\begin{equation}
\nabla\cdot\left(\mu_0\vec{H}\right) = 0
\end{equation}
\sphinxAtStartPar
Donde:
\begin{itemize}
\item {} 
\sphinxAtStartPar
\(\vec{H}\), es la \sphinxstylestrong{intensidad de campo magnetico} (se mide en unidades de \(\mathrm{A/m}\)).

\item {} 
\sphinxAtStartPar
\(\mu_0 = 4\pi\times10^{-7}\) \(\mathrm{N/A^2}\), es la \sphinxstylestrong{permeabilidad} magnética en el vacío.

\end{itemize}
\begin{quote}

\sphinxAtStartPar
A diferencia del campo eléctrico, el campo magnético es continuo. Es decir no tiene fuentes ni sumideros
\end{quote}

\sphinxAtStartPar
Es común en los textos de física ver las ecuaciones de campo magnético respresentadas en base al \sphinxstylestrong{vector campo magnético} \(\vec{B}\) y no a \(\vec{H}\) Esto, porque \(\vec{B}\) representa la componente “experimentalmente medible” del campo magnético y la que efectivamente afecta a las cargas en movimiento. Ambas variables se relaciona mediante \(\vec{B} =\mu_0\vec{H}\).

\sphinxAtStartPar
De igual manera, el análogo del campo eléctrico se denomina \sphinxstylestrong{desplazamiento eléctrico}, y se relaciona con el campo eléctrico mediante \(\vec{D}=\varepsilon_0\vec{E}\). En este caso, la componente “experimentalmente medible” es \(\vec{E}\) y, por ende, es formalmente utilizada en los textos de física.


\subsection{Ley de Faraday}
\label{\detokenize{1_ondas_electromagneticas/1_ondas_electromagneticas:ley-de-faraday}}
\sphinxAtStartPar
\sphinxstyleemphasis{Un campo magnético variable en el tiempo induce un campo eléctrico rotacional}
\label{equation:1_ondas_electromagneticas/1_ondas_electromagneticas:2324d2fb-0dce-425d-aa9f-032560abfe02}\begin{equation}
\nabla\times\vec{E} = -\mu_0\frac{\partial \vec{H}}{\partial t}
\end{equation}\begin{itemize}
\item {} 
\sphinxAtStartPar
Notar que el campo magnético debe ser variable en el tiempo para poder inducir una corriente.

\end{itemize}


\subsection{Ley de Ampere}
\label{\detokenize{1_ondas_electromagneticas/1_ondas_electromagneticas:ley-de-ampere}}
\sphinxAtStartPar
\sphinxstyleemphasis{Una corriente eléctrica induce un campo magnético rotacional alrededor de ella}
\label{equation:1_ondas_electromagneticas/1_ondas_electromagneticas:0b577b82-cc79-45b7-9c98-a41cd6ae3f38}\begin{equation}
\nabla\times\vec{H} = \vec{J}
\end{equation}
\sphinxAtStartPar
Donde:
\begin{itemize}
\item {} 
\sphinxAtStartPar
\(\vec{J}\), es la \sphinxstylestrong{densidad de corriente}  eléctrica (se mide en unidades de \(\mathrm{A/m^2}\)).

\end{itemize}

\sphinxAtStartPar
La ley de Ampere y de Faraday son la base del funcionamiento de motores de inducción, motores DC, transformadores, etc.

\begin{sphinxuseclass}{cell}\begin{sphinxVerbatimInput}

\begin{sphinxuseclass}{cell_input}
\begin{sphinxVerbatim}[commandchars=\\\{\}]
\PYG{k+kn}{from} \PYG{n+nn}{IPython}\PYG{n+nn}{.}\PYG{n+nn}{display} \PYG{k+kn}{import} \PYG{n}{YouTubeVideo}
\PYG{n}{YouTubeVideo}\PYG{p}{(}\PYG{l+s+s1}{\PYGZsq{}}\PYG{l+s+s1}{CWulQ1ZSE3c}\PYG{l+s+s1}{\PYGZsq{}}\PYG{p}{,} \PYG{n}{width}\PYG{o}{=}\PYG{l+m+mi}{600}\PYG{p}{,} \PYG{n}{height}\PYG{o}{=}\PYG{l+m+mi}{400}\PYG{p}{,}  \PYG{n}{playsinline}\PYG{o}{=}\PYG{l+m+mi}{0}\PYG{p}{,} \PYG{n}{start}\PYG{o}{=}\PYG{l+m+mi}{42}\PYG{p}{)}
\end{sphinxVerbatim}

\end{sphinxuseclass}\end{sphinxVerbatimInput}
\begin{sphinxVerbatimOutput}

\begin{sphinxuseclass}{cell_output}
\noindent\sphinxincludegraphics{{1_ondas_electromagneticas_33_0}.jpg}

\end{sphinxuseclass}\end{sphinxVerbatimOutput}

\end{sphinxuseclass}

\subsection{Corrección de la ley de Ampere}
\label{\detokenize{1_ondas_electromagneticas/1_ondas_electromagneticas:correccion-de-la-ley-de-ampere}}
\sphinxAtStartPar
Es posible demostrar que, para un campo vectorial \(\vec{f}\), se cumple la siguiente identidad
\begin{equation*}
\begin{split}\nabla\cdot\nabla\times\vec{f} = 0,\end{split}
\end{equation*}
\sphinxAtStartPar
Analicemos el divergente en la ley de Faraday
\begin{equation*}
\nabla\cdot \nabla\times\vec{E} = -\nabla\cdot\mu_0\frac{\partial \vec{H}}{\partial t} \Rightarrow -\frac{\partial }{\partial t}\left(\nabla\cdot\mu_0\vec{H}\right) =0
\end{equation*}
\sphinxAtStartPar
La relación se cumple por la ley de continuidad del campo magnético

\sphinxAtStartPar
Por otro lado, el divergente en la ley de Ampere:
\begin{equation*}
\nabla\cdot \nabla\times\vec{H} = \nabla\cdot\vec{J} \Rightarrow \nabla\cdot\vec{J} = 0.
\end{equation*}
\sphinxAtStartPar
Sin embargo, por la ley de conservación de masa:
\begin{equation*}
\nabla\cdot\vec{J} = - \frac{\partial \rho}{\partial t}
\end{equation*}
\sphinxAtStartPar
Claramente, la ecuación de Ampere no está completa. La corrección, fue propuesta por James Maxwell
\label{equation:1_ondas_electromagneticas/1_ondas_electromagneticas:f244fa1b-e6e4-4b00-a963-8eb807f1fb60}\begin{equation}
\nabla\times\vec{H} = \vec{J} + \varepsilon_0\frac{\partial\vec{E}}{\partial t}
\end{equation}
\sphinxAtStartPar
El último término es conocido como la \sphinxstyleemphasis{corriente de desplazamiento de Maxwell.}
\begin{quote}

\sphinxAtStartPar
A través de esta contribución James C. Maxwell logra unificar las teorías de electricidad, magnetismo y la luz en un solo fenómeno, \sphinxstylestrong{las ondas electromagnéticas}.
\end{quote}


\section{Ondas electromagnéticas}
\label{\detokenize{1_ondas_electromagneticas/1_ondas_electromagneticas:ondas-electromagneticas}}

\subsection{Ondas electromagnéticas en el vacío}
\label{\detokenize{1_ondas_electromagneticas/1_ondas_electromagneticas:ondas-electromagneticas-en-el-vacio}}
\sphinxAtStartPar
En el vacío, no existen cargas eléctricas (\(\rho=0\)) ni corrientes eléctricas (\(\vec{J} = 0\)), y por lo tanto las ecuaciones de Maxwell son:
\begin{align*}
\nabla\cdot\vec{E} &= 0 \\
\nabla\cdot\vec{H} &= 0 \\
\nabla\times\vec{E} &= -\mu_0\frac{\partial \vec{H}}{\partial t} \\
\nabla\times\vec{H} &= \varepsilon_0\frac{\partial \vec{E}}{\partial t}
\end{align*}
\sphinxAtStartPar
Analicemos el rotacional sobre la ley de faraday
\begin{equation*}
\nabla\times\nabla\times\vec{E} = -\mu_0\frac{\partial}{\partial t}\left(\nabla\times \vec{H}\right)
\end{equation*}
\sphinxAtStartPar
Mediante la identidad,
\begin{equation*}
\begin{split}\nabla\times\nabla\times\vec{E} = \nabla\left(\nabla\cdot\vec{E}\right) - \nabla\cdot\nabla\vec{E},\end{split}
\end{equation*}
\sphinxAtStartPar
y la ley de Gauss \(\nabla\cdot\vec{E} = 0\), podemos demostrar:
\begin{equation*}
\nabla^2\vec{E} = \mu_0\frac{\partial}{\partial t}\left(\nabla\times \vec{H}\right)
\end{equation*}
\sphinxAtStartPar
Finalmente, mediante la ley de Ampere modificada, determinamos:
\begin{equation*}
\nabla^2\vec{E} - \varepsilon_0\mu_0\frac{\partial^2\vec{E}}{\partial t^2} = 0
\end{equation*}
\sphinxAtStartPar
Esta es la ecuación de onda en su forma tridimensional, la cual acepta soluciones del tipo:
\begin{equation*}
\begin{split}E_0 e^{i\left(\vec{k}\cdot\vec{r} - \omega t\right)} \hat{e},\end{split}
\end{equation*}
\sphinxAtStartPar
donde:
\begin{itemize}
\item {} 
\sphinxAtStartPar
\(\vec{k}\) es el vector de onda

\item {} 
\sphinxAtStartPar
\(\vec{r}\) es un vector de posición

\item {} 
\sphinxAtStartPar
\(\omega\) es la frecuencia angular (rad/s)

\item {} 
\sphinxAtStartPar
\(E_0\) es la amplitud

\item {} 
\sphinxAtStartPar
\(\hat{e}\) es la dirección de oscilación de la onda.

\end{itemize}

\sphinxAtStartPar
Reemplazando esta solución en la ecuación de onda, determinamos la \sphinxstylestrong{relación de dispersión} entre la \sphinxstylestrong{magnitud del vector de onda en el vacío}, \(k_0 = |\vec{k}|\), y la frecuencia angular:
\label{equation:1_ondas_electromagneticas/1_ondas_electromagneticas:dd1f4072-7140-434f-a81b-d2c9957fc57c}\begin{equation}
k_0 = \frac{\omega}{c_0},
\end{equation}
\sphinxAtStartPar
donde
\begin{equation*}
\begin{split}c_0 = \frac{1}{\sqrt{\varepsilon_0\mu_0}} \approx 3.00\times10^8~\mathrm{m/s,}\end{split}
\end{equation*}
\sphinxAtStartPar
es la velocidad de la luz.

\sphinxAtStartPar
En general, estamos más familizarizados con los conceptos de \sphinxstylestrong{longitud de onda \(\lambda\)} y \sphinxstylestrong{frecuencia \(\nu\)}, para caracterizar ondas electromagnéticas. Estas variables se relacionan con el vector de onda y la frecuencia mediante:
\label{equation:1_ondas_electromagneticas/1_ondas_electromagneticas:7be6f4ec-fa31-4186-a78c-9c70866688b5}\begin{equation}
k_0 = \frac{2\pi}{\lambda}, ~~ \omega = 2\pi\nu
\end{equation}
\sphinxAtStartPar
De igual forma, mediante la relación de dispersión, podemos establecer la siguiente relación entre la longitud de onda y la frecuencia:
\label{equation:1_ondas_electromagneticas/1_ondas_electromagneticas:05129d32-7f10-4608-80c7-cbf79ab2717c}\begin{equation}
\lambda\nu = c_0
\end{equation}\begin{quote}

\sphinxAtStartPar
Esto quiere decir, que un punto \(\vec{r}\) arbitrario de la onda, viaja en el vacío a una velocidad constante \(c_0\), \sphinxstylestrong{independendiente de su frecuencia}.
\end{quote}

\sphinxAtStartPar
\sphinxstylestrong{El vector de onda representa la dirección de propagación de la onda}. A partir de la ley de Gauss, podemos demostrar:
\label{equation:1_ondas_electromagneticas/1_ondas_electromagneticas:998ac11a-5f56-45f1-9493-526ab6fc364b}\begin{equation}
\vec{k} \cdot\hat{e} = 0
\end{equation}
\sphinxAtStartPar
Es decir, el campo eléctrico oscila en dirección perpendicular a la dirección de propagación.
\begin{quote}

\sphinxAtStartPar
En otras palabras, el campo electrico representa una \sphinxstylestrong{onda transversal.}
\end{quote}

\noindent{\hspace*{\fill}\sphinxincludegraphics[width=400\sphinxpxdimen]{{onda_transversal}.png}\hspace*{\fill}}



\sphinxAtStartPar
En general, tenemos dos tipos de ondas, transversales, y longitudinales

\begin{sphinxuseclass}{cell}\begin{sphinxVerbatimInput}

\begin{sphinxuseclass}{cell_input}
\begin{sphinxVerbatim}[commandchars=\\\{\}]
\PYG{k+kn}{from} \PYG{n+nn}{IPython}\PYG{n+nn}{.}\PYG{n+nn}{display} \PYG{k+kn}{import} \PYG{n}{IFrame}\PYG{p}{,} \PYG{n}{display}
\PYG{n}{display}\PYG{p}{(}\PYG{n}{IFrame}\PYG{p}{(}\PYG{l+s+s1}{\PYGZsq{}}\PYG{l+s+s1}{https://www.geogebra.org/material/iframe/id/auyft2pd/width/640/height/480/border/888888/sfsb/true/smb/false/stb/false/stbh/false/ai/false/asb/false/sri/true/rc/false/ld/false/sdz/false/ctl/false}\PYG{l+s+s1}{\PYGZsq{}}\PYG{p}{,}\PYG{l+s+s1}{\PYGZsq{}}\PYG{l+s+s1}{700px}\PYG{l+s+s1}{\PYGZsq{}}\PYG{p}{,} \PYG{l+s+s1}{\PYGZsq{}}\PYG{l+s+s1}{450px}\PYG{l+s+s1}{\PYGZsq{}}\PYG{p}{)}\PYG{p}{)}
\end{sphinxVerbatim}

\end{sphinxuseclass}\end{sphinxVerbatimInput}
\begin{sphinxVerbatimOutput}

\begin{sphinxuseclass}{cell_output}
\begin{sphinxVerbatim}[commandchars=\\\{\}]
\PYGZlt{}IPython.lib.display.IFrame at 0x7f09d85397c0\PYGZgt{}
\end{sphinxVerbatim}

\end{sphinxuseclass}\end{sphinxVerbatimOutput}

\end{sphinxuseclass}
\sphinxAtStartPar
Debido a que el campo eléctrico toma la forma \(\vec{E} = E_0 e^{i\left(\vec{k}\cdot\vec{r} - \omega t\right)} \hat{e},\)\$  decimos que se comporta como una \sphinxstylestrong{onda plana}, debido a que el campo es constante sobre un plano perpendicular a la dirección de propagación

\noindent{\hspace*{\fill}\sphinxincludegraphics[width=400\sphinxpxdimen]{{planewave}.png}\hspace*{\fill}}

\sphinxAtStartPar
De igual forma, podemos demostrar que la intensidad de campo magnético en el vacío también satisface la ecuación de onda:
\begin{equation*}
\nabla^2\vec{H} - \varepsilon_0\mu_0\frac{\partial^2\vec{H}}{\partial t^2} = 0
\end{equation*}
\sphinxAtStartPar
Utilizando un tratamiento similar al de \(\vec{E}\), concluiremos que \(\vec{H}\):
\begin{itemize}
\item {} 
\sphinxAtStartPar
Se comporta como una onda de la forma \(H_0 e^{ i\left(\vec{k}\cdot\vec{r} - \omega t\right)} \hat{h}\)

\item {} 
\sphinxAtStartPar
Se mueve en el vacio a una velocidad constante \(c_0 \approx 3.00\times10^8~\mathrm{m/s}\).

\item {} 
\sphinxAtStartPar
Es una onda transversal (\(\vec{k}\cdot\hat{h} = 0\), por la ley de continuidad de \(\vec{H}\)).

\end{itemize}

\sphinxAtStartPar
Finalmente, mediante la ley de Faraday (o Ampere), deducimos:
\begin{equation*}
\hat{h}{H}_0 = \frac{E_0}{Z_0}\left(\hat{k}\times\hat{e}\right),
\end{equation*}
\sphinxAtStartPar
donde \(Z_0 = \sqrt{\frac{\mu_0}{\varepsilon_0}}\), es la \sphinxstylestrong{impedancia del vacío} (se mide en \(\Omega\)).

\sphinxAtStartPar
De esta relación concluímos:
\begin{enumerate}
\sphinxsetlistlabels{\arabic}{enumi}{enumii}{}{.}%
\item {} 
\sphinxAtStartPar
El campo eléctrico magnético y el vector de onda son mutuamente perpendiculares (\(H\perp E\perp k\))

\item {} 
\sphinxAtStartPar
La amplitud de la intensidad de campo magnético y del campo eléctrico, estan relacionadas por: \({H}_0 = \frac{E_0}{Z_0}\)

\end{enumerate}

\sphinxAtStartPar
En resumen:
\begin{enumerate}
\sphinxsetlistlabels{\arabic}{enumi}{enumii}{}{.}%
\item {} 
\sphinxAtStartPar
En el vacío, \(\vec{E}\) y \(\vec{H}\) se comportan como ondas trasversales de la forma \(\propto e^{ i\left(\vec{k}\cdot\vec{r} - \omega t\right)}\).

\item {} 
\sphinxAtStartPar
El vector de onda \(\vec{k}\) representa la dirección de propagación de \(\vec{E}\) y \(\vec{H}\).

\item {} 
\sphinxAtStartPar
\(\vec{E}\) y \(\vec{H}\) se propagan a una velocidad constante \(c_0 = \frac{1}{\sqrt{\mu_0\varepsilon_0}} \approx 3.00\times10^8\) m/s.

\item {} 
\sphinxAtStartPar
La magnitud del vector de onda en el vacío, \(k_0\), y la frecuencia angular, \(\omega\), están relacionadas por \(k_0 = \omega/c_0\)

\item {} 
\sphinxAtStartPar
\(\vec{E}\), \(\vec{H}\) y \(\vec{k}\) son mutuamente perpendiculares.

\item {} 
\sphinxAtStartPar
Las amplitudes de \(\vec{E}\) y \(\vec{H}\) están asociadas por la relación \({H}_0 = \frac{E_0}{Z_0}\), donde \(Z_0 = \sqrt{\frac{\mu_0}{\varepsilon_0}}\).

\end{enumerate}

\noindent{\hspace*{\fill}\sphinxincludegraphics[width=400\sphinxpxdimen]{{EM-Wave}.gif}\hspace*{\fill}}




\subsection{Vector de Poynting}
\label{\detokenize{1_ondas_electromagneticas/1_ondas_electromagneticas:vector-de-poynting}}
\sphinxAtStartPar
El vector de Poynting, \(\vec{S}\), representa el flujo de energía electromagnética por unidad de área. Está dado por la relación:
\label{equation:1_ondas_electromagneticas/1_ondas_electromagneticas:5ab5c704-f006-4252-8fb4-c5f1ed66d3ab}\begin{equation}
\langle\vec{S}\rangle = \frac{1}{2}\mathrm{Re}\left(\vec{E}\times\vec{H}^*\right),
\end{equation}
\sphinxAtStartPar
donde \(\langle\cdots\rangle\) reprensenta el promedio temporal, y el símbolo “\(*\)” reprenta el complejo conjugado.

\sphinxAtStartPar
Consideremos, por ejemplo, el vector de Poynting para una onda plana que se propaga en en el vacio:
\begin{align*}
\langle\vec{S}\rangle &= \frac{1}{2}\mathrm{Re}\left(\vec{E}\times\vec{H}^*\right) \\
&=\frac{1}{2}\mathrm{Re}\left[E_0 e^{i\left(k_0\hat{k}\cdot\vec{r} - \omega t\right)}\frac{E_0}{Z_0}e^{-i\left(k_0\hat{k}\cdot\vec{r} - \omega t\right)}\right] \left(\hat{e}\times\hat{h}\right)\\
&=\frac{E_0^2}{2Z_0} \hat{k}
\end{align*}
\sphinxAtStartPar
Así, el flujo de energía que transporta onda electromagnética en el vacío es \(\frac{E_0^2}{2Z_0}\)

\sphinxAtStartPar
Ahora con los conceptos de ondas electromagnéticas y vector de Poynting, pasemos a revisar este video explicativo de como fluye la energía en las redes eléctricas

\begin{sphinxuseclass}{cell}\begin{sphinxVerbatimInput}

\begin{sphinxuseclass}{cell_input}
\begin{sphinxVerbatim}[commandchars=\\\{\}]
\PYG{k+kn}{from} \PYG{n+nn}{IPython}\PYG{n+nn}{.}\PYG{n+nn}{display} \PYG{k+kn}{import} \PYG{n}{YouTubeVideo}
\PYG{n}{YouTubeVideo}\PYG{p}{(}\PYG{l+s+s1}{\PYGZsq{}}\PYG{l+s+s1}{bHIhgxav9LY}\PYG{l+s+s1}{\PYGZsq{}}\PYG{p}{,} \PYG{n}{width}\PYG{o}{=}\PYG{l+m+mi}{600}\PYG{p}{,} \PYG{n}{height}\PYG{o}{=}\PYG{l+m+mi}{400}\PYG{p}{,}  \PYG{n}{playsinline}\PYG{o}{=}\PYG{l+m+mi}{0}\PYG{p}{,} \PYG{n}{start}\PYG{o}{=}\PYG{l+m+mi}{42}\PYG{p}{)}
\end{sphinxVerbatim}

\end{sphinxuseclass}\end{sphinxVerbatimInput}
\begin{sphinxVerbatimOutput}

\begin{sphinxuseclass}{cell_output}
\noindent\sphinxincludegraphics{{1_ondas_electromagneticas_67_0}.jpg}

\end{sphinxuseclass}\end{sphinxVerbatimOutput}

\end{sphinxuseclass}

\section{Referencias}
\label{\detokenize{1_ondas_electromagneticas/1_ondas_electromagneticas:referencias}}
\sphinxAtStartPar
Griffths D., \sphinxstyleemphasis{Introduction to Electrodynamics}, 4th Ed, Pearson, 2013
\begin{itemize}
\item {} 
\sphinxAtStartPar
1.2 Diferential Calculus

\item {} 
\sphinxAtStartPar
7.2 Electromagnetic Induction

\item {} 
\sphinxAtStartPar
7.3 Maxwell’s Equations (hasta  7.3.3)

\item {} 
\sphinxAtStartPar
8 Concervation laws (solo 8.1)

\item {} 
\sphinxAtStartPar
9 Electromagnetic Waves (hasta 9.2)

\end{itemize}

\sphinxstepscope

\sphinxAtStartPar
MEC501 \sphinxhyphen{} Manejo y Conversión de Energía Solar Térmica


\chapter{Ondas electromagnéticas en la materia}
\label{\detokenize{2_ondas_EM_en_la_materia/2_ondas_EM_en_la_materia:ondas-electromagneticas-en-la-materia}}\label{\detokenize{2_ondas_EM_en_la_materia/2_ondas_EM_en_la_materia::doc}}
\sphinxAtStartPar

Profesor: Francisco Ramírez CueSvas
Fecha: 19 de Agosto 2022


\section{Ecuaciones de Maxwell en un medio}
\label{\detokenize{2_ondas_EM_en_la_materia/2_ondas_EM_en_la_materia:ecuaciones-de-maxwell-en-un-medio}}
\sphinxAtStartPar
La materia esta compuesta por cargas (electrónes, átomos, moléculas). Por lo tanto, a diferencia del vacío, la densidad de carga (\(\rho\)) y de corriente (\(\vec{J}\)) eléctricas están presentes en las ecuaciones de Maxwell:
\begin{align*}
\nabla\cdot\vec{E} &= \frac{\rho}{\varepsilon_0} \\
\nabla\cdot\vec{B} &= 0 \\
\nabla\times\vec{E} &= -\mu_0\frac{\partial \vec{H}}{\partial t} \\
\nabla\times\vec{H} &= \vec{J} + \varepsilon_0\frac{\partial \vec{E}}{\partial t}
\end{align*}\begin{quote}

\sphinxAtStartPar
En general, existe un tercer término asociado con la polarización magnética del material. Sin embargo, en este curso veremos solo materiales paramagnéticos y, por lo tanto, este término será ignorado.
\end{quote}

\sphinxAtStartPar
Asumiendo un medio homogéneo, y mediante la relación:
\label{equation:2_ondas_EM_en_la_materia/2_ondas_EM_en_la_materia:b2c40785-0c66-4052-8e9d-99a169aba12c}\begin{equation}
\vec{D} = \varepsilon_0\varepsilon\vec{E}
\end{equation}
\sphinxAtStartPar
donde \(\vec{D}\) es el desplazamiento eléctrico, y \(\varepsilon = \varepsilon' + i\varepsilon''\), es la \sphinxstylestrong{constante dieléctrica} compleja;

\sphinxAtStartPar
podemos demostrar que las ecuaciones de Maxwell se pueden reescribir en la forma:
\begin{align*}
\nabla\cdot\vec{E} &= 0 \\
\nabla\cdot\vec{H} &= 0 \\
\nabla\times\vec{E} &= -\mu_0\frac{\partial \vec{H}}{\partial t} \\
\nabla\times\vec{H} &= \varepsilon_0 \varepsilon\frac{\partial\vec{E}}{\partial t}
\end{align*}
\sphinxAtStartPar
Estas ecuaciones tiene la misma forma que las ecuaciones de Maxwell en el vacío, y por lo tanto todas las conclusiones anteriores aplican a este caso.

\sphinxAtStartPar
La gran diferencia está en la relación de dispersión. En este caso:
\label{equation:2_ondas_EM_en_la_materia/2_ondas_EM_en_la_materia:2137b04e-17a0-42d3-a2a1-38b8d139a2e6}\begin{equation}
k = N \frac{\omega}{c_0}
\end{equation}
\sphinxAtStartPar
donde \(N = \sqrt{\varepsilon} = n +i\kappa\), es el \sphinxstylestrong{índice de refracción complejo}. En general \(n\) se conoce como el \sphinxstylestrong{índice de refracción}, y \(\kappa\) como \sphinxstylestrong{extinsión}.
\begin{quote}

\sphinxAtStartPar
Notar que la velocidad de la onda también cambia a \(c = c_0/n\)
\end{quote}

\sphinxAtStartPar
Igualmente la relación entre \(H_0\) y \(E_0\), es de la forma
\begin{equation*}
\begin{split}H_0 = \frac{E_0}{Z_0Z_r},\end{split}
\end{equation*}
\sphinxAtStartPar
donde \(Z_r = \sqrt{\frac{1}{\varepsilon}}\) es la \sphinxstylestrong{impedancia relativa}.
\begin{quote}

\sphinxAtStartPar
Notar que para materiales paramagnéticos,
\begin{equation*}Z_r = \sqrt{\frac{1}{\varepsilon}} = \frac{1}{N}\end{equation*}\end{quote}

\sphinxAtStartPar
\sphinxstylestrong{¿Qué representa la constante dielectrica compleja?}

\sphinxAtStartPar
Los materiales están compuestos de átomos, con un núcleo positivo y electrones negativos. Estos electrones interactúan con los átomos de distintas formas; algunos orbitan alrededor del núcleo mientras que otros se mueven libremente por el material. Así, podemos separar las cargas eléctricas en dos tipos: \sphinxstylestrong{cargas ligadas}, y \sphinxstylestrong{cargas libres}.

\noindent{\hspace*{\fill}\sphinxincludegraphics[width=400\sphinxpxdimen]{{atomic_lattice}.png}\hspace*{\fill}}

\sphinxAtStartPar
La interacción de las onda electromagnéticas con las cargas ligadas induce \sphinxstylestrong{polarización}, es decir, el nucleo y el electrón se polarizan, oscilando en sincronía con el campo externo. Esta respuesta está representada por la parte real de la constante dielectrica (\(\varepsilon'\)).

\noindent{\hspace*{\fill}\sphinxincludegraphics[width=550\sphinxpxdimen]{{constante_dielectrica}.png}\hspace*{\fill}}

\sphinxAtStartPar
Las ondas electromagnéticas aceleran las cargas libres, generando \sphinxstylestrong{corrientes eléctricas inducidas}. Algunas cargas libres móbiles colicionan con otros electrónes o núcleos, disipando energía. Esta respuesta está representada por la parte imaginaria de la constante dieléctrica (\(\varepsilon''\)).

\sphinxAtStartPar
Esta discipación de energía está representada por la resistencia eléctrica, y es la reponsable de la generación de calor en metaeles.

\sphinxAtStartPar
De hecho, la conductividad eléctrica \(\sigma\) en la ley de Ohm, \(\vec{J} = \sigma\vec{E}\), está relacionada con la parte imaginaria de la constante dielectrica por:
\label{equation:2_ondas_EM_en_la_materia/2_ondas_EM_en_la_materia:310c74db-4c39-4b08-918b-c366abdd920d}\begin{equation}
\sigma = \varepsilon_0\omega\varepsilon''
\end{equation}
\noindent{\hspace*{\fill}\sphinxincludegraphics[width=300\sphinxpxdimen]{{light_bulb}.jpg}\hspace*{\fill}}

\sphinxAtStartPar
\sphinxstylestrong{¿Que significa que el vector de onda sea complejo?}

\sphinxAtStartPar
Analicemos la solución general de la ecuación de onda:
\begin{align*}
\vec{E} &= E_0 e^{i\left(\vec{k}\cdot\vec{r} - \omega t\right)} \hat{e} \\
&= E_0 e^{i\left(Nk_0\hat{k}\cdot\vec{r} - \omega t\right)} \hat{e} \\
&= E_0 e^{i\left(nk_0\hat{k}\cdot\vec{r} - \omega t\right)}e^{-\kappa k_0\left(\hat{k}\cdot\vec{r}\right)} \hat{e}
\end{align*}
\noindent{\hspace*{\fill}\sphinxincludegraphics[width=300\sphinxpxdimen]{{decaying_wave}.png}\hspace*{\fill}}

\sphinxAtStartPar
Lo que notamos es, mientas que el índice de refracción \(n\) representa el \sphinxstylestrong{cambio en la oscilación espacial de la onda}, la extinsión \(\kappa\) indica un \sphinxstylestrong{decaimiento en la amplitud}.

\sphinxAtStartPar
En resumen, en materiales paramagnéticos:
\begin{enumerate}
\sphinxsetlistlabels{\arabic}{enumi}{enumii}{}{.}%
\item {} 
\sphinxAtStartPar
\(\vec{E}\) y \(\vec{H}\) se comportan como ondas trasversales de la forma \(\propto e^{ i\left(\vec{k}\cdot\vec{r} - \omega t\right)}\).

\item {} 
\sphinxAtStartPar
La relación de dispersión esta dada por \(k = N\frac{\omega}{c_0},\)
donde \(N = n + i\kappa\) es el índice de refracción complejo.

\item {} 
\sphinxAtStartPar
\(N =\sqrt{\varepsilon} =\sqrt{\varepsilon'+i\varepsilon''}\), donde \(\varepsilon\) es la constante dieléctrica

\item {} 
\sphinxAtStartPar
\(\vec{E}\) y \(\vec{H}\) se propagan a una velocidad constante \(c = c_0/n\)

\item {} 
\sphinxAtStartPar
\(\kappa\) representa la extinsión de la onda en el espacio.

\item {} 
\sphinxAtStartPar
\(\vec{E}\), \(\vec{H}\) y \(\vec{k}\) son mutuamente perpendiculares.

\item {} 
\sphinxAtStartPar
Las amplitudes de \(\vec{E}\) y \(\vec{H}\) están asociadas por la relación \({H}_0 = \frac{E_0}{Z_0Z_r}\), donde \(Z_r = \frac{1}{\sqrt{\varepsilon}}\).

\end{enumerate}

\noindent{\hspace*{\fill}\sphinxincludegraphics[width=400\sphinxpxdimen]{{em_wave_decaying}.jpg}\hspace*{\fill}}




\subsection{Vector de Poynting}
\label{\detokenize{2_ondas_EM_en_la_materia/2_ondas_EM_en_la_materia:vector-de-poynting}}
\sphinxAtStartPar
El vector de Poynting, \(\vec{S}\), representa el flujo de energía electromagnética por unidad de área. Está dado por la relación:
\label{equation:2_ondas_EM_en_la_materia/2_ondas_EM_en_la_materia:e263de16-8406-447f-bf47-196e49763a2d}\begin{equation}
\langle\vec{S}\rangle = \frac{1}{2}\mathrm{Re}\left(\vec{E}\times\vec{H}^*\right),
\end{equation}
\sphinxAtStartPar
donde \(\langle\cdots\rangle\) reprensenta el promedio en un periodo, y \(^*\) reprenta el complejo conjugado.

\sphinxAtStartPar
Consideremos, por ejemplo, el vector de Poynting para una onda plana que se propaga en un material con índice de refracción \(N = n+i\kappa\):
\begin{align*}
\langle\vec{S}\rangle &= \frac{1}{2}\mathrm{Re}\left[\vec{E}\times\vec{H}^*\right] \\
&=\frac{1}{2}\mathrm{Re}\left[E_0 e^{i\left(\vec{k}\cdot\vec{r} - \omega t\right)}H_0^* e^{-i\left(\vec{k}^*\cdot\vec{r} - \omega t\right)}\left(\hat{e}\times\hat{h}\right)\right] \\
&=\frac{1}{2}\mathrm{Re}\left[\frac{E_0^2}{Z_0Z_r^*} e^{i\left(k_0N\hat{k}\cdot\vec{r} - \omega t\right)}e^{-i\left(k_0N^*\hat{k}\cdot\vec{r} - \omega t\right)} \hat{k}\right] \\
&=\frac{1}{2}\mathrm{Re}\left[\frac{E_0^2}{Z_0Z_r^*}  \hat{k}\right]e^{-2k_0\kappa\left(\hat{k}\cdot\vec{r}\right)} \\
&=\mathrm{Re}\left[N^*\hat{k}\right]\frac{E_0^2}{2Z_0}e^{-\alpha\left(\hat{k}\cdot\vec{r}\right)}
\end{align*}
\sphinxAtStartPar
El término \(\alpha = \frac{4\pi\kappa}{\lambda}\) es el \sphinxstylestrong{coeficiente de absorpción}. El inverso, \(\delta = 1/\alpha\), se denomina \sphinxstylestrong{profundidad superficial} y representa la profundidad de penetración de la onda electromagnética en un material.

\sphinxAtStartPar
Como referencia, \(\delta\sim 1000\) m en fibras ópticas a \(\lambda = 1.55\) \(\mu\)m, que es la longitud de onda utilizada en comunicación óptica. Por otro lado, en metales como la plata, oro o aluminio, \(\delta\sim 10\) nm para \(\lambda \sim 500\) nm (espectro de luz visible)


\subsection{Condiciones de borde}
\label{\detokenize{2_ondas_EM_en_la_materia/2_ondas_EM_en_la_materia:condiciones-de-borde}}
\sphinxAtStartPar
Hasta ahora hemos revisado las ecuaciones de Maxwell en un medio homogeneo, y como estas dan lugar a la solución en forma de ondas electromagnéticas.
\begin{quote}

\sphinxAtStartPar
Recordemos que para un medio con índice de refracción \(N\), la solución general es:
\begin{align*}
\vec{E} &= E_0 e^{i\left(Nk_0\hat{k}\cdot\vec{r} - \omega t\right)} \hat{e} \\
\vec{H} &= \frac{NE_0}{Z_0} e^{i\left(Nk_0\hat{k}\cdot\vec{r} - \omega t\right)} \hat{k}\times\hat{e}
\end{align*}\end{quote}

\sphinxAtStartPar
\sphinxstylestrong{¿Que sucede cuando una onda electromagnética encuentra la frontera entre dos medios distintos?}

\sphinxAtStartPar
Como toda ecuación diferencial, la solución particular de las ecuaciones de Maxwell está definida por las condiciones de borde. Estas condiciones de borde surgen al aplicar las ecuaciones de Maxwell en una frontera (cuya derivación no revisaremos aqui). En general son 4 condiciones de borde. Sin embargo, para los problemas que veremos en este curso solo se necesitan dos:
\label{equation:2_ondas_EM_en_la_materia/2_ondas_EM_en_la_materia:b3ea5a4d-6d37-4b1d-8dce-a50a48b162ac}\begin{align}
E^{\parallel}_1 - E^{\parallel}_2 &= 0 \\
H^{\parallel}_1 - H^{\parallel}_2 &= 0
\end{align}
\sphinxAtStartPar
donde \(1\) y \(2\) son dos medios distintos, y el símbolo \(\parallel\) representa la componente paralela a la interface entre los medios \(1\) y \(2\)
\begin{quote}

\sphinxAtStartPar
\sphinxstylestrong{En la interface entre dos medios \(1\) y \(2\) las componentes de \(\vec{E}\) y \(\vec{H}\) paralelas a la interface, se conservan.}
\end{quote}


\section{Reflexión y transmisión de ondas electromagnéticas en una interface}
\label{\detokenize{2_ondas_EM_en_la_materia/2_ondas_EM_en_la_materia:reflexion-y-transmision-de-ondas-electromagneticas-en-una-interface}}

\subsection{Coeficientes de Fresnel}
\label{\detokenize{2_ondas_EM_en_la_materia/2_ondas_EM_en_la_materia:coeficientes-de-fresnel}}
\sphinxAtStartPar
Consideremos el fenómeno de reflección y transmissión de una onda electromagnetética en dirección \(\hat{k}_i\) que incide sobre la interface entre dos medios 1 y 2, con índices de refracción reales \(n_1\) y \(n_2\), respectivamente

\sphinxAtStartPar
Definimos como \(\hat{n}\) al vector normal al plano de interface entre los dos medios, y como \sphinxstylestrong{plano de incidencia,} al plano formado por los vectores \(\hat{k}_i\) y \(\hat{n}\).

\sphinxAtStartPar
La dirección de la onda reflejada y transmitida está definida por los vectores \(\hat{k}_r\) y \(\hat{k}_t\), respectivamente.

\noindent{\hspace*{\fill}\sphinxincludegraphics[width=350\sphinxpxdimen]{{plano_incidencia}.png}\hspace*{\fill}}

\sphinxAtStartPar
A partir de la dirección de \(\vec{E}\) y \(\vec{H}\) respecto al plano de incidencia, podemos distunguir dos polarizaciones:
\begin{itemize}
\item {} 
\sphinxAtStartPar
Si \sphinxstylestrong{\(\vec{H}\) oscila en dirección perpendicular al plano de incidencia}, hablamos de una \sphinxstylestrong{polarización transversal magnética o TM}.

\item {} 
\sphinxAtStartPar
Si \sphinxstylestrong{\(\vec{E}\) oscila en dirección perpendicular al plano de incidencia}, hablamos de una \sphinxstylestrong{polarización transversal eléctrica o TE}.

\end{itemize}

\sphinxAtStartPar
Como ejemplo, consideremos una \sphinxstylestrong{onda transversal magnética (TM)}

\noindent{\hspace*{\fill}\sphinxincludegraphics[width=350\sphinxpxdimen]{{em_reflection}.png}\hspace*{\fill}}

\sphinxAtStartPar
La figura muestra la reflexión y transmisión de la onda visto desde el plano de incidencia

\sphinxAtStartPar
A través de las ecuaciones de Maxwell, podemos establecer la solución general para cada onda electromagnética:
\begin{align*}
\vec{E}_i &= E_i e^{ i\left(k_0n_1\hat{k}_i\cdot\vec{r} - \omega t\right)} \hat{e}_i\quad\quad \mathrm{onda~incidente}
 \\
\vec{E}_r &= E_r e^{ i\left(k_0n_1\hat{k}_r\cdot\vec{r} - \omega t\right)} \hat{e}_r\quad\quad \mathrm{onda~reflejada}
\\
\vec{E}_t &= E_t e^{ i\left(k_0n_2\hat{k}_t\cdot\vec{r} - \omega t\right)} \hat{e}_t\quad\quad 
\mathrm{onda~transmitida}
\end{align*}
\sphinxAtStartPar
donde:
\begin{eqnarray*}
\hat{k}_i &=& \hat{x}\sin\theta_i + \hat{z}\cos\theta_i 
&\quad\mathrm{y}\quad& 
\hat{e}_i &=& \hat{x}\cos\theta_i - \hat{z}\sin\theta_i
\\
\hat{k}_r &=& \hat{x}\sin\theta_r - \hat{z}\cos\theta_r
&\quad\mathrm{y}\quad& 
\hat{e}_r &=& \hat{x}\cos\theta_r + \hat{z}\sin\theta_r
\\
\hat{k}_t &=& \hat{x}\sin\theta_t + \hat{z}\cos\theta_t
&\quad\mathrm{y}\quad& 
\hat{e}_t &=& \hat{x}\cos\theta_t - \hat{z}\sin\theta_t
\end{eqnarray*}
\sphinxAtStartPar
Reemplazando en las soluciones generales,
\begin{eqnarray*}
\vec{E}_i &=& E_i e^{ ik_0n_1\left(z\cos\theta_i + x\sin\theta_i\right)}e^{-i\omega t} \left(\hat{x}\cos\theta_i -\hat{z}\sin\theta_i\right)\quad\quad &&\mathrm{onda~incidente}
 \\
\vec{E}_r &=& E_r e^{ ik_0n_1\left(-z\cos\theta_r + x\sin\theta_r\right)}e^{-i\omega t} \left(\hat{x}\cos\theta_r +\hat{z}\sin\theta_r\right)\quad\quad &&\mathrm{onda~reflejada}
\\
\vec{E}_t &=& E_t e^{ ik_0n_2\left(z\cos\theta_t + x\sin\theta_t\right)}e^{-i\omega t} \left(\hat{x}\cos\theta_t -\hat{z}\sin\theta_t\right)\quad\quad 
&&\mathrm{onda~transmitida}
\end{eqnarray*}
\sphinxAtStartPar
De igual forma, a partir de la relación \(\vec{H} = \frac{E}{Z_0Z_r}\left(\hat{k}\times\hat{e}\right)\),
\begin{eqnarray*}
\vec{H}_i &=& \frac{n_1E_i}{Z_0}e^{ ik_0n_1\left(z\cos\theta_i + x\sin\theta_i\right)}e^{-i\omega t}\left(\hat{y}\right)\quad\quad &&\mathrm{onda~incidente}
 \\
\vec{H}_r &=& \frac{n_1E_r}{Z_0} e^{ ik_0n_1\left(-z\cos\theta_r + x\sin\theta_r\right)}e^{-i\omega t} \left(-\hat{y}\right)\quad\quad &&\mathrm{onda~reflejada}
\\
\vec{H}_t &=& \frac{n_2E_t}{Z_0} e^{ ik_0n_2\left(z\cos\theta_t + x\sin\theta_t\right)}e^{-i\omega t} \left(\hat{y}\right)\quad\quad 
&&\mathrm{onda~transmitida}
\end{eqnarray*}
\sphinxAtStartPar
A partir de la condición de borde en la interface \(z =0\):
\begin{equation*}
\begin{split}E^{\parallel}_1|_{z=0} - E^{\parallel}_2|_{z=0} = 0\end{split}
\end{equation*}
\sphinxAtStartPar
Tenemos:
\begin{equation*}
E_i\cos\theta_i e^{ ik_0n_1x\sin\theta_i}+E_r\cos\theta_r e^{ ik_0n_1x\sin\theta_r} - E_t\cos\theta_t e^{ ik_0n_1x\sin\theta_t} = 0
\end{equation*}
\sphinxAtStartPar
Dado que esta ecuación se debe satisfacer para cualquier punto \(x\), los exponentes debe ser iguales:
\begin{equation*}
n_1\sin\theta_i = n_1\sin\theta_r = n_2\sin\theta_t
\end{equation*}
\sphinxAtStartPar
Esto nos lleva a las leyes de Snell, para reflexión y transmisión:
\begin{equation*}
\theta_i = \theta_r\quad\quad\mathrm{y}\quad\quad n_1\sin\theta_i = n_2\sin\theta_t
\end{equation*}
\sphinxAtStartPar
Finalmente, la condición de borde del campo eléctrico queda:
\label{equation:2_ondas_EM_en_la_materia/2_ondas_EM_en_la_materia:383ad7ea-5b2c-4df5-852c-58caaf807007}\begin{equation}\label{eq:boundaryE}
E_i\cos\theta_i +E_r\cos\theta_r  - E_t\cos\theta_t = 0
\end{equation}
\sphinxAtStartPar
De igual forma, de la condición de borde \(H^{\parallel}_1 - H^{\parallel}_2 = 0\), deducimos:
\label{equation:2_ondas_EM_en_la_materia/2_ondas_EM_en_la_materia:ce051e2a-0111-43e2-8021-3824091eafec}\begin{equation}
n_1E_i - n_1E_r  - n_2E_t = 0 \label{eq:boundaryH}
\end{equation}
\sphinxAtStartPar
A partir de estas dos ecuaciones, determinamos los coeficientes de Fresnel de reflexión (\(r_\mathrm{TM}\)) y transmisión (\(t_\mathrm{TM}\)) para una onda TM:
\label{equation:2_ondas_EM_en_la_materia/2_ondas_EM_en_la_materia:540b24d3-434c-4c97-944a-35086df54b0c}\begin{align}
r_\mathrm{TM} &= \frac{E_r^\mathrm{TM}}{E_i^\mathrm{TM}} = \frac{n_1\cos\theta_t-n_2\cos\theta_i}
{n_1\cos\theta_t+n_2\cos\theta_i}
\\[10pt]
t_\mathrm{TM} &= \frac{E_t^\mathrm{TM}}{E_i^\mathrm{TM}} =\frac{2n_1\cos\theta_t}
{n_1\cos\theta_t+n_2\cos\theta_i}
\end{align}
\sphinxAtStartPar
Similarmente, para una onda transversal eléctrica (TE), los coeficientes de Fresnel son:
\label{equation:2_ondas_EM_en_la_materia/2_ondas_EM_en_la_materia:5bfc37a8-ff0e-4203-8c82-3c9ea1a7b061}\begin{align}
r_\mathrm{TE} &= \frac{E_r^\mathrm{TE}}{E_i^\mathrm{TE}} = \frac{n_1\cos\theta_i -n_2\cos\theta_t}
{n_1\cos\theta_i+n_2\cos\theta_t}
\\[10pt]
t_\mathrm{TE} &= \frac{E_t^\mathrm{TE}}{E_i^\mathrm{TE}} = \frac{2n_1\cos\theta_i}
{n_1\cos\theta_i+n_2\cos\theta_t}
\end{align}\begin{quote}

\sphinxAtStartPar
Las relaciones para los coeficientes de Fresnel se mantienen para índices de refracción complejos. En este caso, solo debemos reemplazar \(n_1\) por \(N_1\), y \(n_2\) por \(N_2\)
\end{quote}


\subsection{Reflectividad y transmisividad}
\label{\detokenize{2_ondas_EM_en_la_materia/2_ondas_EM_en_la_materia:reflectividad-y-transmisividad}}
\sphinxAtStartPar
Los coeficientes de Fresnel permiten determinar la magnitud del campo eléctrico (y magnético) reflejado y transmitido por una interface. Para determinar el flujo de energía a través de la interface, utilizamos el vector de Poynting. En el caso de la onda \(\mathrm{TM}\), y considerando indices de refracción complejos en los medios 1 y 2:
\begin{eqnarray*}
\biggl\langle{\vec{S}_i^\mathrm{TM}}\biggl\rangle &=& \frac{1}{2}\mathrm{Re}\left[\vec{E}_i\times\vec{H}_i^*\right] &=& \mathrm{Re}\left[N_1^* \hat{k}_i\right]\frac{{\left(E_i^\mathrm{TM}\right)}^2}{2Z_0}
\\
\biggl\langle{\vec{S}_r^\mathrm{TM}}\biggl\rangle &=& \frac{1}{2}\mathrm{Re}\left[\vec{E}_r\times\vec{H}_r^*\right] &=& \mathrm{Re}\left[N_1^* \hat{k}_r\right]\frac{{\left(E_r^\mathrm{TM}\right)}^2}{2Z_0}
\\
\biggl\langle{\vec{S}_t^\mathrm{TM}}\biggl\rangle &=& \frac{1}{2}\mathrm{Re}\left[\vec{E}_t\times\vec{H}_t^*\right] &=& \mathrm{Re}\left[N_2^* \hat{k}_t\right]\frac{{\left(E_t^\mathrm{TM}\right)}^2}{2Z_0}
\end{eqnarray*}
\sphinxAtStartPar
La \sphinxstylestrong{reflectividad (\(R\))} y \sphinxstylestrong{transmissivitdad (\(T\))} se definen, repectivamente, como \sphinxstylestrong{el flujo de energía reflejada y transmitida relativa al flujo de energía incidente, y en dirección normal a la interface.}

\sphinxAtStartPar
Así, considerando la componente del vector de Poynting normal a \(\hat{n}\) (notar que \(\hat{n} = - \hat{z}\) en nuestro ejemplo), tenemos:
\label{equation:2_ondas_EM_en_la_materia/2_ondas_EM_en_la_materia:c7d7b241-2eae-48e4-bcf8-14b5e25a12e5}\begin{eqnarray}
R_\mathrm{TM} &=& \frac{S_{r,z}^\mathrm{TM}}{S_{i,z}^\mathrm{TM}} &=& \lvert r_\mathrm{TM}\rvert^2
\\[10pt]
T_\mathrm{TM} &=& \frac{S_{t,z}^\mathrm{TM}}{S_{i,z}^\mathrm{TM}} &=& \frac{\mathrm{Re}\left(N_2^*\cos\theta_t\right)}{\mathrm{Re}\left(N_1^*\cos\theta_i\right)}\lvert t_\mathrm{TM}\rvert^2
\end{eqnarray}
\sphinxAtStartPar
De igual forma, para una onda TE, tenemos
\label{equation:2_ondas_EM_en_la_materia/2_ondas_EM_en_la_materia:6a60fc09-bc25-481a-8434-a9afdad84bf1}\begin{eqnarray}
R_\mathrm{TE} &=& \lvert r_\mathrm{TE}\rvert^2
\\[10pt]
T_\mathrm{TE} &=& \frac{\mathrm{Re}\left(N_2\cos\theta_t\right)}{\mathrm{Re}\left(N_1\cos\theta_i\right)}\lvert t_\mathrm{TE}\rvert^2
\end{eqnarray}
\sphinxAtStartPar
Notar que por conservación de energía:
\label{equation:2_ondas_EM_en_la_materia/2_ondas_EM_en_la_materia:b4157c3c-735a-4e84-b977-b4771b05ed92}\begin{equation}
R + T = 1
\end{equation}

\subsection{Casos particulares}
\label{\detokenize{2_ondas_EM_en_la_materia/2_ondas_EM_en_la_materia:casos-particulares}}
\sphinxAtStartPar
Asumiendo dos medios 1 y 2, con índice de refracción real, analicemos la reflectancia en función del ángulo de incidencia:
\begin{itemize}
\item {} 
\sphinxAtStartPar
caso 1, \(n_1 < n_2\)

\item {} 
\sphinxAtStartPar
caso 1, \(n_1 > n_2\)

\end{itemize}

\begin{sphinxuseclass}{cell}\begin{sphinxVerbatimInput}

\begin{sphinxuseclass}{cell_input}
\begin{sphinxVerbatim}[commandchars=\\\{\}]
\PYG{k+kn}{import} \PYG{n+nn}{numpy} \PYG{k}{as} \PYG{n+nn}{np}
\PYG{k+kn}{from} \PYG{n+nn}{numpy} \PYG{k+kn}{import} \PYG{n}{radians} \PYG{k}{as} \PYG{n}{rad} \PYG{c+c1}{\PYGZsh{} convertimos grados a radianes}
\PYG{k+kn}{import} \PYG{n+nn}{matplotlib}\PYG{n+nn}{.}\PYG{n+nn}{pyplot} \PYG{k}{as} \PYG{n+nn}{plt}
\PYG{k+kn}{from} \PYG{n+nn}{empylib}\PYG{n+nn}{.}\PYG{n+nn}{waveoptics} \PYG{k+kn}{import} \PYG{n}{interface}

\PYG{n}{theta} \PYG{o}{=} \PYG{n}{np}\PYG{o}{.}\PYG{n}{linspace}\PYG{p}{(}\PYG{l+m+mi}{0}\PYG{p}{,}\PYG{l+m+mi}{90}\PYG{p}{,}\PYG{l+m+mi}{100}\PYG{p}{)} \PYG{c+c1}{\PYGZsh{} Ángulo de incidencia}

\PYG{c+c1}{\PYGZsh{} Reflectividad en una interface}
\PYG{n}{Rp} \PYG{o}{=} \PYG{k}{lambda} \PYG{n}{n1}\PYG{p}{,}\PYG{n}{n2} \PYG{p}{:} \PYG{n}{interface}\PYG{p}{(}\PYG{n}{rad}\PYG{p}{(}\PYG{n}{theta}\PYG{p}{)}\PYG{p}{,}\PYG{n}{n1}\PYG{p}{,}\PYG{n}{n2}\PYG{p}{,}\PYG{n}{pol}\PYG{o}{=}\PYG{l+s+s1}{\PYGZsq{}}\PYG{l+s+s1}{TM}\PYG{l+s+s1}{\PYGZsq{}}\PYG{p}{)}\PYG{p}{[}\PYG{l+m+mi}{0}\PYG{p}{]} \PYG{c+c1}{\PYGZsh{} TM}
\PYG{n}{Rs} \PYG{o}{=} \PYG{k}{lambda} \PYG{n}{n1}\PYG{p}{,}\PYG{n}{n2} \PYG{p}{:} \PYG{n}{interface}\PYG{p}{(}\PYG{n}{rad}\PYG{p}{(}\PYG{n}{theta}\PYG{p}{)}\PYG{p}{,}\PYG{n}{n1}\PYG{p}{,}\PYG{n}{n2}\PYG{p}{,}\PYG{n}{pol}\PYG{o}{=}\PYG{l+s+s1}{\PYGZsq{}}\PYG{l+s+s1}{TE}\PYG{l+s+s1}{\PYGZsq{}}\PYG{p}{)}\PYG{p}{[}\PYG{l+m+mi}{0}\PYG{p}{]} \PYG{c+c1}{\PYGZsh{} TE}

\PYG{c+c1}{\PYGZsh{} preparamos el ploteo}
\PYG{k}{def} \PYG{n+nf}{plot\PYGZus{}R\PYGZus{}interface}\PYG{p}{(}\PYG{n}{n1}\PYG{p}{,}\PYG{n}{n2}\PYG{p}{)}\PYG{p}{:}
    \PYG{n}{fig}\PYG{p}{,} \PYG{n}{ax} \PYG{o}{=} \PYG{n}{plt}\PYG{o}{.}\PYG{n}{subplots}\PYG{p}{(}\PYG{p}{)}
    \PYG{n}{fig}\PYG{o}{.}\PYG{n}{set\PYGZus{}size\PYGZus{}inches}\PYG{p}{(}\PYG{l+m+mi}{9}\PYG{p}{,} \PYG{l+m+mi}{6}\PYG{p}{)}
    \PYG{n}{plt}\PYG{o}{.}\PYG{n}{rcParams}\PYG{p}{[}\PYG{l+s+s1}{\PYGZsq{}}\PYG{l+s+s1}{font.size}\PYG{l+s+s1}{\PYGZsq{}}\PYG{p}{]} \PYG{o}{=} \PYG{l+s+s1}{\PYGZsq{}}\PYG{l+s+s1}{18}\PYG{l+s+s1}{\PYGZsq{}}
    \PYG{n}{ax}\PYG{o}{.}\PYG{n}{plot}\PYG{p}{(}\PYG{n}{theta}\PYG{p}{,}\PYG{n}{Rp}\PYG{p}{(}\PYG{n}{n1}\PYG{p}{,}\PYG{n}{n2}\PYG{p}{)}\PYG{p}{,} \PYG{n}{label}\PYG{o}{=}\PYG{l+s+s1}{\PYGZsq{}}\PYG{l+s+s1}{\PYGZdl{}R\PYGZus{}}\PYG{l+s+s1}{\PYGZbs{}}\PYG{l+s+s1}{mathrm}\PYG{l+s+si}{\PYGZob{}TM\PYGZcb{}}\PYG{l+s+s1}{\PYGZdl{}}\PYG{l+s+s1}{\PYGZsq{}}\PYG{p}{,} \PYG{n}{color}\PYG{o}{=}\PYG{l+s+s1}{\PYGZsq{}}\PYG{l+s+s1}{red}\PYG{l+s+s1}{\PYGZsq{}}\PYG{p}{)}
    \PYG{n}{ax}\PYG{o}{.}\PYG{n}{plot}\PYG{p}{(}\PYG{n}{theta}\PYG{p}{,}\PYG{n}{Rs}\PYG{p}{(}\PYG{n}{n1}\PYG{p}{,}\PYG{n}{n2}\PYG{p}{)}\PYG{p}{,} \PYG{n}{label}\PYG{o}{=}\PYG{l+s+s1}{\PYGZsq{}}\PYG{l+s+s1}{\PYGZdl{}R\PYGZus{}}\PYG{l+s+s1}{\PYGZbs{}}\PYG{l+s+s1}{mathrm}\PYG{l+s+si}{\PYGZob{}TE\PYGZcb{}}\PYG{l+s+s1}{\PYGZdl{}}\PYG{l+s+s1}{\PYGZsq{}}\PYG{p}{,}\PYG{n}{color}\PYG{o}{=}\PYG{l+s+s1}{\PYGZsq{}}\PYG{l+s+s1}{blue}\PYG{l+s+s1}{\PYGZsq{}}\PYG{p}{)}
    \PYG{n}{ax}\PYG{o}{.}\PYG{n}{set\PYGZus{}xlim}\PYG{p}{(}\PYG{p}{[}\PYG{l+m+mi}{0}\PYG{p}{,}\PYG{l+m+mi}{90}\PYG{p}{]}\PYG{p}{)}
    \PYG{n}{ax}\PYG{o}{.}\PYG{n}{set\PYGZus{}ylim}\PYG{p}{(}\PYG{p}{[}\PYG{l+m+mi}{0}\PYG{p}{,}\PYG{l+m+mf}{1.0}\PYG{p}{]}\PYG{p}{)}
    \PYG{n}{ax}\PYG{o}{.}\PYG{n}{set\PYGZus{}xlabel}\PYG{p}{(}\PYG{l+s+s1}{\PYGZsq{}}\PYG{l+s+s1}{Ángulo de incidencia (°)}\PYG{l+s+s1}{\PYGZsq{}}\PYG{p}{)}
    \PYG{n}{ax}\PYG{o}{.}\PYG{n}{set\PYGZus{}ylabel}\PYG{p}{(}\PYG{l+s+s1}{\PYGZsq{}}\PYG{l+s+s1}{Reflectividad}\PYG{l+s+s1}{\PYGZsq{}}\PYG{p}{)}
    \PYG{n}{ax}\PYG{o}{.}\PYG{n}{legend}\PYG{p}{(}\PYG{n}{frameon}\PYG{o}{=}\PYG{k+kc}{False}\PYG{p}{)}
\end{sphinxVerbatim}

\end{sphinxuseclass}\end{sphinxVerbatimInput}

\end{sphinxuseclass}
\begin{sphinxuseclass}{cell}\begin{sphinxVerbatimInput}

\begin{sphinxuseclass}{cell_input}
\begin{sphinxVerbatim}[commandchars=\\\{\}]
\PYG{k+kn}{from} \PYG{n+nn}{ipywidgets} \PYG{k+kn}{import} \PYG{n}{interact}

\PYG{n+nd}{@interact}\PYG{p}{(} \PYG{n}{n1}\PYG{o}{=}\PYG{p}{(}\PYG{l+m+mi}{1}\PYG{p}{,}\PYG{l+m+mi}{5}\PYG{p}{,} \PYG{l+m+mf}{0.1}\PYG{p}{)}\PYG{p}{,} \PYG{n}{n2}\PYG{o}{=}\PYG{p}{(}\PYG{l+m+mi}{1}\PYG{p}{,}\PYG{l+m+mi}{5}\PYG{p}{,} \PYG{l+m+mf}{0.1}\PYG{p}{)}\PYG{p}{)}
\PYG{k}{def} \PYG{n+nf}{g}\PYG{p}{(}\PYG{n}{n1}\PYG{o}{=}\PYG{l+m+mf}{1.0}\PYG{p}{,} \PYG{n}{n2}\PYG{o}{=}\PYG{l+m+mf}{1.5}\PYG{p}{)}\PYG{p}{:}
    \PYG{k}{return} \PYG{n}{plot\PYGZus{}R\PYGZus{}interface}\PYG{p}{(}\PYG{n}{n1}\PYG{p}{,}\PYG{n}{n2}\PYG{p}{)}
\end{sphinxVerbatim}

\end{sphinxuseclass}\end{sphinxVerbatimInput}
\begin{sphinxVerbatimOutput}

\begin{sphinxuseclass}{cell_output}
\begin{sphinxVerbatim}[commandchars=\\\{\}]
interactive(children=(FloatSlider(value=1.0, description=\PYGZsq{}n1\PYGZsq{}, max=5.0, min=1.0), FloatSlider(value=1.5, descr…
\end{sphinxVerbatim}

\end{sphinxuseclass}\end{sphinxVerbatimOutput}

\end{sphinxuseclass}
\sphinxAtStartPar
Cuando \(n_1 < n_2\) vemos que \(R_\mathrm{TM} = 0\) en un cierto ángulo. Este ángulo se denomina \sphinxstylestrong{ángulo de Brewster.} En este ángulo solo la componente TE es reflejada.

\sphinxAtStartPar
Los lentes polarizados toman ventaja del ángulo de Brewster. Estos lentes están diseñados para bloquear las ondas TE, y de esta forma reducir el brillo enceguecedor generado por la reflección de la luz solar

\noindent{\hspace*{\fill}\sphinxincludegraphics[width=350\sphinxpxdimen]{{polarized_glasses}.jpg}\hspace*{\fill}}

\sphinxAtStartPar
Así, si giramos los lentes en posición vertical (asumiento lentes con alto nivel de polarización), el efecto se invierte. Es decir, las ondas TE se transmiten y las TM no.

\sphinxAtStartPar
Por otro lado, cuando \(n_1 > n_2\), vemos que \(R_\mathrm{TM} = R_\mathrm{TE} = 0\) sobre cierto ángulo. Este ángulo se denomina \sphinxstylestrong{ángulo crítico (\(\theta_c\)).} Para deterinar el ángulo crítico usamos la ley de Snell.

\sphinxAtStartPar
El ángulo máximo para la onda transmitida es \(\theta_t = 90^o\), la ley de Snell nos indica que existe un ángulo crítico. Sobre este valor, no existe solución real.
\label{equation:2_ondas_EM_en_la_materia/2_ondas_EM_en_la_materia:fb9f4dad-64c6-4a91-b2cd-e752f49e2ff4}\begin{equation}
n_1\sin\theta_c = n_2\sin90^o \Rightarrow \theta_c = \arcsin\left(n_2/n_1\right)
\end{equation}
\sphinxAtStartPar
\sphinxstylestrong{Para \(\theta_i > \theta_c\), \(R_\mathrm{TE} = R_\mathrm{TM} = 1\).}

\sphinxAtStartPar
Este mecanismo se llama \sphinxstylestrong{reflección interna total} y es la base para el funcionamiento de fibras ópticas y lasers
\sphinxincludegraphics[width=450\sphinxpxdimen]{{optical_fiber}.png}


\section{Reflección y transmissión en películas delgadas}
\label{\detokenize{2_ondas_EM_en_la_materia/2_ondas_EM_en_la_materia:refleccion-y-transmission-en-peliculas-delgadas}}
\sphinxAtStartPar
En el caso materiales de película delgada, las ondas electromagnéticas se reflejan y transmiten múltiples veces.

\noindent{\hspace*{\fill}\sphinxincludegraphics[width=600\sphinxpxdimen]{{reflectance_thinfilm}.png}\hspace*{\fill}}

\sphinxAtStartPar
Considerando los medios 1,2 y 3 ordenados consecutivamente en dirección de la onda incidente, con el medio 2 condicionado por un espesor \(d\), se puede demostrar que en este caso los coeficientes de Fresnel, para indices de refracción reales son:
\label{equation:2_ondas_EM_en_la_materia/2_ondas_EM_en_la_materia:129f3915-5ddd-4bb3-b141-a20c4128b5c4}\begin{align}
r &= \frac{r_{12}+r_{23}e^{2i\varphi_2}}
          {1+r_{12}r_{23}e^{2i\varphi_2}}
\\[10pt]
t &= \frac{t_{12}t_{23}e^{i\varphi_2}}
          {1+r_{12}r_{23}e^{2i\varphi_2}}
\end{align}
\sphinxAtStartPar
donde \(\varphi_2 = N_2k_0d\cos\theta_2\) (\(\theta_2\) es el ángulo de transmisión en el medio 2); \(r_{12}\), \(r_{23}\) y \(t_{12}\), \(t_{23}\) son, respectivamente, los coeficientes de Fresnel desde el medio 1 al medio 2, y desde el medio 2 al medio 3. Estas fórmulas son válidas tanto para ondas TE como para ondas TM.

\sphinxAtStartPar
Basado en estas expresiones, podemos calcular la reflectividad y tranmissividad de la película:
\label{equation:2_ondas_EM_en_la_materia/2_ondas_EM_en_la_materia:66cb98b4-f45d-44c7-b747-8c13140f1a1b}\begin{align}
R = {\lvert r\rvert}^2 &= \frac{r_{12}^2+r_{23}^2+2r_{12}r_{23}\cos 2\varphi_2}
                              {1 + 2r_{12}r_{23}\cos 2\varphi_2 + r_{12}^2r_{23}^2}
\\[10pt]
T = \frac{n_3\cos\theta_t}{n_1\cos\theta_i}{\lvert t\rvert}^2 &= 
\frac{\left(1 - r_{12}^2\right)\left(1 - r_{23}^2\right)}
     {1 + 2r_{12}r_{23}\cos 2\varphi_2 + r_{12}^2r_{23}^2}
\end{align}
\sphinxAtStartPar
Analicemos como se comportan estas ecuaciones en un caso real.

\sphinxAtStartPar
Como ejemplo, consideremos la reflectividad de una película delgada de sílice (SiO\(_2\)) sobre un sustrato de silicio. Esta capa se genera naturalmente debido a la oxidación del silicio

\noindent{\hspace*{\fill}\sphinxincludegraphics[width=600\sphinxpxdimen]{{sio2_coating}.png}\hspace*{\fill}}

\sphinxAtStartPar
Para simplificar, consideremos:
\begin{itemize}
\item {} 
\sphinxAtStartPar
índice de refracción del aire: 1.0

\item {} 
\sphinxAtStartPar
índice de refracción de sílice: 1.5

\item {} 
\sphinxAtStartPar
índice de refracción del silicio: 4.3

\item {} 
\sphinxAtStartPar
espectro de longitudes de onda: 300 \sphinxhyphen{} 800 nm (visible)

\item {} 
\sphinxAtStartPar
espesor del sílice, \(d\): variable

\item {} 
\sphinxAtStartPar
ángulo de incidencia \(\theta_i\): variable

\end{itemize}

\begin{sphinxuseclass}{cell}\begin{sphinxVerbatimInput}

\begin{sphinxuseclass}{cell_input}
\begin{sphinxVerbatim}[commandchars=\\\{\}]
\PYG{k+kn}{import} \PYG{n+nn}{numpy} \PYG{k}{as} \PYG{n+nn}{np}
\PYG{k+kn}{from} \PYG{n+nn}{numpy} \PYG{k+kn}{import} \PYG{n}{radians} \PYG{k}{as} \PYG{n}{rad} \PYG{c+c1}{\PYGZsh{} convertimos grados a radianes}
\PYG{k+kn}{import} \PYG{n+nn}{matplotlib}\PYG{n+nn}{.}\PYG{n+nn}{pyplot} \PYG{k}{as} \PYG{n+nn}{plt}
\PYG{k+kn}{from} \PYG{n+nn}{empylib}\PYG{n+nn}{.}\PYG{n+nn}{waveoptics} \PYG{k+kn}{import} \PYG{n}{multilayer}

\PYG{c+c1}{\PYGZsh{} Reflectividad en capa delgada}
\PYG{n}{lam} \PYG{o}{=} \PYG{n}{np}\PYG{o}{.}\PYG{n}{linspace}\PYG{p}{(}\PYG{l+m+mf}{0.3}\PYG{p}{,}\PYG{l+m+mf}{0.8}\PYG{p}{,}\PYG{l+m+mi}{100}\PYG{p}{)}          \PYG{c+c1}{\PYGZsh{} longitud de onda (en um)}
\PYG{n}{n\PYGZus{}layers} \PYG{o}{=} \PYG{p}{(}\PYG{l+m+mf}{1.0}\PYG{p}{,}\PYG{l+m+mf}{1.5}\PYG{p}{,}\PYG{l+m+mf}{4.3}\PYG{p}{)}         \PYG{c+c1}{\PYGZsh{} índices de refracción n1, n2, n3}
\PYG{n}{Rp} \PYG{o}{=} \PYG{k}{lambda} \PYG{n}{tt}\PYG{p}{,}\PYG{n}{d} \PYG{p}{:} \PYG{n}{multilayer}\PYG{p}{(}\PYG{n}{lam}\PYG{p}{,} \PYG{n}{rad}\PYG{p}{(}\PYG{n}{tt}\PYG{p}{)}\PYG{p}{,}\PYG{n}{n\PYGZus{}layers}\PYG{p}{,} \PYG{p}{(}\PYG{n}{d}\PYG{p}{,}\PYG{p}{)}\PYG{p}{,} \PYG{l+s+s1}{\PYGZsq{}}\PYG{l+s+s1}{TM}\PYG{l+s+s1}{\PYGZsq{}}\PYG{p}{)}\PYG{p}{[}\PYG{l+m+mi}{0}\PYG{p}{]}
\PYG{n}{Rs} \PYG{o}{=} \PYG{k}{lambda} \PYG{n}{tt}\PYG{p}{,}\PYG{n}{d} \PYG{p}{:} \PYG{n}{multilayer}\PYG{p}{(}\PYG{n}{lam}\PYG{p}{,} \PYG{n}{rad}\PYG{p}{(}\PYG{n}{tt}\PYG{p}{)}\PYG{p}{,}\PYG{n}{n\PYGZus{}layers}\PYG{p}{,} \PYG{p}{(}\PYG{n}{d}\PYG{p}{,}\PYG{p}{)}\PYG{p}{,} \PYG{l+s+s1}{\PYGZsq{}}\PYG{l+s+s1}{TE}\PYG{l+s+s1}{\PYGZsq{}}\PYG{p}{)}\PYG{p}{[}\PYG{l+m+mi}{0}\PYG{p}{]}

\PYG{c+c1}{\PYGZsh{} preparamos el ploteo}
\PYG{k}{def} \PYG{n+nf}{plot\PYGZus{}R\PYGZus{}multi}\PYG{p}{(}\PYG{n}{theta}\PYG{p}{,}\PYG{n}{d}\PYG{p}{)}\PYG{p}{:}
    \PYG{n}{fig}\PYG{p}{,} \PYG{n}{ax} \PYG{o}{=} \PYG{n}{plt}\PYG{o}{.}\PYG{n}{subplots}\PYG{p}{(}\PYG{p}{)}
    \PYG{n}{fig}\PYG{o}{.}\PYG{n}{set\PYGZus{}size\PYGZus{}inches}\PYG{p}{(}\PYG{l+m+mi}{9}\PYG{p}{,} \PYG{l+m+mi}{6}\PYG{p}{)}
    \PYG{n}{plt}\PYG{o}{.}\PYG{n}{rcParams}\PYG{p}{[}\PYG{l+s+s1}{\PYGZsq{}}\PYG{l+s+s1}{font.size}\PYG{l+s+s1}{\PYGZsq{}}\PYG{p}{]} \PYG{o}{=} \PYG{l+s+s1}{\PYGZsq{}}\PYG{l+s+s1}{16}\PYG{l+s+s1}{\PYGZsq{}}
    \PYG{n}{ax}\PYG{o}{.}\PYG{n}{plot}\PYG{p}{(}\PYG{n}{lam}\PYG{p}{,}\PYG{n}{Rp}\PYG{p}{(}\PYG{n}{theta}\PYG{p}{,}\PYG{n}{d}\PYG{p}{)}\PYG{p}{,} \PYG{n}{label}\PYG{o}{=}\PYG{l+s+s1}{\PYGZsq{}}\PYG{l+s+s1}{\PYGZdl{}R\PYGZus{}}\PYG{l+s+s1}{\PYGZbs{}}\PYG{l+s+s1}{mathrm}\PYG{l+s+si}{\PYGZob{}TM\PYGZcb{}}\PYG{l+s+s1}{\PYGZdl{}}\PYG{l+s+s1}{\PYGZsq{}}\PYG{p}{,} \PYG{n}{color}\PYG{o}{=}\PYG{l+s+s1}{\PYGZsq{}}\PYG{l+s+s1}{red}\PYG{l+s+s1}{\PYGZsq{}}\PYG{p}{)}
    \PYG{n}{ax}\PYG{o}{.}\PYG{n}{plot}\PYG{p}{(}\PYG{n}{lam}\PYG{p}{,}\PYG{n}{Rs}\PYG{p}{(}\PYG{n}{theta}\PYG{p}{,}\PYG{n}{d}\PYG{p}{)}\PYG{p}{,} \PYG{n}{label}\PYG{o}{=}\PYG{l+s+s1}{\PYGZsq{}}\PYG{l+s+s1}{\PYGZdl{}R\PYGZus{}}\PYG{l+s+s1}{\PYGZbs{}}\PYG{l+s+s1}{mathrm}\PYG{l+s+si}{\PYGZob{}TE\PYGZcb{}}\PYG{l+s+s1}{\PYGZdl{}}\PYG{l+s+s1}{\PYGZsq{}}\PYG{p}{,}\PYG{n}{color}\PYG{o}{=}\PYG{l+s+s1}{\PYGZsq{}}\PYG{l+s+s1}{blue}\PYG{l+s+s1}{\PYGZsq{}}\PYG{p}{)}
    \PYG{n}{ax}\PYG{o}{.}\PYG{n}{set\PYGZus{}xlim}\PYG{p}{(}\PYG{p}{[}\PYG{n+nb}{min}\PYG{p}{(}\PYG{n}{lam}\PYG{p}{)}\PYG{p}{,}\PYG{n+nb}{max}\PYG{p}{(}\PYG{n}{lam}\PYG{p}{)}\PYG{p}{]}\PYG{p}{)}
    \PYG{n}{ax}\PYG{o}{.}\PYG{n}{set\PYGZus{}ylim}\PYG{p}{(}\PYG{p}{[}\PYG{l+m+mi}{0}\PYG{p}{,}\PYG{l+m+mf}{1.0}\PYG{p}{]}\PYG{p}{)}
    \PYG{n}{ax}\PYG{o}{.}\PYG{n}{set\PYGZus{}xlabel}\PYG{p}{(}\PYG{l+s+s1}{\PYGZsq{}}\PYG{l+s+s1}{Longitud de onda (\PYGZdl{}}\PYG{l+s+s1}{\PYGZbs{}}\PYG{l+s+s1}{mu\PYGZdl{}m)}\PYG{l+s+s1}{\PYGZsq{}}\PYG{p}{)}
    \PYG{n}{ax}\PYG{o}{.}\PYG{n}{set\PYGZus{}ylabel}\PYG{p}{(}\PYG{l+s+s1}{\PYGZsq{}}\PYG{l+s+s1}{Reflectividad}\PYG{l+s+s1}{\PYGZsq{}}\PYG{p}{)}
    \PYG{n}{ax}\PYG{o}{.}\PYG{n}{legend}\PYG{p}{(}\PYG{n}{frameon}\PYG{o}{=}\PYG{k+kc}{False}\PYG{p}{)}
\end{sphinxVerbatim}

\end{sphinxuseclass}\end{sphinxVerbatimInput}

\end{sphinxuseclass}
\begin{sphinxuseclass}{cell}\begin{sphinxVerbatimInput}

\begin{sphinxuseclass}{cell_input}
\begin{sphinxVerbatim}[commandchars=\\\{\}]
\PYG{k+kn}{from} \PYG{n+nn}{ipywidgets} \PYG{k+kn}{import} \PYG{n}{interact}

\PYG{n+nd}{@interact}\PYG{p}{(}\PYG{n}{theta}\PYG{o}{=}\PYG{p}{(}\PYG{l+m+mi}{0}\PYG{p}{,}\PYG{l+m+mi}{89}\PYG{p}{,}\PYG{l+m+mi}{10}\PYG{p}{)}\PYG{p}{,} \PYG{n}{d}\PYG{o}{=}\PYG{p}{(}\PYG{l+m+mi}{0}\PYG{p}{,}\PYG{l+m+mf}{1.0}\PYG{p}{,}\PYG{l+m+mf}{0.01}\PYG{p}{)}\PYG{p}{)}
\PYG{k}{def} \PYG{n+nf}{g}\PYG{p}{(}\PYG{n}{theta}\PYG{o}{=}\PYG{l+m+mi}{30}\PYG{p}{,} \PYG{n}{d}\PYG{o}{=}\PYG{l+m+mf}{0.3}\PYG{p}{)}\PYG{p}{:}
    \PYG{k}{return} \PYG{n}{plot\PYGZus{}R\PYGZus{}multi}\PYG{p}{(}\PYG{n}{theta}\PYG{p}{,}\PYG{n}{d}\PYG{p}{)}
\end{sphinxVerbatim}

\end{sphinxuseclass}\end{sphinxVerbatimInput}
\begin{sphinxVerbatimOutput}

\begin{sphinxuseclass}{cell_output}
\begin{sphinxVerbatim}[commandchars=\\\{\}]
interactive(children=(IntSlider(value=30, description=\PYGZsq{}theta\PYGZsq{}, max=89, step=10), FloatSlider(value=0.3, descri…
\end{sphinxVerbatim}

\end{sphinxuseclass}\end{sphinxVerbatimOutput}

\end{sphinxuseclass}
\sphinxAtStartPar
Esta oscilaciones en la reflectancia al variar \(\theta_i\) y \(d\) son el resultado de la interferencia entre las ondas reflejadas en la parte inferior y superior de la película de silicio.

\noindent{\hspace*{\fill}\sphinxincludegraphics[width=350\sphinxpxdimen]{{interference}.png}\hspace*{\fill}}

\sphinxAtStartPar
En palabras simples, este fenómeno ocurre por qué la onda reflejada en la parte inferior de la película debe recorre un camino más largo. Esto produce un desface con las ondas reflejadas en la parte superior que deriva en interferencia constructiva (alta reflectividad) y destructiva (baja reflectividad)

\sphinxAtStartPar
Este fenómeno se manifiesta en forma de color ya que nuestros ojos son sensibles a los cambios de radiación en este espectro.

\sphinxAtStartPar
Analicemos como se manifiesta este fenómeno en forma de color:

\begin{sphinxuseclass}{cell}\begin{sphinxVerbatimInput}

\begin{sphinxuseclass}{cell_input}
\begin{sphinxVerbatim}[commandchars=\\\{\}]
\PYG{k+kn}{import} \PYG{n+nn}{numpy} \PYG{k}{as} \PYG{n+nn}{np}
\PYG{k+kn}{from} \PYG{n+nn}{numpy} \PYG{k+kn}{import} \PYG{n}{radians} \PYG{k}{as} \PYG{n}{rad} \PYG{c+c1}{\PYGZsh{} convertimos grados a radianes}
\PYG{k+kn}{import} \PYG{n+nn}{matplotlib}\PYG{n+nn}{.}\PYG{n+nn}{pyplot} \PYG{k}{as} \PYG{n+nn}{plt}
\PYG{k+kn}{from} \PYG{n+nn}{empylib}\PYG{n+nn}{.}\PYG{n+nn}{waveoptics} \PYG{k+kn}{import} \PYG{n}{multilayer}
\PYG{k+kn}{from} \PYG{n+nn}{empylib}\PYG{n+nn}{.}\PYG{n+nn}{ref\PYGZus{}spectra} \PYG{k+kn}{import} \PYG{n}{AM15}
\PYG{k+kn}{from} \PYG{n+nn}{empylib}\PYG{n+nn}{.}\PYG{n+nn}{ref\PYGZus{}spectra} \PYG{k+kn}{import} \PYG{n}{color\PYGZus{}system} \PYG{k}{as} \PYG{n}{cs}
\PYG{n}{cs} \PYG{o}{=} \PYG{n}{cs}\PYG{o}{.}\PYG{n}{hdtv}

\PYG{c+c1}{\PYGZsh{} Reflectividad en capa delgada}
\PYG{n}{lam} \PYG{o}{=} \PYG{n}{np}\PYG{o}{.}\PYG{n}{linspace}\PYG{p}{(}\PYG{l+m+mf}{0.3}\PYG{p}{,}\PYG{l+m+mf}{0.8}\PYG{p}{,}\PYG{l+m+mi}{100}\PYG{p}{)}   \PYG{c+c1}{\PYGZsh{} longitud de onda (en um)}
\PYG{n}{n\PYGZus{}layers} \PYG{o}{=} \PYG{p}{(}\PYG{l+m+mf}{1.0}\PYG{p}{,}\PYG{l+m+mf}{1.5}\PYG{p}{,}\PYG{l+m+mf}{4.3}\PYG{p}{)}         \PYG{c+c1}{\PYGZsh{} índices de refracción n1, n2, n3}
\PYG{n}{Rp} \PYG{o}{=} \PYG{k}{lambda} \PYG{n}{tt}\PYG{p}{,}\PYG{n}{d} \PYG{p}{:} \PYG{n}{multilayer}\PYG{p}{(}\PYG{n}{lam}\PYG{p}{,} \PYG{n}{rad}\PYG{p}{(}\PYG{n}{tt}\PYG{p}{)}\PYG{p}{,}\PYG{n}{n\PYGZus{}layers}\PYG{p}{,} \PYG{p}{(}\PYG{n}{d}\PYG{p}{,}\PYG{p}{)}\PYG{p}{,} \PYG{l+s+s1}{\PYGZsq{}}\PYG{l+s+s1}{TM}\PYG{l+s+s1}{\PYGZsq{}}\PYG{p}{)}\PYG{p}{[}\PYG{l+m+mi}{0}\PYG{p}{]}
\PYG{n}{Rs} \PYG{o}{=} \PYG{k}{lambda} \PYG{n}{tt}\PYG{p}{,}\PYG{n}{d} \PYG{p}{:} \PYG{n}{multilayer}\PYG{p}{(}\PYG{n}{lam}\PYG{p}{,} \PYG{n}{rad}\PYG{p}{(}\PYG{n}{tt}\PYG{p}{)}\PYG{p}{,}\PYG{n}{n\PYGZus{}layers}\PYG{p}{,} \PYG{p}{(}\PYG{n}{d}\PYG{p}{,}\PYG{p}{)}\PYG{p}{,} \PYG{l+s+s1}{\PYGZsq{}}\PYG{l+s+s1}{TE}\PYG{l+s+s1}{\PYGZsq{}}\PYG{p}{)}\PYG{p}{[}\PYG{l+m+mi}{0}\PYG{p}{]}

\PYG{n}{cs}\PYG{o}{.}\PYG{n}{interp\PYGZus{}internals}\PYG{p}{(}\PYG{n}{lam}\PYG{p}{)}
\PYG{k}{def} \PYG{n+nf}{color\PYGZus{}R\PYGZus{}film}\PYG{p}{(}\PYG{n}{d}\PYG{p}{)}\PYG{p}{:}
    \PYG{c+c1}{\PYGZsh{} formateamos la figura}
    \PYG{n}{fig}\PYG{p}{,} \PYG{n}{ax} \PYG{o}{=} \PYG{n}{plt}\PYG{o}{.}\PYG{n}{subplots}\PYG{p}{(}\PYG{p}{)}
    \PYG{n}{fig}\PYG{o}{.}\PYG{n}{set\PYGZus{}size\PYGZus{}inches}\PYG{p}{(}\PYG{l+m+mi}{9}\PYG{p}{,} \PYG{l+m+mi}{5}\PYG{p}{)}
    \PYG{n}{plt}\PYG{o}{.}\PYG{n}{rcParams}\PYG{p}{[}\PYG{l+s+s1}{\PYGZsq{}}\PYG{l+s+s1}{font.size}\PYG{l+s+s1}{\PYGZsq{}}\PYG{p}{]} \PYG{o}{=} \PYG{l+s+s1}{\PYGZsq{}}\PYG{l+s+s1}{16}\PYG{l+s+s1}{\PYGZsq{}}
    
    \PYG{n}{theta} \PYG{o}{=} \PYG{n}{np}\PYG{o}{.}\PYG{n}{linspace}\PYG{p}{(}\PYG{l+m+mi}{0}\PYG{p}{,}\PYG{l+m+mi}{90}\PYG{p}{,}\PYG{l+m+mi}{100}\PYG{p}{)} \PYG{c+c1}{\PYGZsh{} angulo de incidencia}
    \PYG{k}{for} \PYG{n}{tt} \PYG{o+ow}{in} \PYG{n}{theta}\PYG{p}{:} 
        \PYG{n}{R} \PYG{o}{=} \PYG{l+m+mf}{0.5}\PYG{o}{*}\PYG{n}{Rp}\PYG{p}{(}\PYG{n}{tt}\PYG{p}{,}\PYG{n}{d}\PYG{p}{)} \PYG{o}{+} \PYG{l+m+mf}{0.5}\PYG{o}{*}\PYG{n}{Rs}\PYG{p}{(}\PYG{n}{tt}\PYG{p}{,}\PYG{n}{d}\PYG{p}{)}
        \PYG{n}{Irad} \PYG{o}{=} \PYG{n}{R}\PYG{o}{*}\PYG{n}{AM15}\PYG{p}{(}\PYG{n}{lam}\PYG{p}{)}
        \PYG{n}{html\PYGZus{}rgb} \PYG{o}{=} \PYG{n}{cs}\PYG{o}{.}\PYG{n}{spec\PYGZus{}to\PYGZus{}rgb}\PYG{p}{(}\PYG{n}{Irad}\PYG{p}{,} \PYG{n}{out\PYGZus{}fmt}\PYG{o}{=}\PYG{l+s+s1}{\PYGZsq{}}\PYG{l+s+s1}{html}\PYG{l+s+s1}{\PYGZsq{}}\PYG{p}{)}
        \PYG{n}{ax}\PYG{o}{.}\PYG{n}{axvline}\PYG{p}{(}\PYG{n}{tt}\PYG{p}{,} \PYG{n}{color}\PYG{o}{=}\PYG{n}{html\PYGZus{}rgb}\PYG{p}{,} \PYG{n}{linewidth}\PYG{o}{=}\PYG{l+m+mi}{6}\PYG{p}{)} 
    \PYG{n}{ax}\PYG{o}{.}\PYG{n}{set\PYGZus{}xlim}\PYG{p}{(}\PYG{p}{[}\PYG{n+nb}{min}\PYG{p}{(}\PYG{n}{theta}\PYG{p}{)}\PYG{p}{,}\PYG{n+nb}{max}\PYG{p}{(}\PYG{n}{theta}\PYG{p}{)}\PYG{p}{]}\PYG{p}{)}
    \PYG{n}{ax}\PYG{o}{.}\PYG{n}{set\PYGZus{}ylim}\PYG{p}{(}\PYG{p}{[}\PYG{l+m+mi}{0}\PYG{p}{,}\PYG{l+m+mf}{1.0}\PYG{p}{]}\PYG{p}{)}
    \PYG{n}{ax}\PYG{o}{.}\PYG{n}{axes}\PYG{o}{.}\PYG{n}{yaxis}\PYG{o}{.}\PYG{n}{set\PYGZus{}visible}\PYG{p}{(}\PYG{k+kc}{False}\PYG{p}{)}
    \PYG{n}{ax}\PYG{o}{.}\PYG{n}{set\PYGZus{}xlabel}\PYG{p}{(}\PYG{l+s+s1}{\PYGZsq{}}\PYG{l+s+s1}{Ángulo de incidencia (deg)}\PYG{l+s+s1}{\PYGZsq{}}\PYG{p}{)}
\end{sphinxVerbatim}

\end{sphinxuseclass}\end{sphinxVerbatimInput}

\end{sphinxuseclass}
\begin{sphinxuseclass}{cell}\begin{sphinxVerbatimInput}

\begin{sphinxuseclass}{cell_input}
\begin{sphinxVerbatim}[commandchars=\\\{\}]
\PYG{k+kn}{from} \PYG{n+nn}{ipywidgets} \PYG{k+kn}{import} \PYG{n}{interact}

\PYG{n+nd}{@interact}\PYG{p}{(}\PYG{n}{d}\PYG{o}{=}\PYG{p}{(}\PYG{l+m+mi}{0}\PYG{p}{,}\PYG{l+m+mf}{1.0}\PYG{p}{,}\PYG{l+m+mf}{0.001}\PYG{p}{)}\PYG{p}{)}
\PYG{k}{def} \PYG{n+nf}{g}\PYG{p}{(}\PYG{n}{d}\PYG{o}{=}\PYG{l+m+mf}{0.28}\PYG{p}{)}\PYG{p}{:}
    \PYG{k}{return} \PYG{n}{color\PYGZus{}R\PYGZus{}film}\PYG{p}{(}\PYG{n}{d}\PYG{p}{)}
\end{sphinxVerbatim}

\end{sphinxuseclass}\end{sphinxVerbatimInput}
\begin{sphinxVerbatimOutput}

\begin{sphinxuseclass}{cell_output}
\begin{sphinxVerbatim}[commandchars=\\\{\}]
interactive(children=(FloatSlider(value=0.28, description=\PYGZsq{}d\PYGZsq{}, max=1.0, step=0.001), Output()), \PYGZus{}dom\PYGZus{}classes=(…
\end{sphinxVerbatim}

\end{sphinxuseclass}\end{sphinxVerbatimOutput}

\end{sphinxuseclass}

\section{Referencias}
\label{\detokenize{2_ondas_EM_en_la_materia/2_ondas_EM_en_la_materia:referencias}}
\sphinxAtStartPar
Griffths D., \sphinxstyleemphasis{Introduction to Electrodynamics}, 4th Ed, Pearson, 2013
\begin{itemize}
\item {} 
\sphinxAtStartPar
7.3 Maxwell’s Equations (7.36)

\item {} 
\sphinxAtStartPar
9 Electromagnetic Waves (9.3 y 9.4)

\end{itemize}

\sphinxstepscope

\sphinxAtStartPar
MEC501 \sphinxhyphen{} Manejo y Conversión de Energía Solar Térmica


\chapter{Interacción materia\sphinxhyphen{}luz}
\label{\detokenize{3_Interacci_xf3n_materia-luz/3_Interacci_xf3n_materia-luz:interaccion-materia-luz}}\label{\detokenize{3_Interacci_xf3n_materia-luz/3_Interacci_xf3n_materia-luz::doc}}
\sphinxAtStartPar

Profesor: Francisco Ramírez CueSvas
Fecha: 26 de Agosto 2022


\section{Repaso de vibraciones mecánicas}
\label{\detokenize{3_Interacci_xf3n_materia-luz/3_Interacci_xf3n_materia-luz:repaso-de-vibraciones-mecanicas}}

\subsection{Frecuencia natural de un sistema vibratorio}
\label{\detokenize{3_Interacci_xf3n_materia-luz/3_Interacci_xf3n_materia-luz:frecuencia-natural-de-un-sistema-vibratorio}}
\sphinxAtStartPar
La frecuencia natural de un sistema vibratorio representa la frecuencia de oscilación del sistema en ausencia de amortiguación y fuerzas externas.

\sphinxAtStartPar
Por ejemplo, en el caso simplificado de un sistema masa resorte

\noindent{\hspace*{\fill}\sphinxincludegraphics[width=250\sphinxpxdimen]{{free_sistem_one_degree}.png}\hspace*{\fill}}

\sphinxAtStartPar
donde, \(k\) es la constante de rigidez del resorte, y \(m\) es la masa.

\sphinxAtStartPar
La ecuación gobernante está dada por:
\begin{equation*}
\ddot{x} + \omega_n^2 x = 0 
\end{equation*}
\sphinxAtStartPar
donde \(\omega_n = \sqrt{k/m}\), es la frecuencia natural del sistema


\subsection{Vibración forzada amortiguada con un grado de libertad}
\label{\detokenize{3_Interacci_xf3n_materia-luz/3_Interacci_xf3n_materia-luz:vibracion-forzada-amortiguada-con-un-grado-de-libertad}}
\sphinxAtStartPar
La frecuencia natural cobra relevancia cuando analizamos vibraciones forzadas.

\sphinxAtStartPar
Consideremos, por ejemplo, el sistema masa\sphinxhyphen{}resorte amortiguado, con constante de amortiguación \(c\), excitado por una fuerza externa oscilatoria de la forma \(F(t) = F_0 e^{i\omega t}\), donde \(F_0\) es una constante.

\noindent{\hspace*{\fill}\sphinxincludegraphics[width=150\sphinxpxdimen]{{forced_damped_system}.png}\hspace*{\fill}}

\sphinxAtStartPar
La ecuación gobernante de este sistema está dada por:
\begin{equation*}
\ddot{x} + 2\zeta\omega_n\dot{x} +  \omega_n^2 x = \frac{F_0}{m} e^{i\omega t},
\end{equation*}
\sphinxAtStartPar
donde \(\zeta = \frac{c}{2\omega_nm}\) es la razón de amortiguación

\sphinxAtStartPar
En su forma general, la solución a este sistema está dada \(x(t) = x_p(t) + x_h(t)\), donde:
\begin{eqnarray*}
x_h(t) &=& C e^{-\zeta \omega_n t}e^{i\omega_n\sqrt{1 - \zeta^2} t} &\quad&\mathrm{solución~homogenea}
\\[10pt]
x_p(t) &=& \frac{F_0/k}{\omega^2 - \omega_n^2 + 2i\zeta \omega_n \omega}e^{i\omega t} &\quad&\mathrm{solución~particular}
\end{eqnarray*}
\sphinxAtStartPar
donde \(C\) es una amplitud arbitraria definida por las condiciones iniciales del sistema.

\sphinxAtStartPar
La solución homogénea \(x_h(t)\), representa la oscilación libre del sistema. Debido al término \(e^{-\zeta \omega_n t}\), esta componente decae en el tiempo y se le conoce como \sphinxstylestrong{respuesta transciente}.

\sphinxAtStartPar
La solución particular \(x_p(t)\), por otro lado, representa la vibración en estado estacionario.

\noindent{\hspace*{\fill}\sphinxincludegraphics[width=400\sphinxpxdimen]{{forced_damped_motion}.png}\hspace*{\fill}}

\sphinxAtStartPar
Podemos repesentar la amplitud de \(x_p = Ae^{i\omega t}\), como \(A = |A|e^{i\phi}\), donde \(\phi\) es la fase. Asi, la solución estacionaria queda de la forma:
\begin{equation*}
x_p = |A|e^{i\left(\omega t + \phi\right)}
\end{equation*}
\sphinxAtStartPar
Cuando \(\omega/\omega_n = 1\), el sistema entra en resonancia. Esta respuesta se manifiesta con la característica amplificación de \(|A|\).

\noindent{\hspace*{\fill}\sphinxincludegraphics[width=600\sphinxpxdimen]{{resonance}.png}\hspace*{\fill}}


\section{Interacción de luz con moléculas}
\label{\detokenize{3_Interacci_xf3n_materia-luz/3_Interacci_xf3n_materia-luz:interaccion-de-luz-con-moleculas}}

\subsection{El oscilador armónico}
\label{\detokenize{3_Interacci_xf3n_materia-luz/3_Interacci_xf3n_materia-luz:el-oscilador-armonico}}
\sphinxAtStartPar
Consideremos la molecula de agua.

\noindent{\hspace*{\fill}\sphinxincludegraphics[width=400\sphinxpxdimen]{{water_molecule}.png}\hspace*{\fill}}

\sphinxAtStartPar
Esta molécula es del tipo polar, es decir, posee una carga eléctrica neta positiva en un extremo y negativa en otro. Esto ocurre debido a que la densidad de electrónes es mayor en la región cercana al núcleo del oxígeno.

\sphinxAtStartPar
El enlace entre el hidrógeno y el oxígeno es del tipo covalente. El siguiente modelo representa la energía potencial, \(U\) en función de la separación entre los núcleos \(r\):

\noindent{\hspace*{\fill}\sphinxincludegraphics[width=400\sphinxpxdimen]{{covalent_model}.png}\hspace*{\fill}}
\begin{equation*}
U(r) = \alpha_0e^{-r/\rho_0} - \frac{q_1q_2}{4\pi\varepsilon_0 r}
\end{equation*}
\sphinxAtStartPar
donde \(\alpha_0\) y \(\rho_0\) son constantes de ajuste, y \(q_1\) y \(q_2\) son, respectivamente, la carga eléctrica neta negativa y positiva.

\sphinxAtStartPar
En equilibrio, los núcleos pemanecen en la posición de equilibrio definida por \(r_0\)

\sphinxAtStartPar
Se puede demostrar que para oscilaciones pequeñas:
\begin{equation*}
U(r)\approx \frac{1}{2} k(r - r_0)^2
\end{equation*}
\sphinxAtStartPar
donde \(k = \frac{\partial^2 U}{\partial r^2}\big|_{r = r_0}\).

\sphinxAtStartPar
En otras palabras, la vibración de la molécula de agua puede ser representada como un oscilador armónico, con una fuerza de restauración definida por:
\begin{equation*}
F = -k(r - r_0)
\end{equation*}

\subsection{Modelo de Lorentz}
\label{\detokenize{3_Interacci_xf3n_materia-luz/3_Interacci_xf3n_materia-luz:modelo-de-lorentz}}
\sphinxAtStartPar
Podemos representar la interacción de la molécula con una onda electromagnética como una vibración forzada amortiguada.

\noindent{\hspace*{\fill}\sphinxincludegraphics[width=200\sphinxpxdimen]{{molecule_spring_mass}.png}\hspace*{\fill}}

\sphinxAtStartPar
Para una onda electromagnética de la forma \(E_0e^{-i\omega t}\), la fuerza sobre ejercida sobre cada polo es
\begin{equation*}
F = qE_0e^{-i\omega t},
\end{equation*}
\sphinxAtStartPar
donde \(q\) es la carga eléctrica del polo positivo (o negativo).

\sphinxAtStartPar
Por otro lado, la amortiguación del sistema surge a raíz de la colición entre los electrones y los nucleos, además de otras interacciónes electromagnéticas.

\sphinxAtStartPar
Asumiendo un eje de referencia situado en el polo positivo, la ecuación de movimiento está dada por:
\begin{equation*}
m\ddot{x} + m\Gamma \dot{x} +  k x = qE_0 e^{-i\omega t},
\end{equation*}
\sphinxAtStartPar
donde \(m\) es la masa del polo positivo, \(k\) es la constante de rigidez del enlace entre los polos, y \sphinxstylestrong{\(\Gamma\) es la taza de decaimiento} (se mide en unidades 1/s).

\sphinxAtStartPar
La solución estacionaria está dada por la solución particular:
\begin{equation*}
x_p(t) = \frac{q/mE_0}{\omega_n^2 - \omega^2 - i\Gamma \omega}e^{-i\omega t}
\end{equation*}
\sphinxAtStartPar
El desplazamiento del polo positivo respecto a su estado en equilibrio induce un \sphinxstylestrong{momento dipolar}, \(\vec{p}\), el cual expresamos a través de la relación:
\begin{equation*}
\vec{p} = q\vec{x}_p(t) =  \frac{q^2/m}{\omega_n^2 - \omega^2 - i\Gamma \omega}E_0e^{-i\omega t}\hat{e}\quad\mathrm{[C\cdot m]}
\end{equation*}
\sphinxAtStartPar
donde \(\hat{e}\) es la dirección del campo eléctrico

\sphinxAtStartPar
Definimos como \sphinxstylestrong{densidad de polarización}, \(\vec{P}\), al momento dipolar total por unidad de volumen:
\begin{equation*}
\vec{P} = N_p \vec{p} = \frac{N_pq^2/m}{\omega_n^2 - \omega^2 - i\Gamma \omega}\vec{E}\quad\mathrm{\left[\frac{C\cdot m}{m^3}\right]}
\end{equation*}
\sphinxAtStartPar
En presencia de un medio polarizado, la ley de Gauss se modifica como: \(\nabla\cdot\left(\varepsilon_0\vec{E} + \vec{P}\right) = 0\).

\sphinxAtStartPar
Representando esta relación en la forma, \(\nabla\cdot\varepsilon_0\varepsilon\vec{E} = \rho\), podemos deducir un modelo para la constante dieléctrica del sistema \(\varepsilon\):
\begin{equation*}
\varepsilon = 1 +\frac{\omega_p^2}{\omega_n^2 - \omega^2 - i\Gamma \omega},
\end{equation*}
\sphinxAtStartPar
con \(\omega_p^2 = \frac{N_pq^2}{\varepsilon_0 m}\)

\sphinxAtStartPar
En el caso del agua, la molecula posee un dipolo eléctrico neto adicional al dipolo inducido. El efecto de la polarización neta se puede representar, cambiando el primer término por un valor constante \(\varepsilon_\infty\).

\sphinxAtStartPar
El modelo completo se conoce como \sphinxstylestrong{modelo de Lorentz}:
\label{equation:3_Interacción_materia-luz/3_Interacción_materia-luz:104a7075-a176-4f2d-85dd-4c1eb1f6528b}\begin{equation}
\varepsilon = \varepsilon_\infty + \frac{\omega_p^2}{\omega_n^2 - \omega^2 - i\Gamma \omega}\quad\mathrm{Modelo~de~Lorentz},
\end{equation}
\sphinxAtStartPar
Por ejemplo, para \(\varepsilon_\infty = 2.0\), \(\Gamma = 0.1\omega_n\), y \(\omega_p = \omega_n\)

\noindent{\hspace*{\fill}\sphinxincludegraphics[width=800\sphinxpxdimen]{{Lorentz_model}.svg}\hspace*{\fill}}
\begin{quote}

\sphinxAtStartPar
\sphinxstylestrong{El modelo de Lorentz se utiliza como modelo de ajuste para representar la interacción de la luz con los modos vibratorios en la materia}
\end{quote}

\sphinxAtStartPar
Por ejemplo, la molécula de agua tiene 3 modos de vibración fundamentales en las longitudes de onda \(\lambda = \) 2.98, 2.93 y 5.91 \(\mu\)m (3351, 3412 y 1691 cm\(^{-1}\))

\begin{sphinxuseclass}{cell}\begin{sphinxVerbatimInput}

\begin{sphinxuseclass}{cell_input}
\begin{sphinxVerbatim}[commandchars=\\\{\}]
\PYG{k+kn}{from} \PYG{n+nn}{IPython}\PYG{n+nn}{.}\PYG{n+nn}{display} \PYG{k+kn}{import} \PYG{n}{YouTubeVideo}
\PYG{n}{YouTubeVideo}\PYG{p}{(}\PYG{l+s+s1}{\PYGZsq{}}\PYG{l+s+s1}{1uE2lvVkKW0}\PYG{l+s+s1}{\PYGZsq{}}\PYG{p}{,} \PYG{n}{width}\PYG{o}{=}\PYG{l+m+mi}{600}\PYG{p}{,} \PYG{n}{height}\PYG{o}{=}\PYG{l+m+mi}{400}\PYG{p}{,}  \PYG{n}{playsinline}\PYG{o}{=}\PYG{l+m+mi}{0}\PYG{p}{)}
\end{sphinxVerbatim}

\end{sphinxuseclass}\end{sphinxVerbatimInput}
\begin{sphinxVerbatimOutput}

\begin{sphinxuseclass}{cell_output}
\noindent\sphinxincludegraphics{{3_Interacción_materia-luz_38_0}.jpg}

\end{sphinxuseclass}\end{sphinxVerbatimOutput}

\end{sphinxuseclass}
\sphinxAtStartPar
El índice de refracción del agua muestra dos oscilaciones de Lorentz.

\noindent{\hspace*{\fill}\sphinxincludegraphics[width=800\sphinxpxdimen]{{h2o_nk}.svg}\hspace*{\fill}}

\sphinxAtStartPar
La resonancia en \(\lambda =\) 2.98 \(\mu\)m (3351 cm\(^{-1}\)) no está presente en el espectro.

\sphinxAtStartPar
Este modo no es compatible con la oscilación de una onda electromagnética plana. Así la luz no interactúa con esta vibración y, por lo tanto, no se ve representada en el espectro del índice de refracción.


\section{Interacción de la luz con materiales sólidos}
\label{\detokenize{3_Interacci_xf3n_materia-luz/3_Interacci_xf3n_materia-luz:interaccion-de-la-luz-con-materiales-solidos}}
\sphinxAtStartPar
El análisis anterior generalmente se aplica a gases, donde las moléculas no interactúan entre sí. En el caso de materiales sólidos, la interacción entre moléculas es fuerte, y genera bandas electrónicas de energía.
\begin{itemize}
\item {} 
\sphinxAtStartPar
\sphinxstylestrong{Banda de valencia}: corresponde a la banda ocupada por electrones con el mayor nivel de energía. En esta banda los electrones permanecen en un estado “ligado” al núcleo.

\item {} 
\sphinxAtStartPar
\sphinxstylestrong{Banda de conducción}: corresponde a la banda no ocupada por electrones con el menor nivel de energía. En esta banda los electrones se mueven líbremente por el material

\item {} 
\sphinxAtStartPar
\sphinxstylestrong{Banda prohibida (\sphinxstyleemphasis{band\sphinxhyphen{}gap})}: Es la diferencia entre la banda de conducción y la banda de valencia

\end{itemize}

\noindent{\hspace*{\fill}\sphinxincludegraphics[width=300\sphinxpxdimen]{{bandas_electronicas}.png}\hspace*{\fill}}

\sphinxAtStartPar
A partir de la separación entre la banda de conducción y la banda de valencia, podemos clasificar tres tipos de materiales en función de sus propiedades electrónicas:

\noindent{\hspace*{\fill}\sphinxincludegraphics[width=450\sphinxpxdimen]{{energy_bands_clasification}.png}\hspace*{\fill}}
\begin{itemize}
\item {} 
\sphinxAtStartPar
\sphinxstylestrong{Conductor}, donde las bandas de conducción y valencia están traslapadas (bandgap = 0). En estos materiales, parte de los electrones están alojados en la banda de conducción y, por lo tanto, son capaces de conducir corriente eléctrica en presencia de un campo eléctrico.

\item {} 
\sphinxAtStartPar
\sphinxstylestrong{Semiconductor}, donde las bandas de conducción y valencia están separadas. Sin embargo, la energía del bandgap es relativamente pequeña, de manera que un electrón puede ser llevado a la banda de conducción mediante un potencial eléctrico razonable, o mediante una onda electromagnética.

\item {} 
\sphinxAtStartPar
\sphinxstylestrong{Aislante}, donde las bandas de conducción y valencia están muy separadas. El umbral para excitar un electrón a la banda de conducción es demaciado grande y, por lo tanto, el material no es capaz de conducir corriente.

\end{itemize}

\sphinxAtStartPar
La respuesta óptica de cada tipo de material está condicionada por sus propiedades electrónicas


\subsection{Aislantes (modelo de Lorentz)}
\label{\detokenize{3_Interacci_xf3n_materia-luz/3_Interacci_xf3n_materia-luz:aislantes-modelo-de-lorentz}}
\sphinxAtStartPar
Debido a que los electrones en un aislante están fuertemente ligados al núcleo, la respuesta óptica de este material está condicionada por los modos de vibración de la red atómica. Así, la constante dieléctrica y el indice de refración siguen un comportamiento similar al modelo de Lorentz.

\sphinxAtStartPar
Por ejemplo, el sílice (SiO\(_2\))

\noindent{\hspace*{\fill}\sphinxincludegraphics[width=800\sphinxpxdimen]{{sio2_nk}.svg}\hspace*{\fill}}


\subsection{Conductores (modelo de Drude)}
\label{\detokenize{3_Interacci_xf3n_materia-luz/3_Interacci_xf3n_materia-luz:conductores-modelo-de-drude}}
\sphinxAtStartPar
En este caso los electrónes se mueven libremente por la red atómica.

\sphinxAtStartPar
Podemos representar la interacción de los electrones libres con una onda electromagnética utilizando la ecuación de movimiento. En este caso, la fuerza de restauración \(kx = 0\), y la ecuación es:
\begin{equation*}
m_e\ddot{x} + m_e\Gamma_e \dot{x} = eE_0 e^{-i\omega t},
\end{equation*}
\sphinxAtStartPar
donde \(m_e\), \(e\) y \(\Gamma_e\) son, respectivamente, la masa, la carga elemental y la taza de decaimiento del electrón.

\sphinxAtStartPar
Mediante un procedimiento similar al utilzado para el modelo de Lorentz, derivamos el \sphinxstylestrong{modelo de Drude} para conductores:
\label{equation:3_Interacción_materia-luz/3_Interacción_materia-luz:3b07a6e8-13e7-48a5-a2b2-6b7ed4ccd3da}\begin{equation}
\varepsilon = \varepsilon_\infty - \frac{\omega_p^2}{\omega^2 + i\Gamma_e \omega},\quad\mathrm{Modelo~de~Drude},
\end{equation}
\sphinxAtStartPar
donde \(\omega_p^2 = \frac{N_ee^2}{\varepsilon_0 m}\) se conoce como frecuencia de plasma, y \(N_e\) es la densidad de número de electrones. Similar al modelo de Lorentz, \(\varepsilon_\infty\) representa la polarización neta del material.

\sphinxAtStartPar
En el siguiente ejemplo, \(\varepsilon_\infty = 2.0\) y \(\Gamma_e = 0.1\omega_p\)

\noindent{\hspace*{\fill}\sphinxincludegraphics[width=800\sphinxpxdimen]{{drude_model}.svg}\hspace*{\fill}}

\sphinxAtStartPar
Una característica del modelo de Drude es que \(\varepsilon' < 0\) para \(\omega_p/\omega \gtrsim 1\). Esto se manifiesta en el índice de refracción como \(\kappa > n\)

\sphinxAtStartPar
Cuando \(\kappa > n\) la reflectividad aumenta significativamente. Así, los materiales conductores tienen alta reflectividad en ambas polarizaciones, cuando \(\omega_p/\omega \gtrsim 1\)

\noindent{\hspace*{\fill}\sphinxincludegraphics[width=450\sphinxpxdimen]{{reflectance_drude}.svg}\hspace*{\fill}}

\sphinxAtStartPar
En general, los metales pueden ser bien representados por el modelo de Drude. Por ejemplo, el aluminio:

\noindent{\hspace*{\fill}\sphinxincludegraphics[width=800\sphinxpxdimen]{{aluminium_nk}.svg}\hspace*{\fill}}

\sphinxAtStartPar
Notemos como para \(\lambda \approx 0.8\) \(\mu\)m, la respuesta del material se desvía del modelo de Drude. Esta respuesta esta asociada a un modo de vibración (modelo de Lorentz).

\sphinxAtStartPar
Además \(\omega_p \lesssim 0.1\) \(\mu\)m, y por lo tanto, refleja todas las longitudes de onda del espectro visible (\(\lambda \in [0.38,0.70]\) \(\mu\)m). Esto explica el efecto espejo del aluminio y otros metales


\subsection{Semiconductores (absorción interbanda)}
\label{\detokenize{3_Interacci_xf3n_materia-luz/3_Interacci_xf3n_materia-luz:semiconductores-absorcion-interbanda}}
\sphinxAtStartPar
En este caso las interacciones con ondas electromagnéticas están dictadas por bandas de absorción asociadas a la excitación de electrones de valencia a la banda de conducción.

\sphinxAtStartPar
Este fenómeno se conoce como \sphinxstylestrong{absorpción interbanda}, y ocurre cuando la energía del fotón \(\hbar\omega\) (\(\hbar = 6.58\times 10^16\) eV\(\cdot\)s) es superior al bandgap del material.

\noindent{\hspace*{\fill}\sphinxincludegraphics[width=300\sphinxpxdimen]{{photoexcited_electrons}.png}\hspace*{\fill}}

\sphinxAtStartPar
Los semiconductores son los materiales fundamentales en transistores, LED y celdas fotovoltaicas. El semiconductor más conocido es el silicio (Si).

\sphinxAtStartPar
El índice de refracción del silicio es:

\noindent{\hspace*{\fill}\sphinxincludegraphics[width=700\sphinxpxdimen]{{si_nk}.png}\hspace*{\fill}}

\sphinxAtStartPar
En general, los materiales pueden presentar más de un tipo de respuesta.

\sphinxAtStartPar
Por ejemplo, el oro tiene absorción interbanda en longitudes de onda \(\lambda < 0.5\) \(\mu\)m, combinado con el modelo de Drude.
\sphinxincludegraphics[width=350\sphinxpxdimen]{{gold_nk}.png}

\sphinxAtStartPar
Debido a esta respuesta, el oro absorbe las longitudes de onda correspondientes al azul y violeta, y refleja el resto de los colores.

\sphinxAtStartPar
La siguiente figura muestra el color del oro según el ángulo de incidencia en base al espectro de reflección de una interface aire/oro.
\sphinxincludegraphics[width=500\sphinxpxdimen]{{gold_color}.svg}

\sphinxAtStartPar
Otro ejemplo es el dioxido de titanio (TiO\(_2\)), el cual presenta absorción interbanda en el espectro ultravioleta, y oscilaciones de Lorentz en el infrarojo.

\noindent{\hspace*{\fill}\sphinxincludegraphics[width=700\sphinxpxdimen]{{tio2_nk}.png}\hspace*{\fill}}

\sphinxAtStartPar
Debido a la absorcion UV, TiO\(_2\) es muy utilizado en cremas para protección solar.


\section{Espectro electromagnético}
\label{\detokenize{3_Interacci_xf3n_materia-luz/3_Interacci_xf3n_materia-luz:espectro-electromagnetico}}
\sphinxAtStartPar
Podemos clasificar la radiación electromagnética según su longitud de onda (o frecuencia).

\noindent{\hspace*{\fill}\sphinxincludegraphics[width=700\sphinxpxdimen]{{em_spectrum}.png}\hspace*{\fill}}

\sphinxAtStartPar
Para este curso, los espectros más importantes son:


\begin{savenotes}\sphinxattablestart
\centering
\begin{tabulary}{\linewidth}[t]{|T|T|T|T|}
\hline
\sphinxstyletheadfamily 
\sphinxAtStartPar
Espectro
&\sphinxstyletheadfamily 
\sphinxAtStartPar
Longitud de onda
&\sphinxstyletheadfamily 
\sphinxAtStartPar
Frecuecia
&\sphinxstyletheadfamily 
\sphinxAtStartPar
Energía
\\
\hline
\sphinxAtStartPar
UIltravioleta (UV)
&
\sphinxAtStartPar
1 \sphinxhyphen{} 400 nm
&
\sphinxAtStartPar
100 \sphinxhyphen{} 0.75 PHz
&
\sphinxAtStartPar
414 \sphinxhyphen{} 3.1 eV
\\
\hline
\sphinxAtStartPar
Visible (vis)
&
\sphinxAtStartPar
400 \sphinxhyphen{} 750 nm
&
\sphinxAtStartPar
750 \sphinxhyphen{} 400 THz
&
\sphinxAtStartPar
3.1 \sphinxhyphen{} 1.65 eV
\\
\hline
\sphinxAtStartPar
Infrarrojo cercano (near\sphinxhyphen{}IR)
&
\sphinxAtStartPar
750 nm \sphinxhyphen{} 1.4 \(\mu\)m
&
\sphinxAtStartPar
400 \sphinxhyphen{} 214 THz
&
\sphinxAtStartPar
3.1 \sphinxhyphen{} 1.65 eV
\\
\hline
\sphinxAtStartPar
Infrarrojo de onda corta (SWIR)
&
\sphinxAtStartPar
1.4 \sphinxhyphen{} 3 \(\mu\)m
&
\sphinxAtStartPar
214 \sphinxhyphen{} 100 THz
&
\sphinxAtStartPar
885 \sphinxhyphen{} 414 meV
\\
\hline
\sphinxAtStartPar
Infrarrojo de onda media (MWIR)
&
\sphinxAtStartPar
3 \sphinxhyphen{} 8 \(\mu\)m
&
\sphinxAtStartPar
100 \sphinxhyphen{} 37 THz
&
\sphinxAtStartPar
414 \sphinxhyphen{} 153 meV
\\
\hline
\sphinxAtStartPar
Infrarrojo de onda larga (LWIR)
&
\sphinxAtStartPar
8 \sphinxhyphen{} 15 \(\mu\)m
&
\sphinxAtStartPar
37 \sphinxhyphen{}20 THz
&
\sphinxAtStartPar
153 eV \sphinxhyphen{} 82 meV
\\
\hline
\sphinxAtStartPar
Infrarrojo lejano (far\sphinxhyphen{}IR)
&
\sphinxAtStartPar
15 \sphinxhyphen{} 1000 \(\mu\)m
&
\sphinxAtStartPar
20 \sphinxhyphen{}0.3 THz
&
\sphinxAtStartPar
82 \sphinxhyphen{} 1.24 meV
\\
\hline
\end{tabulary}
\par
\sphinxattableend\end{savenotes}
\begin{itemize}
\item {} 
\sphinxAtStartPar
1 eV = \(1.602\times 10^{−19}\) J es un \sphinxstylestrong{electron volt}. Representa la energía cinética de un electron bajo un potencial de 1 volt.

\end{itemize}

\sphinxAtStartPar
El espectro SWIR + MWIR + LWIR se conoce también como \sphinxstylestrong{infrarrojo medio (mid\sphinxhyphen{}IR)}

\sphinxAtStartPar
La siguiente tabla es útil para la conversión de unidades:

\noindent{\hspace*{\fill}\sphinxincludegraphics[width=600\sphinxpxdimen]{{conversion_table}.png}\hspace*{\fill}}
\begin{itemize}
\item {} 
\sphinxAtStartPar
\(h = 4.136\times 10^{−15}\) eV/Hz es la \sphinxstylestrong{constante de Planck}.

\item {} 
\sphinxAtStartPar
\(\hbar = \frac{h}{2\pi} = 6.582\times10^{−16}\) eV/Hz es la \sphinxstylestrong{constante reducida de Planck o constante de Dirac}.

\end{itemize}

\sphinxAtStartPar
El espectro electromagnético nos sirve como referencia para clasificar el tipo de interacción con la materia. Así, en base a nuestra clásificación de modos de vibración de la materia, en general tenemos:
\begin{itemize}
\item {} 
\sphinxAtStartPar
Absorción interbanda: espectro UV y vis

\item {} 
\sphinxAtStartPar
Resonancia de plasma de electrones, \(\omega_p\) (modelo de Drude): espectro vis y near\sphinxhyphen{}IR

\item {} 
\sphinxAtStartPar
Resonancia de vibraciones moleculares, \(\omega_n\) (modelo de Lorentz): mid\sphinxhyphen{}IR

\end{itemize}


\section{Referencias}
\label{\detokenize{3_Interacci_xf3n_materia-luz/3_Interacci_xf3n_materia-luz:referencias}}\begin{itemize}
\item {} 
\sphinxAtStartPar
Rao S. S. \sphinxstylestrong{Chapter 4 \sphinxhyphen{} Vibration Under General Forcing Conditions} in \sphinxstyleemphasis{Mechanical Vibrations}, 6th Ed, Pearson, 2018

\item {} 
\sphinxAtStartPar
Griffths D., \sphinxstylestrong{Chapter 4.1 \sphinxhyphen{} Polarization} in \sphinxstyleemphasis{Introduction to Electrodynamics}, 4th Ed, Pearson, 2013

\item {} 
\sphinxAtStartPar
Simmons J. and Potter K., \sphinxstylestrong{Chapter 2 and 3} in \sphinxstyleemphasis{Optical Materials}, 1st Ed, Academic Press, 2000

\end{itemize}

\sphinxstepscope

\sphinxAtStartPar
MEC501 \sphinxhyphen{} Manejo y Conversión de Energía Solar Térmica


\chapter{Scattering electromagnético}
\label{\detokenize{4_Scattering/4_Scattering:scattering-electromagnetico}}\label{\detokenize{4_Scattering/4_Scattering::doc}}
\sphinxAtStartPar

Profesor: Francisco Ramírez Cuevas
Fecha: 2 de Septiembre 2022


\section{Interacción de luz según el tamaño de un cuerpo}
\label{\detokenize{4_Scattering/4_Scattering:interaccion-de-luz-segun-el-tamano-de-un-cuerpo}}
\sphinxAtStartPar
Hasta el momento hemos analizado las ecuaciones de Maxwell y condiciones de borde en coordenadas cartesianas. Estas relaciones se aplican a interfaces rectas.

\sphinxAtStartPar
En el caso de cuerpos curvos, los coeficientes de Fresnel y otras fórmulas relacionadas aún son aplicables, siembre y cuando el radio de curvartura del cuerpo \(R \gg \lambda\)

\noindent{\hspace*{\fill}\sphinxincludegraphics[width=500\sphinxpxdimen]{{fresnel_curvature}.png}\hspace*{\fill}}


\subsection{Interacción de luz con cuerpos grandes}
\label{\detokenize{4_Scattering/4_Scattering:interaccion-de-luz-con-cuerpos-grandes}}
\sphinxAtStartPar
A través de este principio podemos explicar la separación de colores en un arcoiris.

\sphinxAtStartPar
Primero, es importante notar que el índice de refracción del agua en el espectro visible no es constante. Este índice tiene un pequeño grado de dispersión, y decae a medida que la longitud de onda crece. Así, a partir de la ley de Snell, el ángulo de transmisión de cada onda (o color), crece proporcional a la longitud de onda.

\noindent{\hspace*{\fill}\sphinxincludegraphics[width=600\sphinxpxdimen]{{rainbow_refraction}.png}\hspace*{\fill}}

\sphinxAtStartPar
Este fenómeno produce que las ondas se separen en el espacio en función de su longitud de onda.

\sphinxAtStartPar
En una gota de agua, el efecto de separación de colores se magnifica a medida que la luz se refleja en el interior

\noindent{\hspace*{\fill}\sphinxincludegraphics[width=300\sphinxpxdimen]{{rainbows}.jpg}\hspace*{\fill}}


\subsection{Interacción de luz con cuerpos pequeños}
\label{\detokenize{4_Scattering/4_Scattering:interaccion-de-luz-con-cuerpos-pequenos}}
\sphinxAtStartPar
Cuándo las dimensiones del cuerpo, \(D\), son comparables a la longitud de onda, el radio de curvatura se hace significativo y las soluciones de las ecuaciones de Maxwell para una interface plana no son aplicables

\sphinxAtStartPar
En este caso, se produce el fenómeno de \sphinxstylestrong{\sphinxstyleemphasis{scattering} de luz} asociado a la disperción de luz en múltiples direcciones. Además del scattering, tenemos el fenómeno de \sphinxstylestrong{absorción de luz} asociada con la porción de la energía incidente absorbida por el objeto. Por último, llamamos \sphinxstylestrong{extinción de luz} a la suma de la energía de scattering y absorción.

\noindent{\hspace*{\fill}\sphinxincludegraphics[width=450\sphinxpxdimen]{{scattering_schematic}.png}\hspace*{\fill}}


\section{Scattering en esferas (solución de Mie)}
\label{\detokenize{4_Scattering/4_Scattering:scattering-en-esferas-solucion-de-mie}}
\sphinxAtStartPar
Consideremos el modelo simple una onda electromagnética interactuando con una esfera de radio \(R\) y diámetro \(D\) tal que \(D/\lambda \sim 1\)
\sphinxincludegraphics[width=300\sphinxpxdimen]{{em_wave_sphere}.png}

\sphinxAtStartPar
Llamaremos al índice de refracción de la esfera \(N_p\), y al índice de refracción del exterior \(N_h\).

\sphinxAtStartPar
En este caso asumimos que el índice de refracción del exterior no tiene componente compleja, es decir \(N_h = n_h\)

\sphinxAtStartPar
El espacio está definido en coordenadas esféricas, donde:
\begin{itemize}
\item {} 
\sphinxAtStartPar
\(\theta\): ángulo cenital

\item {} 
\sphinxAtStartPar
\(\phi\): ángulo azimutal

\item {} 
\sphinxAtStartPar
\(r\): posición radial

\end{itemize}

\sphinxAtStartPar
La solución, propuesta por Gustav Mie, se basa una expansión en serie de ondas esféricas \(\vec{M}_{lm}(r, \theta,\phi)\) y \(\vec{N}_{lm}(r, \theta,\phi)\) (más información en las referencias).

\sphinxAtStartPar
Por ejemplo, la componente del campo eléctrico correspondiente al scattering, \(\vec{E}_\mathrm{sca}\) es:
\begin{equation*}
\vec{E}_\mathrm{sca}(r, \theta,\phi) = \sum_{l=1}^\infty \mathrm{Im}\left[E_0\frac{2l+1}{l(l+1)}i^l\left(a_l \vec{N}_{l1}^{(3)} - b_l \vec{M}_{l1}^{(3)}\right)\right]
\end{equation*}
\sphinxAtStartPar
donde los coeficientes \(a_l\) y \(b_l\) están dados por la funciones de Ricatti\sphinxhyphen{}Bessel, \(\psi(x)\) y \(\xi(x)\), en la forma:
\begin{equation*}
a_l = \frac{p\psi_l(px)\psi'_l(x) - \psi_l(x)\psi'_l(px)}{p\psi_l(px)\xi'_l(x) - \xi_l(x)\psi'_l(px)},
\quad\quad\quad
b_l = \frac{\psi_l(px)\psi'_l(x) - p\psi_l(x)\psi'_l(px)}{\psi_l(px)\xi'_l(x) - p\xi_l(x)\psi'_l(px)},
\end{equation*}
\sphinxAtStartPar
donde \(x = n_hk_0R\), y \(p = N_p/n_h\).

\sphinxAtStartPar
El campo magnético está dado por \(\vec{H}_\mathrm{sca} = \frac{n_h}{Z_0} \left(\hat{k}\times\vec{E}_\mathrm{sca}\right)\).

\sphinxAtStartPar
A partir de la solución de Mie, deducimos que la intensidad y distribución del scattering electromagnético depende de dos parámetros:
\begin{itemize}
\item {} 
\sphinxAtStartPar
\(x = n_hk_0R\approx D/\lambda_h\), que representa la proporción entre el tamaño de la particula (\(D\)) y la longitud de onda en el medio circundante (\(\lambda_h= \lambda_0/n_h\))

\item {} 
\sphinxAtStartPar
\(px = N_pk_0R\approx D/\lambda_p\) que representa la proporción entre el tamaño de la particula y la longitud de onda dentro de la partícula (\(\lambda_p=\lambda_0/n_p\)).

\end{itemize}


\subsection{Distribución del campo eléctrico}
\label{\detokenize{4_Scattering/4_Scattering:distribucion-del-campo-electrico}}
\sphinxAtStartPar
A partir de esta solución, podemos visualizar la distribución del campo eléctrico durante el fenómeno de scattering.

\sphinxAtStartPar
La siguiente figura representa el scattering electromagnético a partir de la solución de Mie. La dirección de la onda incidente es \(\hat{k}_\mathrm{inc} = \hat{x}\), con el campo eléctrico polarizado en dirección \(\hat{e}=\hat{z}\). En la figura de la izquierda mostramos la distribución del campo electrico total, es decir el campo eléctrico incidente (\(\vec{E}_\mathrm{inc}\)) y de scattering (\(\vec{E}_\mathrm{sca}\)). En la figura de la derecha, hemos removido \(\vec{E}_\mathrm{inc}\) para poder visualizar \(\vec{E}_\mathrm{sca}\)

\noindent{\hspace*{\fill}\sphinxincludegraphics[width=600\sphinxpxdimen]{{scattering_distribution}.png}\hspace*{\fill}}

\sphinxAtStartPar
Utilizando la dirección de la onda incidente como referencia, podemos ver que la intensidad del scattering es mayor hacia adelante (\(\theta = 0^o\)) y decrece a medida de \(\theta\) aumenta. Debido a la simetría axial, el scattering no varía en \(\phi\).

\sphinxAtStartPar
En general, la distribución del scattering depende del tamaño de la partícula en relación la longitud de onda.

\noindent{\hspace*{\fill}\sphinxincludegraphics[width=550\sphinxpxdimen]{{scattering_vs_size}.png}\hspace*{\fill}}

\sphinxAtStartPar
Particularmente, cuando \(D/\lambda \ll 1\), se denomina Rayleight scattering. En este caso el campo scattering está distribuido uniformemente alrededor de la partícula


\subsection{Flujo de energía}
\label{\detokenize{4_Scattering/4_Scattering:flujo-de-energia}}
\sphinxAtStartPar
Al igual que con el estudio de reflexión y transmisión, la solución \(\vec{E}_\mathrm{sca}\) nos permite determinar el el vector de Poyinting asociado a scattering, \(\langle\vec{S_\mathrm{sca}}\rangle = \frac{1}{2}\mathrm{Re}\left(\vec{E}_\mathrm{sca}\times\vec{H}^*_\mathrm{sca}\right)~\mathrm{[W/m^2]}\).

\sphinxAtStartPar
Notar que, en general, \(\langle\vec{S_\mathrm{sca}}\rangle\) varía según \(\theta\), \(\phi\) y \(r\).

\noindent{\hspace*{\fill}\sphinxincludegraphics[width=300\sphinxpxdimen]{{poynting_vector}.png}\hspace*{\fill}}

\sphinxAtStartPar
La potencia neta por scattering, \(W_\mathrm{sca}\) se obtiene integrando \(\langle\vec{S_\mathrm{sca}}\rangle\) sobre la superticie de la esfera:
\begin{align*}
W_\mathrm{sca} &= \oint_{S} \langle\vec{S_\mathrm{sca}}\rangle\cdot\hat{r}dS 
\\
&= \int_0^{2\pi}\int_0^{\pi} \left[\langle\vec{S_\mathrm{sca}}\rangle \cdot \hat{r}\right]R^2 \sin\theta d\theta~d\phi
\\
&= \Phi_\mathrm{inc}~2\pi\int_0^{\pi}  P_\mathrm{sca}(\theta) \sin\theta d\theta
\quad\mathrm{[W]}
\end{align*}
\sphinxAtStartPar
donde \(\Phi_\mathrm{inc} = \frac{n_hE_0}{2Z_0}~\mathrm{[W/m^2]}\) es el flujo de energía o intensidad de la onda incidente, y \(P_\mathrm{sca}(\theta)  = \frac{R^2}{\Phi_\mathrm{inc}}\left[\langle\vec{S_\mathrm{sca}}\rangle \cdot \hat{r}\right]\), es la \sphinxstylestrong{función de distribución de scattering} o \sphinxstylestrong{función de fase}.

\sphinxAtStartPar
La función de fase se define como la \sphinxstylestrong{energía de scattering por unidad de ángulo sólido \(d\Omega = \sin\theta d\theta d\phi\) relativo al flujo de energía de la onda incidente, \(\Phi_\mathrm{inc}\)}.

\sphinxAtStartPar
En otras palabras, para una onda incidente con intensidad \(\Phi_\mathrm{inc}\), la energía de scattering en dirección \(\theta\) es \(\Phi_\mathrm{inc} P_\mathrm{sca}(\theta)d\Omega\)

\noindent{\hspace*{\fill}\sphinxincludegraphics[width=300\sphinxpxdimen]{{phase_function}.png}\hspace*{\fill}}

\sphinxAtStartPar
Mediante un proceso similar, podemos determinar la potencia extinguida, \(W_\mathrm{ext}\), a partir del campo total \(\vec{E}_\mathrm{tot} = \vec{E}_\mathrm{inc} + \vec{E}_\mathrm{sca}\)

\sphinxAtStartPar
Al igual que con los coeficientes de Fresnel, es común definir la energía relativa a \(I_\mathrm{inc}\):
\begin{eqnarray*}
C_\mathrm{sca} &=& \frac{W_\mathrm{sca}}{I_\mathrm{inc}} = \frac{2\pi}{k^2}\sum_{l=1}^\infty (2l+1)\left(|a_l|^2 + |b_l|^2\right)&\quad&\mathrm{[m^2]}
\\[10pt]
C_\mathrm{ext} &=& \frac{W_\mathrm{ext}}{I_\mathrm{inc}} = \frac{2\pi}{k^2}\sum_{l=1}^\infty (2l+1)\mathrm{Re}\left(a_l + b_l\right)&\quad&\mathrm{[m^2]}
\end{eqnarray*}
\sphinxAtStartPar
debido a que \(C_\mathrm{sca}\) y \(C_\mathrm{ext}\) están definidos en unidades de área, se denominan \sphinxstylestrong{sección transversal de scattering y extinción, respectivamente}.

\sphinxAtStartPar
Por conservación de energía, la sección transversal de absorción, \(C_\mathrm{abs} = C_\mathrm{ext} - C_\mathrm{sca}\).


\subsection{Parámetro de asimetría}
\label{\detokenize{4_Scattering/4_Scattering:parametro-de-asimetria}}
\sphinxAtStartPar
El parámetro de asimetría, \(\mu_\mathrm{sca} \in [-1,1]\), nos permite cuantificar la anisotropía en la distribución del scattering.

\noindent{\hspace*{\fill}\sphinxincludegraphics[width=550\sphinxpxdimen]{{asymmetry_parameter}.png}\hspace*{\fill}}

\sphinxAtStartPar
En el caso de esferas, se define por:
\begin{equation*}
\mu_\mathrm{sca} = \frac{4\pi}{k^2C_\mathrm{sca}}\left[
\sum_{l=1}^\infty \frac{l(l+2)}{l+1}\mathrm{Re}\left(a_la_{l+1}^* + b_lb_{l+1}^*\right) +
\sum_{l=1}^\infty\frac{2l+1}{l(l+1)}\mathrm{Re}\left(a_lb_l^*\right)
\right]
\end{equation*}

\section{Analisis de scattering}
\label{\detokenize{4_Scattering/4_Scattering:analisis-de-scattering}}
\sphinxAtStartPar
Los parámetros \(C_\mathrm{sca}\), \(C_\mathrm{abs}\) y \(C_\mathrm{ext}\) permiten cuantificar la energía de scattering, absorción y extinción relativa a la intensidad de la fuente \(I_\mathrm{inc}\), así como también su distribución en el espectro.


\subsection{Particulas con índice de refracción real (dieléctricos)}
\label{\detokenize{4_Scattering/4_Scattering:particulas-con-indice-de-refraccion-real-dielectricos}}
\sphinxAtStartPar
Por ejemplo, analicemos el scattering de una esfera de agua (\(N_p\approx 1.33\)) en el aire (\(n_h = 1.0\)).

\sphinxAtStartPar
Notar que \(N_p\approx 1.33\) implica \(C_\mathrm{abs} = 0\)

\begin{sphinxuseclass}{cell}\begin{sphinxVerbatimInput}

\begin{sphinxuseclass}{cell_input}
\begin{sphinxVerbatim}[commandchars=\\\{\}]
\PYG{o}{\PYGZpc{}\PYGZpc{}capture} show\PYGZus{}plot
\PYG{k+kn}{import} \PYG{n+nn}{empylib}\PYG{n+nn}{.}\PYG{n+nn}{miescattering} \PYG{k}{as} \PYG{n+nn}{mie}
\PYG{k+kn}{import} \PYG{n+nn}{matplotlib}\PYG{n+nn}{.}\PYG{n+nn}{pyplot} \PYG{k}{as} \PYG{n+nn}{plt}
\PYG{k+kn}{import} \PYG{n+nn}{numpy} \PYG{k}{as} \PYG{n+nn}{np}

\PYG{n}{lam} \PYG{o}{=} \PYG{n}{np}\PYG{o}{.}\PYG{n}{linspace}\PYG{p}{(}\PYG{l+m+mf}{0.3}\PYG{p}{,}\PYG{l+m+mf}{1.4}\PYG{p}{,}\PYG{l+m+mi}{200}\PYG{p}{)}  \PYG{c+c1}{\PYGZsh{} espectro de longitudes de onda}
\PYG{n}{nh} \PYG{o}{=} \PYG{l+m+mf}{1.0}                        \PYG{c+c1}{\PYGZsh{} índice de refracción del material circundante}
\PYG{n}{Np} \PYG{o}{=} \PYG{l+m+mf}{1.33}                       \PYG{c+c1}{\PYGZsh{} índice de refracción de la partícula}
\PYG{n}{D} \PYG{o}{=} \PYG{p}{[}\PYG{l+m+mf}{0.1}\PYG{p}{,} \PYG{l+m+mf}{0.3}\PYG{p}{,} \PYG{l+m+mf}{0.5}\PYG{p}{,} \PYG{l+m+mf}{0.7}\PYG{p}{,} \PYG{l+m+mf}{1.0}\PYG{p}{]}   \PYG{c+c1}{\PYGZsh{} distribución de diámetros }

\PYG{n}{fig}\PYG{p}{,} \PYG{n}{ax} \PYG{o}{=} \PYG{n}{plt}\PYG{o}{.}\PYG{n}{subplots}\PYG{p}{(}\PYG{p}{)}                \PYG{c+c1}{\PYGZsh{} creamos ejes para graficar}
\PYG{n}{colors} \PYG{o}{=} \PYG{n}{plt}\PYG{o}{.}\PYG{n}{cm}\PYG{o}{.}\PYG{n}{jet}\PYG{p}{(}\PYG{n}{np}\PYG{o}{.}\PYG{n}{linspace}\PYG{p}{(}\PYG{l+m+mi}{0}\PYG{p}{,}\PYG{l+m+mi}{1}\PYG{p}{,}\PYG{n+nb}{len}\PYG{p}{(}\PYG{n}{D}\PYG{p}{)}\PYG{p}{)}\PYG{p}{)} \PYG{c+c1}{\PYGZsh{} set de colores para las curvas}
\PYG{k}{for} \PYG{n}{i} \PYG{o+ow}{in} \PYG{n+nb}{range}\PYG{p}{(}\PYG{n+nb}{len}\PYG{p}{(}\PYG{n}{D}\PYG{p}{)}\PYG{p}{)}\PYG{p}{:}
    \PYG{n}{Ac} \PYG{o}{=} \PYG{n}{np}\PYG{o}{.}\PYG{n}{pi}\PYG{o}{*}\PYG{n}{D}\PYG{p}{[}\PYG{n}{i}\PYG{p}{]}\PYG{o}{*}\PYG{o}{*}\PYG{l+m+mi}{2}\PYG{o}{/}\PYG{l+m+mi}{4}                \PYG{c+c1}{\PYGZsh{} área transversal de la partícula}
    \PYG{n}{Qsca} \PYG{o}{=} \PYG{n}{mie}\PYG{o}{.}\PYG{n}{scatter\PYGZus{}efficiency}\PYG{p}{(}\PYG{n}{lam}\PYG{p}{,}\PYG{n}{nh}\PYG{p}{,}\PYG{n}{Np}\PYG{p}{,}\PYG{n}{D}\PYG{p}{[}\PYG{n}{i}\PYG{p}{]}\PYG{p}{)}\PYG{p}{[}\PYG{l+m+mi}{1}\PYG{p}{]} \PYG{c+c1}{\PYGZsh{} determinamos Csca/Ac}
    \PYG{n}{ax}\PYG{o}{.}\PYG{n}{plot}\PYG{p}{(}\PYG{n}{lam}\PYG{p}{,}\PYG{n}{Qsca}\PYG{o}{*}\PYG{n}{Ac}\PYG{p}{,}\PYG{l+s+s1}{\PYGZsq{}}\PYG{l+s+s1}{\PYGZhy{}}\PYG{l+s+s1}{\PYGZsq{}}\PYG{p}{,} \PYG{n}{color}\PYG{o}{=}\PYG{n}{colors}\PYG{p}{[}\PYG{n}{i}\PYG{p}{]}\PYG{p}{,} \PYG{n}{label}\PYG{o}{=}\PYG{p}{(}\PYG{l+s+s1}{\PYGZsq{}}\PYG{l+s+si}{\PYGZpc{}i}\PYG{l+s+s1}{\PYGZsq{}} \PYG{o}{\PYGZpc{}} \PYG{p}{(}\PYG{n}{D}\PYG{p}{[}\PYG{n}{i}\PYG{p}{]}\PYG{o}{*}\PYG{l+m+mf}{1E3}\PYG{p}{)}\PYG{p}{)}\PYG{p}{)} \PYG{c+c1}{\PYGZsh{} grafico Csca}

\PYG{c+c1}{\PYGZsh{} etiquetas de ejes y formateo de la figura}
\PYG{n}{fig}\PYG{o}{.}\PYG{n}{set\PYGZus{}size\PYGZus{}inches}\PYG{p}{(}\PYG{l+m+mi}{6}\PYG{p}{,} \PYG{l+m+mi}{4}\PYG{p}{)}         \PYG{c+c1}{\PYGZsh{} tamaño de figura}
\PYG{n}{plt}\PYG{o}{.}\PYG{n}{rcParams}\PYG{p}{[}\PYG{l+s+s1}{\PYGZsq{}}\PYG{l+s+s1}{font.size}\PYG{l+s+s1}{\PYGZsq{}}\PYG{p}{]} \PYG{o}{=} \PYG{l+s+s1}{\PYGZsq{}}\PYG{l+s+s1}{14}\PYG{l+s+s1}{\PYGZsq{}}   \PYG{c+c1}{\PYGZsh{} tamaño de fuente}
\PYG{n}{ax}\PYG{o}{.}\PYG{n}{set\PYGZus{}xlabel}\PYG{p}{(}\PYG{l+s+sa}{r}\PYG{l+s+s1}{\PYGZsq{}}\PYG{l+s+s1}{Longitud de onda, \PYGZdl{}}\PYG{l+s+s1}{\PYGZbs{}}\PYG{l+s+s1}{lambda\PYGZdl{} (\PYGZdl{}}\PYG{l+s+s1}{\PYGZbs{}}\PYG{l+s+s1}{mu\PYGZdl{}m)}\PYG{l+s+s1}{\PYGZsq{}}\PYG{p}{,} \PYG{n}{fontsize}\PYG{o}{=}\PYG{l+m+mi}{16}\PYG{p}{)}
\PYG{n}{ax}\PYG{o}{.}\PYG{n}{set\PYGZus{}title}\PYG{p}{(}\PYG{l+s+s1}{\PYGZsq{}}\PYG{l+s+s1}{scattering partícula de agua}\PYG{l+s+s1}{\PYGZsq{}}\PYG{p}{)}
\PYG{n}{ax}\PYG{o}{.}\PYG{n}{set\PYGZus{}ylabel}\PYG{p}{(}\PYG{l+s+sa}{r}\PYG{l+s+s1}{\PYGZsq{}}\PYG{l+s+s1}{\PYGZdl{}C\PYGZus{}}\PYG{l+s+s1}{\PYGZbs{}}\PYG{l+s+s1}{mathrm}\PYG{l+s+si}{\PYGZob{}sca\PYGZcb{}}\PYG{l+s+s1}{\PYGZdl{} (\PYGZdl{}}\PYG{l+s+s1}{\PYGZbs{}}\PYG{l+s+s1}{mu\PYGZdl{}m\PYGZdl{}\PYGZca{}2\PYGZdl{})}\PYG{l+s+s1}{\PYGZsq{}}\PYG{p}{,} \PYG{n}{fontsize}\PYG{o}{=}\PYG{l+m+mi}{16}\PYG{p}{)}
\PYG{n}{ax}\PYG{o}{.}\PYG{n}{legend}\PYG{p}{(}\PYG{n}{frameon}\PYG{o}{=}\PYG{k+kc}{False}\PYG{p}{,} \PYG{n}{title}\PYG{o}{=}\PYG{l+s+sa}{r}\PYG{l+s+s1}{\PYGZsq{}}\PYG{l+s+s1}{\PYGZdl{}D\PYGZdl{} (\PYGZdl{}}\PYG{l+s+s1}{\PYGZbs{}}\PYG{l+s+s1}{mu\PYGZdl{}m)}\PYG{l+s+s1}{\PYGZsq{}}\PYG{p}{)}
\PYG{n}{plt}\PYG{o}{.}\PYG{n}{show}\PYG{p}{(}\PYG{p}{)}
\end{sphinxVerbatim}

\end{sphinxuseclass}\end{sphinxVerbatimInput}

\end{sphinxuseclass}
\begin{sphinxuseclass}{cell}\begin{sphinxVerbatimInput}

\begin{sphinxuseclass}{cell_input}
\begin{sphinxVerbatim}[commandchars=\\\{\}]
\PYG{n}{show\PYGZus{}plot}\PYG{p}{(}\PYG{p}{)}
\end{sphinxVerbatim}

\end{sphinxuseclass}\end{sphinxVerbatimInput}
\begin{sphinxVerbatimOutput}

\begin{sphinxuseclass}{cell_output}
\noindent\sphinxincludegraphics{{4_Scattering_39_0}.png}

\end{sphinxuseclass}\end{sphinxVerbatimOutput}

\end{sphinxuseclass}
\sphinxAtStartPar
A partir de este gráfico podemos identificar algunos patrones comúnes en scattering:
\begin{itemize}
\item {} 
\sphinxAtStartPar
La energía de scattering aumenta con el tamaño de la partícula

\item {} 
\sphinxAtStartPar
A medida que el tamaño aumenta, la longitud de onda para scattering máximo crece (\sphinxstyleemphasis{red\sphinxhyphen{}shifting})

\end{itemize}

\sphinxAtStartPar
Esta es una característica general del scattering de partículas dieléctricas.

\sphinxAtStartPar
A partir de este gráfico podemos entender muchas situaciones de la vida cotidiana.

\sphinxAtStartPar
Por ejemplo, en la niebla las partículas de agua tienen un tamaño microscópico (\(D\sim 1\mu\)m) y, por lo tanto, dispersan la mayor parte de la luz visible.

\noindent{\hspace*{\fill}\sphinxincludegraphics[width=600\sphinxpxdimen]{{difuse_and_specular}.png}\hspace*{\fill}}

\sphinxAtStartPar
Para un haz de luz incidente en un medio con partículas, llamamos \sphinxstylestrong{componente difusa} a la porción de la luz dispersada por scattering, y como \sphinxstylestrong{componente especular} a la porción no dispersada.

\sphinxAtStartPar
En el cielo, en cambio, las moleculas del aire son mucho más pequeñas, y el scattering es más intenso para ondas en el espectro del color azul y violeta (\(\lambda < 450\) nm)

\noindent{\hspace*{\fill}\sphinxincludegraphics[width=600\sphinxpxdimen]{{sky_scattering}.png}\hspace*{\fill}}

\sphinxAtStartPar
La componente difusa, así, corresponde a los colores azul y violeta. La componente especular, corresponde al resto de los colores del espectro visible. El fenómeno expica el color azul del cielo durante el día.


\subsection{Parículas metálicas}
\label{\detokenize{4_Scattering/4_Scattering:pariculas-metalicas}}
\sphinxAtStartPar
El naturaleza del scattering es diferente para los metales. En este caso, el movimiento libre de los electrones genera acumulación de carga en la superficie de la partícula. Como resultado, la partícula se polariza generando fenómenos de resonancia en determinadas longitudes de onda.

\noindent{\hspace*{\fill}\sphinxincludegraphics[width=350\sphinxpxdimen]{{surface_plasmon}.png}\hspace*{\fill}}

\sphinxAtStartPar
En la siguiente figura, graficamos \(C_\mathrm{sca}\) y \(C_\mathrm{abs}\) para partículas de distinto diámetro. Ambas variables son normalizadas por al área transversal de la esfera \(\pi R^2\), para mejor comparación entre esferas de distintas dimensiones.

\begin{sphinxuseclass}{cell}\begin{sphinxVerbatimInput}

\begin{sphinxuseclass}{cell_input}
\begin{sphinxVerbatim}[commandchars=\\\{\}]
\PYG{o}{\PYGZpc{}\PYGZpc{}capture} show\PYGZus{}plot
\PYG{k+kn}{import} \PYG{n+nn}{empylib}\PYG{n+nn}{.}\PYG{n+nn}{miescattering} \PYG{k}{as} \PYG{n+nn}{mie}
\PYG{k+kn}{import} \PYG{n+nn}{empylib}\PYG{n+nn}{.}\PYG{n+nn}{nklib} \PYG{k}{as} \PYG{n+nn}{nk}
\PYG{k+kn}{import} \PYG{n+nn}{matplotlib}\PYG{n+nn}{.}\PYG{n+nn}{pyplot} \PYG{k}{as} \PYG{n+nn}{plt}
\PYG{k+kn}{import} \PYG{n+nn}{numpy} \PYG{k}{as} \PYG{n+nn}{np}

\PYG{n}{lam} \PYG{o}{=} \PYG{n}{np}\PYG{o}{.}\PYG{n}{linspace}\PYG{p}{(}\PYG{l+m+mf}{0.2}\PYG{p}{,}\PYG{l+m+mf}{0.8}\PYG{p}{,}\PYG{l+m+mi}{200}\PYG{p}{)}     \PYG{c+c1}{\PYGZsh{} espectro de longitudes de onda}
\PYG{n}{nh} \PYG{o}{=} \PYG{l+m+mf}{1.0}                           \PYG{c+c1}{\PYGZsh{} índice de refracción del material circundante}
\PYG{n}{Np} \PYG{o}{=} \PYG{n}{nk}\PYG{o}{.}\PYG{n}{silver}\PYG{p}{(}\PYG{n}{lam}\PYG{p}{)}                \PYG{c+c1}{\PYGZsh{} índice de refracción de la partícula}
\PYG{n}{D} \PYG{o}{=} \PYG{p}{[}\PYG{l+m+mf}{0.01}\PYG{p}{,} \PYG{l+m+mf}{0.02}\PYG{p}{,} \PYG{l+m+mf}{0.05}\PYG{p}{,} \PYG{l+m+mf}{0.08}\PYG{p}{,} \PYG{l+m+mf}{0.1}\PYG{p}{]}  \PYG{c+c1}{\PYGZsh{} distribución de diámetros }

\PYG{n}{fig}\PYG{p}{,} \PYG{n}{ax} \PYG{o}{=} \PYG{n}{plt}\PYG{o}{.}\PYG{n}{subplots}\PYG{p}{(}\PYG{l+m+mi}{1}\PYG{p}{,}\PYG{l+m+mi}{2}\PYG{p}{)}        \PYG{c+c1}{\PYGZsh{} creamos ejes para graficar}
\PYG{n}{colors} \PYG{o}{=} \PYG{n}{plt}\PYG{o}{.}\PYG{n}{cm}\PYG{o}{.}\PYG{n}{jet}\PYG{p}{(}\PYG{n}{np}\PYG{o}{.}\PYG{n}{linspace}\PYG{p}{(}\PYG{l+m+mi}{0}\PYG{p}{,}\PYG{l+m+mi}{1}\PYG{p}{,}\PYG{n+nb}{len}\PYG{p}{(}\PYG{n}{D}\PYG{p}{)}\PYG{p}{)}\PYG{p}{)} \PYG{c+c1}{\PYGZsh{} set de colores para las curvas}
\PYG{k}{for} \PYG{n}{i} \PYG{o+ow}{in} \PYG{n+nb}{range}\PYG{p}{(}\PYG{n+nb}{len}\PYG{p}{(}\PYG{n}{D}\PYG{p}{)}\PYG{p}{)}\PYG{p}{:}
    \PYG{n}{Qext}\PYG{p}{,} \PYG{n}{Qsca} \PYG{o}{=} \PYG{n}{mie}\PYG{o}{.}\PYG{n}{scatter\PYGZus{}efficiency}\PYG{p}{(}\PYG{n}{lam}\PYG{p}{,}\PYG{n}{nh}\PYG{p}{,}\PYG{n}{Np}\PYG{p}{,}\PYG{n}{D}\PYG{p}{[}\PYG{n}{i}\PYG{p}{]}\PYG{p}{)}\PYG{p}{[}\PYG{l+m+mi}{0}\PYG{p}{:}\PYG{l+m+mi}{2}\PYG{p}{]} \PYG{c+c1}{\PYGZsh{} determinamos Cext/Ac y Csca/Ac}
    \PYG{n}{Qabs} \PYG{o}{=} \PYG{n}{Qext} \PYG{o}{\PYGZhy{}} \PYG{n}{Qsca}
    \PYG{n}{ax}\PYG{p}{[}\PYG{l+m+mi}{0}\PYG{p}{]}\PYG{o}{.}\PYG{n}{plot}\PYG{p}{(}\PYG{n}{lam}\PYG{p}{,}\PYG{n}{Qsca}\PYG{p}{,}\PYG{l+s+s1}{\PYGZsq{}}\PYG{l+s+s1}{\PYGZhy{}}\PYG{l+s+s1}{\PYGZsq{}}\PYG{p}{,} \PYG{n}{color}\PYG{o}{=}\PYG{n}{colors}\PYG{p}{[}\PYG{n}{i}\PYG{p}{]}\PYG{p}{,} \PYG{n}{label}\PYG{o}{=}\PYG{p}{(}\PYG{l+s+s1}{\PYGZsq{}}\PYG{l+s+si}{\PYGZpc{}i}\PYG{l+s+s1}{\PYGZsq{}} \PYG{o}{\PYGZpc{}} \PYG{p}{(}\PYG{n}{D}\PYG{p}{[}\PYG{n}{i}\PYG{p}{]}\PYG{o}{*}\PYG{l+m+mf}{1E3}\PYG{p}{)}\PYG{p}{)}\PYG{p}{)} \PYG{c+c1}{\PYGZsh{} grafico Csca/Ac}
    \PYG{n}{ax}\PYG{p}{[}\PYG{l+m+mi}{1}\PYG{p}{]}\PYG{o}{.}\PYG{n}{plot}\PYG{p}{(}\PYG{n}{lam}\PYG{p}{,}\PYG{n}{Qabs}\PYG{p}{,}\PYG{l+s+s1}{\PYGZsq{}}\PYG{l+s+s1}{\PYGZhy{}}\PYG{l+s+s1}{\PYGZsq{}}\PYG{p}{,} \PYG{n}{color}\PYG{o}{=}\PYG{n}{colors}\PYG{p}{[}\PYG{n}{i}\PYG{p}{]}\PYG{p}{,} \PYG{n}{label}\PYG{o}{=}\PYG{p}{(}\PYG{l+s+s1}{\PYGZsq{}}\PYG{l+s+si}{\PYGZpc{}i}\PYG{l+s+s1}{\PYGZsq{}} \PYG{o}{\PYGZpc{}} \PYG{p}{(}\PYG{n}{D}\PYG{p}{[}\PYG{n}{i}\PYG{p}{]}\PYG{o}{*}\PYG{l+m+mf}{1E3}\PYG{p}{)}\PYG{p}{)}\PYG{p}{)} \PYG{c+c1}{\PYGZsh{} grafico Cabs/Ac}

\PYG{c+c1}{\PYGZsh{} etiquetas de ejes y formateo de la figura}
\PYG{n}{fig}\PYG{o}{.}\PYG{n}{set\PYGZus{}size\PYGZus{}inches}\PYG{p}{(}\PYG{l+m+mi}{13}\PYG{p}{,} \PYG{l+m+mi}{5}\PYG{p}{)}         \PYG{c+c1}{\PYGZsh{} tamaño de figura}
\PYG{n}{plt}\PYG{o}{.}\PYG{n}{rcParams}\PYG{p}{[}\PYG{l+s+s1}{\PYGZsq{}}\PYG{l+s+s1}{font.size}\PYG{l+s+s1}{\PYGZsq{}}\PYG{p}{]} \PYG{o}{=} \PYG{l+s+s1}{\PYGZsq{}}\PYG{l+s+s1}{14}\PYG{l+s+s1}{\PYGZsq{}}   \PYG{c+c1}{\PYGZsh{} tamaño de fuente}

\PYG{k}{for} \PYG{n}{i} \PYG{o+ow}{in} \PYG{n+nb}{range}\PYG{p}{(}\PYG{l+m+mi}{2}\PYG{p}{)}\PYG{p}{:}
    \PYG{n}{ax}\PYG{p}{[}\PYG{n}{i}\PYG{p}{]}\PYG{o}{.}\PYG{n}{set\PYGZus{}xlabel}\PYG{p}{(}\PYG{l+s+sa}{r}\PYG{l+s+s1}{\PYGZsq{}}\PYG{l+s+s1}{Longitud de onda, \PYGZdl{}}\PYG{l+s+s1}{\PYGZbs{}}\PYG{l+s+s1}{lambda\PYGZdl{} (\PYGZdl{}}\PYG{l+s+s1}{\PYGZbs{}}\PYG{l+s+s1}{mu\PYGZdl{}m)}\PYG{l+s+s1}{\PYGZsq{}}\PYG{p}{,} \PYG{n}{fontsize}\PYG{o}{=}\PYG{l+m+mi}{16}\PYG{p}{)}
    \PYG{n}{ax}\PYG{p}{[}\PYG{n}{i}\PYG{p}{]}\PYG{o}{.}\PYG{n}{set\PYGZus{}ylim}\PYG{p}{(}\PYG{l+m+mi}{0}\PYG{p}{,}\PYG{l+m+mf}{6.2}\PYG{p}{)}

\PYG{n}{ax}\PYG{p}{[}\PYG{l+m+mi}{0}\PYG{p}{]}\PYG{o}{.}\PYG{n}{set\PYGZus{}title}\PYG{p}{(}\PYG{l+s+s1}{\PYGZsq{}}\PYG{l+s+s1}{Scattering partícula de plata}\PYG{l+s+s1}{\PYGZsq{}}\PYG{p}{,} \PYG{n}{fontsize}\PYG{o}{=}\PYG{l+m+mi}{16}\PYG{p}{)}
\PYG{n}{ax}\PYG{p}{[}\PYG{l+m+mi}{1}\PYG{p}{]}\PYG{o}{.}\PYG{n}{set\PYGZus{}title}\PYG{p}{(}\PYG{l+s+s1}{\PYGZsq{}}\PYG{l+s+s1}{Absorción partícula de plata}\PYG{l+s+s1}{\PYGZsq{}}\PYG{p}{,} \PYG{n}{fontsize}\PYG{o}{=}\PYG{l+m+mi}{16}\PYG{p}{)}
\PYG{n}{ax}\PYG{p}{[}\PYG{l+m+mi}{0}\PYG{p}{]}\PYG{o}{.}\PYG{n}{set\PYGZus{}ylabel}\PYG{p}{(}\PYG{l+s+sa}{r}\PYG{l+s+s1}{\PYGZsq{}}\PYG{l+s+s1}{\PYGZdl{}C\PYGZus{}}\PYG{l+s+s1}{\PYGZbs{}}\PYG{l+s+s1}{mathrm}\PYG{l+s+si}{\PYGZob{}sca\PYGZcb{}}\PYG{l+s+s1}{ / }\PYG{l+s+s1}{\PYGZbs{}}\PYG{l+s+s1}{pi R\PYGZca{}2\PYGZdl{}}\PYG{l+s+s1}{\PYGZsq{}}\PYG{p}{,} \PYG{n}{fontsize}\PYG{o}{=}\PYG{l+m+mi}{16}\PYG{p}{)}
\PYG{n}{ax}\PYG{p}{[}\PYG{l+m+mi}{1}\PYG{p}{]}\PYG{o}{.}\PYG{n}{set\PYGZus{}ylabel}\PYG{p}{(}\PYG{l+s+sa}{r}\PYG{l+s+s1}{\PYGZsq{}}\PYG{l+s+s1}{\PYGZdl{}C\PYGZus{}}\PYG{l+s+s1}{\PYGZbs{}}\PYG{l+s+s1}{mathrm}\PYG{l+s+si}{\PYGZob{}abs\PYGZcb{}}\PYG{l+s+s1}{ / }\PYG{l+s+s1}{\PYGZbs{}}\PYG{l+s+s1}{pi R\PYGZca{}2\PYGZdl{}}\PYG{l+s+s1}{\PYGZsq{}}\PYG{p}{,} \PYG{n}{fontsize}\PYG{o}{=}\PYG{l+m+mi}{16}\PYG{p}{)}
\PYG{n}{ax}\PYG{p}{[}\PYG{l+m+mi}{1}\PYG{p}{]}\PYG{o}{.}\PYG{n}{legend}\PYG{p}{(}\PYG{n}{frameon}\PYG{o}{=}\PYG{k+kc}{False}\PYG{p}{,} \PYG{n}{title}\PYG{o}{=}\PYG{l+s+sa}{r}\PYG{l+s+s1}{\PYGZsq{}}\PYG{l+s+s1}{D (nm)}\PYG{l+s+s1}{\PYGZsq{}}\PYG{p}{)}
\PYG{n}{plt}\PYG{o}{.}\PYG{n}{show}\PYG{p}{(}\PYG{p}{)}
\end{sphinxVerbatim}

\end{sphinxuseclass}\end{sphinxVerbatimInput}

\end{sphinxuseclass}
\begin{sphinxuseclass}{cell}\begin{sphinxVerbatimInput}

\begin{sphinxuseclass}{cell_input}
\begin{sphinxVerbatim}[commandchars=\\\{\}]
\PYG{n}{show\PYGZus{}plot}\PYG{p}{(}\PYG{p}{)}
\end{sphinxVerbatim}

\end{sphinxuseclass}\end{sphinxVerbatimInput}
\begin{sphinxVerbatimOutput}

\begin{sphinxuseclass}{cell_output}
\noindent\sphinxincludegraphics{{4_Scattering_51_0}.png}

\end{sphinxuseclass}\end{sphinxVerbatimOutput}

\end{sphinxuseclass}\begin{itemize}
\item {} 
\sphinxAtStartPar
Para \(D < 20\) nm, \(C_\mathrm{sca}\) es despreciable en comparación con \(C_\mathrm{abs}\). El peak en \(C_\mathrm{abs}\) es el resultado de la resonancia del sistema, similar al modelo de Lorentz.

\item {} 
\sphinxAtStartPar
Para \(D > 50\) nm, \(C_\mathrm{sca}\) crece significativamente, superando \(C_\mathrm{abs}\) para \(D > 80\) nm.

\end{itemize}

\sphinxAtStartPar
Este fenómeno se repite en otros metales, aunque con distintas magnitudes y frecuencias de resonancia.

\sphinxAtStartPar
El efecto de de scattering en nanopartículas metálicas permite explicar el cambio en los colores en la copa de Lycurgus.

\noindent{\hspace*{\fill}\sphinxincludegraphics[width=300\sphinxpxdimen]{{LycurgusCup}.jpg}\hspace*{\fill}}

\sphinxAtStartPar
Esta copa del periodo romano, esta compuesta de vidrio con nanopartícula de oro y plata en forma de coloides.


\section{Referencias}
\label{\detokenize{4_Scattering/4_Scattering:referencias}}\begin{itemize}
\item {} 
\sphinxAtStartPar
Bohren C. and Huffman D. \sphinxstylestrong{Chapter 4 \sphinxhyphen{} Absorption and Scattering by a Sphere} in \sphinxstyleemphasis{Absorption and Scattering of Light by Small Particles}, 1st Ed, John Wiley \& Sons, 1983

\item {} 
\sphinxAtStartPar
Jackson. J. D., \sphinxstylestrong{Chapter 10 \sphinxhyphen{} Scattering and Diffraction} in \sphinxstyleemphasis{Classical Electrodynamics}, 3th Ed, John Wiley \& Sons, 1999

\end{itemize}

\sphinxstepscope

\sphinxAtStartPar
MEC501 \sphinxhyphen{} Manejo y Conversión de Energía Solar Térmica


\chapter{Transporte Radiativo}
\label{\detokenize{5_TransporteRadiativo/5_TransporteRadiativo:transporte-radiativo}}\label{\detokenize{5_TransporteRadiativo/5_TransporteRadiativo::doc}}
\sphinxAtStartPar

Profesor: Francisco Ramírez Cuevas
Fecha: 9 de Septiembre 2022


\section{Coherencia de la luz e interferencia}
\label{\detokenize{5_TransporteRadiativo/5_TransporteRadiativo:coherencia-de-la-luz-e-interferencia}}
\sphinxAtStartPar
Como veremos en las próximas clases, las vibraciones moleculares en la materia son las responsables de emitir radiación en forma de ondas electromagnéticas. El mecanismo es similar, pero inverso, al mecanismo de absorción de luz, el cual es originado por la interacción de ondas electromagnéticas con las vibraciones moleculares (ver clase 3).

\sphinxAtStartPar
Dicho esto, dos moléculas pueden emitir radiación con una pequeña variación en la longitud de onda \(\Delta\lambda\). Esto ocurre porque las vibraciones no están 100\% correlacionadas, es decir, existe un grado de aletoriedad en las vibraciones.

\sphinxAtStartPar
Esta aletoriedad en las vibraciones da lugar a una distribución del flujo de radiación emitido por una fuente.

\noindent{\hspace*{\fill}\sphinxincludegraphics[width=800\sphinxpxdimen]{{light_source_spectra}.png}\hspace*{\fill}}

\sphinxAtStartPar
Por ejemplo, consideremos una fuente de luz con una distribución espectral normal, centrada en \(\lambda_0\) y con una desviación estandar \(\pm\sigma_\lambda\lambda_0\), con \(\sigma_\lambda \in [0,1]\). Imaginemos esta fuente como \(N\) emisores, donde cada emisor \(j\) emite ondas electromagnéticas con longitud de onda \(\lambda \pm\Delta \lambda_j\), donde \(\Delta \lambda_j\) es escogido aleatoriamente a partir de la distribución normal.

\noindent{\hspace*{\fill}\sphinxincludegraphics[width=400\sphinxpxdimen]{{normal_distribution}.png}\hspace*{\fill}}

\sphinxAtStartPar
Asumiendo ondas en el aire en dirección \(\hat{k} = \hat{x}\), el campo eléctrico resultante es:
\begin{equation*}
\vec{E}_\mathrm{tot} = \sum_j^N E_0e^{i\left(k_jx - \omega_j t\right)} \hat{z},
\end{equation*}
\sphinxAtStartPar
donde \(k_j = \frac{2\pi}{\lambda \pm\Delta \lambda_j}\), y \(\omega_j = c_0k_j\)

\sphinxAtStartPar
Analicemos el comportamiento de \(\vec{E}_\mathrm{tot} /E_0\) para \(\lambda_0 = 500\) nm

\begin{sphinxuseclass}{cell}\begin{sphinxVerbatimInput}

\begin{sphinxuseclass}{cell_input}
\begin{sphinxVerbatim}[commandchars=\\\{\}]
\PYG{k+kn}{import} \PYG{n+nn}{numpy} \PYG{k}{as} \PYG{n+nn}{np}
\PYG{k+kn}{from} \PYG{n+nn}{numpy}\PYG{n+nn}{.}\PYG{n+nn}{random} \PYG{k+kn}{import} \PYG{n}{normal}
\PYG{k+kn}{import} \PYG{n+nn}{matplotlib}\PYG{n+nn}{.}\PYG{n+nn}{pyplot} \PYG{k}{as} \PYG{n+nn}{plt}

\PYG{k}{def} \PYG{n+nf}{light\PYGZus{}packet}\PYG{p}{(}\PYG{n}{kdir}\PYG{p}{,} \PYG{n}{x}\PYG{p}{,} \PYG{n}{t}\PYG{p}{,} \PYG{n}{lam}\PYG{p}{,} \PYG{n}{sig}\PYG{p}{,} \PYG{n}{N}\PYG{p}{)}\PYG{p}{:}
    \PYG{n}{c0} \PYG{o}{=} \PYG{l+m+mf}{3E8}          \PYG{c+c1}{\PYGZsh{} velocidad de la luz (m/s)}
    \PYG{n}{xx} \PYG{o}{=} \PYG{n}{np}\PYG{o}{.}\PYG{n}{meshgrid}\PYG{p}{(}\PYG{n}{x}\PYG{p}{,}\PYG{n}{np}\PYG{o}{.}\PYG{n}{ones}\PYG{p}{(}\PYG{n}{N}\PYG{p}{)}\PYG{p}{)}\PYG{p}{[}\PYG{l+m+mi}{0}\PYG{p}{]}
    
    \PYG{c+c1}{\PYGZsh{} Generamos arreglo de ondas aleatorias}
    \PYG{n}{dlamj} \PYG{o}{=}  \PYG{n}{normal}\PYG{p}{(}\PYG{l+m+mi}{0}\PYG{p}{,} \PYG{n}{lam}\PYG{o}{*}\PYG{n}{sig}\PYG{p}{,}\PYG{n}{N}\PYG{p}{)}
    \PYG{n}{kj} \PYG{o}{=} \PYG{p}{(}\PYG{l+m+mi}{2}\PYG{o}{*}\PYG{n}{np}\PYG{o}{.}\PYG{n}{pi}\PYG{o}{/}\PYG{p}{(}\PYG{n}{lam} \PYG{o}{+} \PYG{n}{dlamj}\PYG{p}{)}\PYG{p}{)}\PYG{o}{.}\PYG{n}{reshape}\PYG{p}{(}\PYG{o}{\PYGZhy{}}\PYG{l+m+mi}{1}\PYG{p}{,}\PYG{l+m+mi}{1}\PYG{p}{)}
    \PYG{n}{wj} \PYG{o}{=} \PYG{n}{c0}\PYG{o}{*}\PYG{n}{kj}
    \PYG{n}{Erand} \PYG{o}{=} \PYG{n}{np}\PYG{o}{.}\PYG{n}{exp}\PYG{p}{(}\PYG{l+m+mi}{1}\PYG{n}{j}\PYG{o}{*}\PYG{p}{(}\PYG{n}{kdir}\PYG{o}{*}\PYG{n}{kj}\PYG{o}{*}\PYG{n}{xx}\PYG{o}{\PYGZhy{}}\PYG{n}{wj}\PYG{o}{*}\PYG{n}{t}\PYG{p}{)}\PYG{p}{)} 
    
    \PYG{c+c1}{\PYGZsh{} Sumamos todas las ondas}
    \PYG{k}{return} \PYG{n}{np}\PYG{o}{.}\PYG{n}{sum}\PYG{p}{(}\PYG{n}{Erand}\PYG{p}{,}\PYG{n}{axis}\PYG{o}{=}\PYG{l+m+mi}{0}\PYG{p}{)}
 
\PYG{k}{def} \PYG{n+nf}{plot\PYGZus{}light\PYGZus{}packet}\PYG{p}{(}\PYG{n}{N}\PYG{p}{,} \PYG{n}{t}\PYG{p}{,} \PYG{n}{sig}\PYG{p}{)}\PYG{p}{:}
    \PYG{l+s+sd}{\PYGZsq{}\PYGZsq{}\PYGZsq{}}
\PYG{l+s+sd}{    n: número de ondas generadas}
\PYG{l+s+sd}{    t: tiempo en ns}
\PYG{l+s+sd}{    sig: \PYGZpc{} de ancho de banda (dlam = sig*lam)}
\PYG{l+s+sd}{    \PYGZsq{}\PYGZsq{}\PYGZsq{}}
    \PYG{n}{lam} \PYG{o}{=} \PYG{l+m+mf}{0.5}         \PYG{c+c1}{\PYGZsh{} longitud de onda (um)}
    \PYG{n}{t} \PYG{o}{=} \PYG{n}{t}\PYG{o}{*}\PYG{l+m+mf}{1E\PYGZhy{}9} \PYG{c+c1}{\PYGZsh{} convertimos ns a s}
    
    \PYG{c+c1}{\PYGZsh{} recorrido de la onda}
    \PYG{n}{x} \PYG{o}{=} \PYG{n}{np}\PYG{o}{.}\PYG{n}{linspace}\PYG{p}{(}\PYG{o}{\PYGZhy{}}\PYG{l+m+mi}{2}\PYG{p}{,}\PYG{l+m+mi}{2}\PYG{p}{,}\PYG{l+m+mi}{1000}\PYG{p}{)}  \PYG{c+c1}{\PYGZsh{} desde 0 a 4 micrones}
    \PYG{n}{E} \PYG{o}{=} \PYG{n}{light\PYGZus{}packet}\PYG{p}{(}\PYG{l+m+mi}{1}\PYG{p}{,} \PYG{n}{x}\PYG{p}{,} \PYG{n}{t}\PYG{p}{,} \PYG{n}{lam}\PYG{p}{,} \PYG{n}{sig}\PYG{p}{,} \PYG{n}{N}\PYG{p}{)}
    
    \PYG{c+c1}{\PYGZsh{} Graficamos}
    \PYG{n}{fig}\PYG{p}{,} \PYG{n}{ax} \PYG{o}{=} \PYG{n}{plt}\PYG{o}{.}\PYG{n}{subplots}\PYG{p}{(}\PYG{p}{)}
    \PYG{n}{fig}\PYG{o}{.}\PYG{n}{set\PYGZus{}size\PYGZus{}inches}\PYG{p}{(}\PYG{l+m+mi}{9}\PYG{p}{,} \PYG{l+m+mi}{5}\PYG{p}{)}
    \PYG{n}{plt}\PYG{o}{.}\PYG{n}{rcParams}\PYG{p}{[}\PYG{l+s+s1}{\PYGZsq{}}\PYG{l+s+s1}{font.size}\PYG{l+s+s1}{\PYGZsq{}}\PYG{p}{]} \PYG{o}{=} \PYG{l+s+s1}{\PYGZsq{}}\PYG{l+s+s1}{18}\PYG{l+s+s1}{\PYGZsq{}}
    
    \PYG{n}{ax}\PYG{o}{.}\PYG{n}{plot}\PYG{p}{(}\PYG{n}{x}\PYG{p}{,}\PYG{n}{np}\PYG{o}{.}\PYG{n}{real}\PYG{p}{(}\PYG{n}{E}\PYG{p}{)}\PYG{p}{,} \PYG{l+s+s1}{\PYGZsq{}}\PYG{l+s+s1}{k}\PYG{l+s+s1}{\PYGZsq{}}\PYG{p}{)}
    \PYG{n}{ax}\PYG{o}{.}\PYG{n}{set\PYGZus{}xlabel}\PYG{p}{(}\PYG{l+s+s1}{\PYGZsq{}}\PYG{l+s+s1}{x (\PYGZdl{}}\PYG{l+s+s1}{\PYGZbs{}}\PYG{l+s+s1}{mu\PYGZdl{}m)}\PYG{l+s+s1}{\PYGZsq{}}\PYG{p}{)}
    \PYG{n}{ax}\PYG{o}{.}\PYG{n}{set\PYGZus{}ylabel}\PYG{p}{(}\PYG{l+s+s1}{\PYGZsq{}}\PYG{l+s+s1}{Amplitud \PYGZdl{}|E|/E\PYGZus{}0\PYGZdl{}}\PYG{l+s+s1}{\PYGZsq{}}\PYG{p}{)}
    \PYG{n}{ax}\PYG{o}{.}\PYG{n}{set\PYGZus{}ylim}\PYG{p}{(}\PYG{o}{\PYGZhy{}}\PYG{n}{N}\PYG{o}{*}\PYG{l+m+mf}{1.1}\PYG{p}{,}\PYG{n}{N}\PYG{o}{*}\PYG{l+m+mf}{1.1}\PYG{p}{)}
    \PYG{n}{ax}\PYG{o}{.}\PYG{n}{grid}\PYG{p}{(}\PYG{p}{)}
\end{sphinxVerbatim}

\end{sphinxuseclass}\end{sphinxVerbatimInput}

\end{sphinxuseclass}
\begin{sphinxuseclass}{cell}\begin{sphinxVerbatimInput}

\begin{sphinxuseclass}{cell_input}
\begin{sphinxVerbatim}[commandchars=\\\{\}]
\PYG{k+kn}{from} \PYG{n+nn}{ipywidgets} \PYG{k+kn}{import} \PYG{n}{interact}

\PYG{n+nd}{@interact}\PYG{p}{(} \PYG{n}{N}\PYG{o}{=}\PYG{p}{(}\PYG{l+m+mi}{1}\PYG{p}{,}\PYG{l+m+mi}{1000}\PYG{p}{,}\PYG{l+m+mi}{1}\PYG{p}{)}\PYG{p}{,} 
           \PYG{n}{t}\PYG{o}{=}\PYG{p}{(}\PYG{o}{\PYGZhy{}}\PYG{l+m+mi}{10}\PYG{p}{,}\PYG{l+m+mi}{10}\PYG{p}{,}\PYG{l+m+mf}{0.1}\PYG{p}{)}\PYG{p}{,}
           \PYG{n}{sig}\PYG{o}{=}\PYG{p}{(}\PYG{l+m+mi}{0}\PYG{p}{,}\PYG{l+m+mi}{1}\PYG{p}{,}\PYG{l+m+mf}{0.01}\PYG{p}{)}\PYG{p}{)}
\PYG{k}{def} \PYG{n+nf}{g}\PYG{p}{(}\PYG{n}{N}\PYG{o}{=}\PYG{l+m+mi}{1000}\PYG{p}{,} \PYG{n}{t}\PYG{o}{=}\PYG{l+m+mi}{0}\PYG{p}{,} \PYG{n}{sig}\PYG{o}{=}\PYG{l+m+mf}{0.3}\PYG{p}{)}\PYG{p}{:}
    \PYG{k}{return} \PYG{n}{plot\PYGZus{}light\PYGZus{}packet}\PYG{p}{(}\PYG{n}{N}\PYG{p}{,}\PYG{n}{t}\PYG{p}{,}\PYG{n}{sig}\PYG{p}{)}
\end{sphinxVerbatim}

\end{sphinxuseclass}\end{sphinxVerbatimInput}
\begin{sphinxVerbatimOutput}

\begin{sphinxuseclass}{cell_output}
\begin{sphinxVerbatim}[commandchars=\\\{\}]
interactive(children=(IntSlider(value=1000, description=\PYGZsq{}N\PYGZsq{}, max=1000, min=1), FloatSlider(value=0.0, descript…
\end{sphinxVerbatim}

\end{sphinxuseclass}\end{sphinxVerbatimOutput}

\end{sphinxuseclass}

\subsection{Longitud de coherencia}
\label{\detokenize{5_TransporteRadiativo/5_TransporteRadiativo:longitud-de-coherencia}}
\sphinxAtStartPar
Definimos como \sphinxstylestrong{longitud de coherencia}, \(l_c\), a la distancia donde un grupo de ondas electromagnética mantiene correlación entre las fases. Para longitudes mayores a \(l_c\), decimos que la luz es incoherente, es decir, el desface entre las distintas ondas es completamente aleatorio.

\noindent{\hspace*{\fill}\sphinxincludegraphics[width=300\sphinxpxdimen]{{coherence_length}.png}\hspace*{\fill}}



\sphinxAtStartPar
La relación entre \(l_c\), la longitud de onda central \(\lambda\) y el ancho de banda \(\Delta\lambda\) está dado, aproximadamente, por la relación:
\begin{equation*}
l_c \approx \frac{\lambda^2}{n\Delta \lambda},
\end{equation*}
\sphinxAtStartPar
donde \(n\) es el indice de refracción del medio donde se propaga la luz.

\sphinxAtStartPar
Por ejemplo, para lasers He\sphinxhyphen{}Ne (laser rojo)  \(l_c\approx 0.2 - 100\) m.

\sphinxAtStartPar
Para radiación emitida por un cuerpo a temperatura \(T\), la longitud de coherencia está dada por:
\begin{equation*}
l_c T = 2167.8~\mathrm{\mu m~K}
\end{equation*}
\sphinxAtStartPar
Así, por ejemplo, la radiación solar (\(T \approx 5800~\mathrm{K}\)) tiene una longitud de coherencia, \(l_c \approx 370~\mathrm{nm}\)


\subsection{Régimen de trasporte de luz}
\label{\detokenize{5_TransporteRadiativo/5_TransporteRadiativo:regimen-de-trasporte-de-luz}}
\sphinxAtStartPar
Consideremos dos paquetes de onda con una longitud de coherencia \(l_c\), viajando en sentido opuesto.

\sphinxAtStartPar
Podemos ver que ambos paquetes de luz interfieren en \(x = 0\) en un instante \(t\). Al continuar su camino, ambos paquetes de onda recuperan su forma original.

\begin{sphinxuseclass}{cell}\begin{sphinxVerbatimInput}

\begin{sphinxuseclass}{cell_input}
\begin{sphinxVerbatim}[commandchars=\\\{\}]
\PYG{k}{def} \PYG{n+nf}{plot\PYGZus{}2light\PYGZus{}packet}\PYG{p}{(}\PYG{n}{n}\PYG{p}{,} \PYG{n}{t}\PYG{p}{,} \PYG{n}{sig}\PYG{p}{)}\PYG{p}{:}
    \PYG{l+s+sd}{\PYGZsq{}\PYGZsq{}\PYGZsq{}}
\PYG{l+s+sd}{    n: número de ondas generadas}
\PYG{l+s+sd}{    t: tiempo en ns}
\PYG{l+s+sd}{    sig: \PYGZpc{} de ancho de banda (dlam = sig*lam)}
\PYG{l+s+sd}{    \PYGZsq{}\PYGZsq{}\PYGZsq{}}
    \PYG{n}{t} \PYG{o}{=} \PYG{n}{t}\PYG{o}{*}\PYG{l+m+mf}{1E\PYGZhy{}9} \PYG{c+c1}{\PYGZsh{} convertimos ns a s}
    \PYG{n}{lam} \PYG{o}{=} \PYG{l+m+mf}{0.5}
    
    \PYG{c+c1}{\PYGZsh{} recorrido de la onda}
    \PYG{n}{x} \PYG{o}{=} \PYG{n}{np}\PYG{o}{.}\PYG{n}{linspace}\PYG{p}{(}\PYG{o}{\PYGZhy{}}\PYG{l+m+mi}{2}\PYG{p}{,}\PYG{l+m+mi}{2}\PYG{p}{,}\PYG{l+m+mi}{1000}\PYG{p}{)}  \PYG{c+c1}{\PYGZsh{} desde 0 a 4 micrones}
    \PYG{n}{k0} \PYG{o}{=} \PYG{l+m+mi}{2}\PYG{o}{*}\PYG{n}{np}\PYG{o}{.}\PYG{n}{pi}\PYG{o}{/}\PYG{n}{lam}
    \PYG{n}{Efw} \PYG{o}{=} \PYG{n}{light\PYGZus{}packet}\PYG{p}{(} \PYG{l+m+mi}{1}\PYG{p}{,} \PYG{n}{x} \PYG{o}{\PYGZhy{}} \PYG{n}{x}\PYG{p}{[} \PYG{l+m+mi}{0}\PYG{p}{]}\PYG{p}{,} \PYG{n}{t}\PYG{p}{,} \PYG{n}{lam}\PYG{p}{,} \PYG{n}{sig}\PYG{p}{,} \PYG{n}{n}\PYG{p}{)}
    \PYG{n}{Ebw} \PYG{o}{=} \PYG{n}{light\PYGZus{}packet}\PYG{p}{(}\PYG{o}{\PYGZhy{}}\PYG{l+m+mi}{1}\PYG{p}{,} \PYG{n}{x} \PYG{o}{\PYGZhy{}} \PYG{n}{x}\PYG{p}{[}\PYG{o}{\PYGZhy{}}\PYG{l+m+mi}{1}\PYG{p}{]}\PYG{p}{,} \PYG{n}{t}\PYG{p}{,} \PYG{n}{lam}\PYG{p}{,} \PYG{n}{sig}\PYG{p}{,} \PYG{n}{n}\PYG{p}{)}
    
    \PYG{c+c1}{\PYGZsh{} Graficamos}
    \PYG{n}{fig}\PYG{p}{,} \PYG{n}{ax} \PYG{o}{=} \PYG{n}{plt}\PYG{o}{.}\PYG{n}{subplots}\PYG{p}{(}\PYG{p}{)}
    \PYG{n}{fig}\PYG{o}{.}\PYG{n}{set\PYGZus{}size\PYGZus{}inches}\PYG{p}{(}\PYG{l+m+mi}{9}\PYG{p}{,} \PYG{l+m+mi}{5}\PYG{p}{)}
    \PYG{n}{plt}\PYG{o}{.}\PYG{n}{rcParams}\PYG{p}{[}\PYG{l+s+s1}{\PYGZsq{}}\PYG{l+s+s1}{font.size}\PYG{l+s+s1}{\PYGZsq{}}\PYG{p}{]} \PYG{o}{=} \PYG{l+s+s1}{\PYGZsq{}}\PYG{l+s+s1}{18}\PYG{l+s+s1}{\PYGZsq{}}
    
    \PYG{n}{ax}\PYG{o}{.}\PYG{n}{plot}\PYG{p}{(}\PYG{n}{x}\PYG{p}{,}\PYG{n}{np}\PYG{o}{.}\PYG{n}{real}\PYG{p}{(}\PYG{n}{Efw} \PYG{o}{+} \PYG{n}{Ebw}\PYG{p}{)}\PYG{p}{,} \PYG{l+s+s1}{\PYGZsq{}}\PYG{l+s+s1}{k}\PYG{l+s+s1}{\PYGZsq{}}\PYG{p}{)}
    \PYG{n}{ax}\PYG{o}{.}\PYG{n}{set\PYGZus{}xlabel}\PYG{p}{(}\PYG{l+s+s1}{\PYGZsq{}}\PYG{l+s+s1}{x (\PYGZdl{}}\PYG{l+s+s1}{\PYGZbs{}}\PYG{l+s+s1}{mu\PYGZdl{}m)}\PYG{l+s+s1}{\PYGZsq{}}\PYG{p}{)}
    \PYG{n}{ax}\PYG{o}{.}\PYG{n}{set\PYGZus{}ylabel}\PYG{p}{(}\PYG{l+s+s1}{\PYGZsq{}}\PYG{l+s+s1}{Amplitud \PYGZdl{}|E|/E\PYGZus{}0\PYGZdl{}}\PYG{l+s+s1}{\PYGZsq{}}\PYG{p}{)}
    \PYG{n}{ax}\PYG{o}{.}\PYG{n}{set\PYGZus{}ylim}\PYG{p}{(}\PYG{o}{\PYGZhy{}}\PYG{n}{n}\PYG{o}{*}\PYG{l+m+mf}{2.1}\PYG{p}{,}\PYG{n}{n}\PYG{o}{*}\PYG{l+m+mf}{2.1}\PYG{p}{)}
    \PYG{n}{ax}\PYG{o}{.}\PYG{n}{grid}\PYG{p}{(}\PYG{p}{)}
\end{sphinxVerbatim}

\end{sphinxuseclass}\end{sphinxVerbatimInput}

\end{sphinxuseclass}
\begin{sphinxuseclass}{cell}\begin{sphinxVerbatimInput}

\begin{sphinxuseclass}{cell_input}
\begin{sphinxVerbatim}[commandchars=\\\{\}]
\PYG{k+kn}{from} \PYG{n+nn}{ipywidgets} \PYG{k+kn}{import} \PYG{n}{interact}

\PYG{n+nd}{@interact}\PYG{p}{(} \PYG{n}{N}\PYG{o}{=}\PYG{p}{(}\PYG{l+m+mi}{1}\PYG{p}{,}\PYG{l+m+mi}{1000}\PYG{p}{,}\PYG{l+m+mi}{1}\PYG{p}{)}\PYG{p}{,} 
           \PYG{n}{t}\PYG{o}{=}\PYG{p}{(}\PYG{l+m+mi}{0}\PYG{p}{,}\PYG{l+m+mi}{20}\PYG{p}{,}\PYG{l+m+mf}{0.1}\PYG{p}{)}\PYG{p}{,}
           \PYG{n}{sig}\PYG{o}{=}\PYG{p}{(}\PYG{l+m+mi}{0}\PYG{p}{,}\PYG{l+m+mi}{1}\PYG{p}{,}\PYG{l+m+mf}{0.01}\PYG{p}{)}\PYG{p}{)}
\PYG{k}{def} \PYG{n+nf}{g}\PYG{p}{(}\PYG{n}{N}\PYG{o}{=}\PYG{l+m+mi}{1000}\PYG{p}{,} \PYG{n}{t}\PYG{o}{=}\PYG{l+m+mi}{2}\PYG{p}{,} \PYG{n}{sig}\PYG{o}{=}\PYG{l+m+mf}{0.3}\PYG{p}{)}\PYG{p}{:}
    \PYG{k}{return} \PYG{n}{plot\PYGZus{}2light\PYGZus{}packet}\PYG{p}{(}\PYG{n}{N}\PYG{p}{,}\PYG{n}{t}\PYG{p}{,}\PYG{n}{sig}\PYG{p}{)}
\end{sphinxVerbatim}

\end{sphinxuseclass}\end{sphinxVerbatimInput}
\begin{sphinxVerbatimOutput}

\begin{sphinxuseclass}{cell_output}
\begin{sphinxVerbatim}[commandchars=\\\{\}]
interactive(children=(IntSlider(value=1000, description=\PYGZsq{}N\PYGZsq{}, max=1000, min=1), FloatSlider(value=2.0, descript…
\end{sphinxVerbatim}

\end{sphinxuseclass}\end{sphinxVerbatimOutput}

\end{sphinxuseclass}
\sphinxAtStartPar
A partir de esto, podemos concluir que los fenómenos de interferencia en películas de capa delgada de espesor \(d\) no serían visibles si \(d > l_c\). En otras palabras, el fenómeno de interferencia solo existe si el paquete de onda interfiere consigo mismo.

\noindent{\hspace*{\fill}\sphinxincludegraphics[width=200\sphinxpxdimen]{{interference_thinfilm}.png}\hspace*{\fill}}



\sphinxAtStartPar
En general, para una longitud características \(d\):
\begin{itemize}
\item {} 
\sphinxAtStartPar
Si \(d > l_c\) el \sphinxstylestrong{transporte de luz es incoherente} . En este régimen, podemos ignorar las propiedades oscilatorias de la luz, y analizar el problema como el transporte de pequeños paquetes de onda, o simplemente como partículas.

\item {} 
\sphinxAtStartPar
Si \(d < l_c\), el \sphinxstylestrong{transporte de luz es coherente}. En este régimen debemos considerar las propiedades oscilatorias a partir de las Ecuaciones de Maxwell.

\end{itemize}

\sphinxAtStartPar
Así, los coeficientes de Fresnel para una película delgada solo son válidos para \(d < l_c\).

\sphinxAtStartPar
Los coeficientes de Fresnel para una interface, en cambio, siempre son válidos.

\noindent{\hspace*{\fill}\sphinxincludegraphics[width=350\sphinxpxdimen]{{interference_interface}.png}\hspace*{\fill}}



\sphinxAtStartPar
Igualmente, en medios particulados, los fenómenos de interferencia pueden ocurrir si las partículas están suficientemente cerca y tienen tamaños similares. Llamamos a esto \sphinxstylestrong{scattering coherente}

\noindent{\hspace*{\fill}\sphinxincludegraphics[width=600\sphinxpxdimen]{{scattering_coherence}.png}\hspace*{\fill}}



\sphinxAtStartPar
El fenómeno de scattering coherente da lugar a los denominados \sphinxstylestrong{colores estructurales} presentes en las alas de las aves y mariposas (mas info \sphinxhref{https://wires.onlinelibrary.wiley.com/doi/10.1002/wnan.1396}{acá})

\noindent{\hspace*{\fill}\sphinxincludegraphics[width=550\sphinxpxdimen]{{structural_colors_birds}.png}\hspace*{\fill}}



\sphinxAtStartPar
Fuente: \sphinxhref{https://www.pnas.org/doi/10.1073/pnas.2015551118}{Hwang, V. et al. PNAS 118 (4) e2015551118
(2020)}

\noindent{\hspace*{\fill}\sphinxincludegraphics[width=450\sphinxpxdimen]{{structural_colors_buterflies}.png}\hspace*{\fill}}



\sphinxAtStartPar
Fuente: \sphinxhref{https://iopscience.iop.org/article/10.1088/2040-8978/18/6/065105}{Tippets C. A. et al. J. Opt. 18 (2016) 065105}


\section{Teoría de transferencia radiativa}
\label{\detokenize{5_TransporteRadiativo/5_TransporteRadiativo:teoria-de-transferencia-radiativa}}
\sphinxAtStartPar
Si el transporte de luz es incoherente, podemos ignorar las propiedades oscilatorias de la luz y analizar el fenómeno óptico como un el transporte de radiación a través de un volumen de control.


\subsection{Intensidad específica}
\label{\detokenize{5_TransporteRadiativo/5_TransporteRadiativo:intensidad-especifica}}
\sphinxAtStartPar
Definimos como \sphinxstylestrong{radiancia espectral o intensidad específica}, \(I_\lambda\), al flujo de energía por \sphinxstylestrong{ángulo sólido diferencial}, \(d\Omega\), para una longitud de onda \(\lambda\).

\noindent{\hspace*{\fill}\sphinxincludegraphics[width=300\sphinxpxdimen]{{specific_intensity}.png}\hspace*{\fill}}

\sphinxAtStartPar
El ángulo sólido define el tamaño relativo del área \(S\) para un observador en \(P\) a una distancia \(r\).

\sphinxAtStartPar
El diferencial está definido por \(d\Omega = \sin\theta d\theta d\phi\)

\noindent{\hspace*{\fill}\sphinxincludegraphics[width=450\sphinxpxdimen]{{solid_angle}.png}\hspace*{\fill}}

\sphinxAtStartPar
Es importante aclarar que los conceptos intensidad específica y vector de Poynting \(\langle\vec{S}\rangle\), son equivalentes: \(I_\lambda(\hat{k})\) es un término utilizado en radiometría para definir el flujo de energía por unidad de ángulo sólido, mientras que \(\langle\vec{S}\rangle\) es un término ulizado en óptica para describir el flujo de energía en dirección \(\hat{k}\). Cabe destacar, sin embargo, que \(I_\lambda(\hat{k})\) es una magnitud, y \(\langle\vec{S}\rangle\) es un vector. Así, podríamos decir que la relación entre estos dos términos está dada por \(I_\lambda(\hat{k}) = \langle\vec{S}\rangle\cdot\hat{k}\).

\sphinxAtStartPar
El ángulo sólido total para una esfera es:
\begin{equation*}
\int_\mathrm{esfera} d\Omega = \int_0^{2\pi}\int_0^{\pi} \sin\theta d\theta d\phi= 4\pi
\end{equation*}

\subsection{Ecuación de transferencia radiativa}
\label{\detokenize{5_TransporteRadiativo/5_TransporteRadiativo:ecuacion-de-transferencia-radiativa}}
\sphinxAtStartPar
La \sphinxstylestrong{ecuación de transferencia radiativa} (RTE por sus siglas en ingles), es una ecuación de transporte que describe la propagación de la radiancia espectral, \(I_\lambda(\vec{r},\hat{k})\), en función de la posición posición \(\vec{r}\) y dirección \(\hat{k}\). En su forma más general, para un problema estacionario:
\label{equation:5_TransporteRadiativo/5_TransporteRadiativo:0b0fe890-cfae-4982-928f-aade87096709}\begin{equation}
\hat{k}\cdot\nabla_r I_\lambda(\vec{r},\hat{k}) = - \left[\frac{f_v}{V_p}C_\mathrm{ext} + 2\kappa k_0\right]I_\lambda(\vec{r},\hat{k}) + \frac{f_v}{V_p}\int_{4\pi} P_\mathrm{sca}(\hat{k},\hat{k}') I_\lambda(\vec{r},\hat{k}') d\Omega'
\end{equation}
\sphinxAtStartPar
donde \(f_v\) y \(V_p\) son, respectivamente la fracción de volúmen y el volúmen de las partículas en el medio, y \(P_\mathrm{sca}(\hat{k},\hat{k}')\) es la función de fase.

\sphinxAtStartPar
Notar que consideramos el caso más generalizado de la función de fase, que depende tanto de la dirección de la radiación incidente \(\hat{k}\) como de la dirección del scattering \(\hat{k}'\). Ambas definidas por ángulo sólido

\sphinxAtStartPar
En el caso de una esfera, \(\hat{k}_\mathrm{sca} = \hat{k}\) y \(\hat{k}_\mathrm{inc} = \hat{k}'\). Así, la función de fase depende de \(\theta\), el cual está definido por \(\cos\theta = \hat{k}\cdot\hat{k}'\)

\sphinxAtStartPar
Con esto en mente, ahora podemos discutir el significado de los términos en la RTE:
\begin{itemize}
\item {} 
\sphinxAtStartPar
El primer término representa el cambio de \(I_\lambda(\vec{r},\hat{k})\) a través volumen diferencial. Por ejemplo, en el caso unidimencional en dirección \(\hat{k} = \hat{x}\), tenemos:

\end{itemize}
\begin{equation*}
\hat{x}\cdot\nabla_r I_\lambda(\vec{r},\hat{k}) = \frac{\partial}{\partial x}I_\lambda(\vec{r},\hat{k})
\end{equation*}\begin{itemize}
\item {} 
\sphinxAtStartPar
El segundo término representa la pérdida de energía radiativa, por extinción \(\left(\frac{f_v}{V_p}C_\mathrm{ext}\right)\) y absorción en el material material (\(2\kappa k_0\)), respectivamente. Recordemos que la extinción representa la energía absorbida por las partículas + la energía de scattering en direcciónes distintas a \(\hat{k}\), es decir \(C_\mathrm{ext}= C_\mathrm{abs} + C_\mathrm{sca}\).

\end{itemize}
\begin{itemize}
\item {} 
\sphinxAtStartPar
El tercer término representa la ganancia de energía radiativa produco del scattering inducido por radiación incidente en dirección \(\hat{k}'\). Este término representa el fenómeno de \sphinxstylestrong{scattering múltiple}.

\end{itemize}


\section{Soluciones de la RTE}
\label{\detokenize{5_TransporteRadiativo/5_TransporteRadiativo:soluciones-de-la-rte}}
\sphinxAtStartPar
La ecuación de transferencia radiativa permite explicar de forma fenomenológica el transporte de radiación en un medio particulado. Debido a su complejidad, existen pocas soluciones analíticas. En esta sección describiremos las tres más conocidas.


\subsection{Película de material sin partículas (Reflectividad y Transmisividad incoherente)}
\label{\detokenize{5_TransporteRadiativo/5_TransporteRadiativo:pelicula-de-material-sin-particulas-reflectividad-y-transmisividad-incoherente}}
\sphinxAtStartPar
En el caso de un medio sin partículas solo debemos considerar el primer término de la RTE.

\sphinxAtStartPar
Para el caso particular de un material de espesor \(t_\mathrm{film}\) e índice de refracción \(N = n + \kappa\), rodeado por un medio con índice de refracción \(N_0\), derivamos las siguientes relaciones de reflectividad y tranmisividad para luz incoherente:
\label{equation:5_TransporteRadiativo/5_TransporteRadiativo:367f64b8-037b-440e-a144-518b71440410}\begin{align}
R_\mathrm{incoh}&= R_0 + \frac{T_0^2R_0e^{-4\kappa k_0 t_\mathrm{film}}}{1 - R_0^2e^{-4\kappa k_0 t_\mathrm{film}}} 
\\[10pt]
T_\mathrm{incoh} &= \frac{T_0^2e^{-4\kappa k_0 t_\mathrm{film}}}{1 - R_0^2e^{-4\kappa k_0 t_\mathrm{film}}}
\end{align}
\sphinxAtStartPar
donde \(R_0\) y \(T_0\) corresponden, respectivamente, a la reflectividad y transmisividad en la interface \(N_0 / N_1\)

\sphinxAtStartPar
En el siguiente ejemplo, utilizamos la función \sphinxcode{\sphinxupquote{incoh\_multilayer}} de la libreria \sphinxcode{\sphinxupquote{empylib.waveoptics}}. Esta función es más general que la ecuación anterior y permite determinar \(R_\mathrm{incoh}\) y \(T_\mathrm{incoh}\) para arreglos multicapas.

\sphinxAtStartPar
En este caso, evaluaremos una película de sílice de espesor \(1~\mu\mathrm{m}\), sobre un sustrato con íncide de refracción \(N_\mathrm{back} = 4.3\), y con aire en la superficie \(N_\mathrm{front} = 1.0\). El espectro de longitudes de onda \(\lambda\in[0.3,0.8]~\mu\mathrm{m}\) y el ángulo de incidencia \(\theta_i = 30°\). Para comparar, determinaremos \(R\) y \(T\) para el caso de luz coherente.

\begin{sphinxuseclass}{cell}\begin{sphinxVerbatimInput}

\begin{sphinxuseclass}{cell_input}
\begin{sphinxVerbatim}[commandchars=\\\{\}]
\PYG{k+kn}{import} \PYG{n+nn}{numpy} \PYG{k}{as} \PYG{n+nn}{np}
\PYG{k+kn}{import} \PYG{n+nn}{empylib}\PYG{n+nn}{.}\PYG{n+nn}{waveoptics} \PYG{k}{as} \PYG{n+nn}{wv}
\PYG{k+kn}{import} \PYG{n+nn}{empylib}\PYG{n+nn}{.}\PYG{n+nn}{nklib} \PYG{k}{as} \PYG{n+nn}{nk}

\PYG{n}{lam} \PYG{o}{=} \PYG{n}{np}\PYG{o}{.}\PYG{n}{linspace}\PYG{p}{(}\PYG{l+m+mi}{2}\PYG{p}{,}\PYG{l+m+mi}{10}\PYG{p}{,}\PYG{l+m+mi}{100}\PYG{p}{)}  \PYG{c+c1}{\PYGZsh{} espectro de longitudes de onda (um)}
\PYG{n}{theta} \PYG{o}{=} \PYG{n}{np}\PYG{o}{.}\PYG{n}{radians}\PYG{p}{(}\PYG{l+m+mi}{30}\PYG{p}{)}          \PYG{c+c1}{\PYGZsh{} ángulo de incidencia}

\PYG{n}{Nfront} \PYG{o}{=} \PYG{l+m+mf}{1.0}                 \PYG{c+c1}{\PYGZsh{} índice de refracción medio superior}
\PYG{n}{N1}     \PYG{o}{=} \PYG{n}{nk}\PYG{o}{.}\PYG{n}{SiO2}\PYG{p}{(}\PYG{n}{lam}\PYG{p}{)}        \PYG{c+c1}{\PYGZsh{} índice de refracción capa delgada}
\PYG{n}{Nback}  \PYG{o}{=} \PYG{l+m+mf}{4.3}                 \PYG{c+c1}{\PYGZsh{} índice de refracción medio inferior}
\PYG{n}{N} \PYG{o}{=} \PYG{p}{(}\PYG{n}{Nfront}\PYG{p}{,} \PYG{n}{N1}\PYG{p}{,} \PYG{n}{Nback}\PYG{p}{)}      \PYG{c+c1}{\PYGZsh{} indices de refracción (above, mid, below)}
\PYG{n}{d} \PYG{o}{=} \PYG{l+m+mf}{1.0}                      \PYG{c+c1}{\PYGZsh{} espesor capa intermedia (um)}

\PYG{c+c1}{\PYGZsh{} caso luz incoherente}
\PYG{n}{Rp\PYGZus{}incoh}\PYG{p}{,} \PYG{n}{Tp\PYGZus{}incoh} \PYG{o}{=} \PYG{n}{wv}\PYG{o}{.}\PYG{n}{incoh\PYGZus{}multilayer}\PYG{p}{(}\PYG{n}{lam}\PYG{p}{,}\PYG{n}{theta}\PYG{p}{,} \PYG{n}{N}\PYG{p}{,} \PYG{n}{d}\PYG{p}{,} \PYG{n}{pol}\PYG{o}{=}\PYG{l+s+s1}{\PYGZsq{}}\PYG{l+s+s1}{TM}\PYG{l+s+s1}{\PYGZsq{}}\PYG{p}{)}
\PYG{c+c1}{\PYGZsh{} caso luz coherente}
\PYG{n}{Rp}\PYG{p}{,} \PYG{n}{Tp} \PYG{o}{=} \PYG{n}{wv}\PYG{o}{.}\PYG{n}{multilayer}\PYG{p}{(}\PYG{n}{lam}\PYG{p}{,}\PYG{n}{theta}\PYG{p}{,} \PYG{n}{N}\PYG{p}{,} \PYG{n}{d}\PYG{p}{,} \PYG{n}{pol}\PYG{o}{=}\PYG{l+s+s1}{\PYGZsq{}}\PYG{l+s+s1}{TM}\PYG{l+s+s1}{\PYGZsq{}}\PYG{p}{)}\PYG{p}{[}\PYG{p}{:}\PYG{l+m+mi}{2}\PYG{p}{]}
\end{sphinxVerbatim}

\end{sphinxuseclass}\end{sphinxVerbatimInput}

\end{sphinxuseclass}
\begin{sphinxuseclass}{cell}\begin{sphinxVerbatimInput}

\begin{sphinxuseclass}{cell_input}
\begin{sphinxVerbatim}[commandchars=\\\{\}]
\PYG{n}{fig}\PYG{p}{,} \PYG{n}{ax} \PYG{o}{=} \PYG{n}{plt}\PYG{o}{.}\PYG{n}{subplots}\PYG{p}{(}\PYG{p}{)}
\PYG{n}{fig}\PYG{o}{.}\PYG{n}{set\PYGZus{}size\PYGZus{}inches}\PYG{p}{(}\PYG{l+m+mi}{7}\PYG{p}{,} \PYG{l+m+mi}{5}\PYG{p}{)}
\PYG{n}{plt}\PYG{o}{.}\PYG{n}{rcParams}\PYG{p}{[}\PYG{l+s+s1}{\PYGZsq{}}\PYG{l+s+s1}{font.size}\PYG{l+s+s1}{\PYGZsq{}}\PYG{p}{]} \PYG{o}{=} \PYG{l+s+s1}{\PYGZsq{}}\PYG{l+s+s1}{16}\PYG{l+s+s1}{\PYGZsq{}}

\PYG{c+c1}{\PYGZsh{} Graficamos el flujo de energía}
\PYG{n}{plt}\PYG{o}{.}\PYG{n}{plot}\PYG{p}{(}\PYG{n}{lam}\PYG{p}{,}\PYG{n}{Rp}\PYG{p}{,}\PYG{l+s+s1}{\PYGZsq{}}\PYG{l+s+s1}{\PYGZhy{}\PYGZhy{}r}\PYG{l+s+s1}{\PYGZsq{}}\PYG{p}{,}\PYG{n}{label}\PYG{o}{=}\PYG{l+s+s1}{\PYGZsq{}}\PYG{l+s+s1}{\PYGZdl{}R\PYGZus{}}\PYG{l+s+s1}{\PYGZbs{}}\PYG{l+s+s1}{mathrm}\PYG{l+s+si}{\PYGZob{}TM\PYGZcb{}}\PYG{l+s+s1}{\PYGZdl{} (coh)}\PYG{l+s+s1}{\PYGZsq{}}\PYG{p}{)}
\PYG{n}{plt}\PYG{o}{.}\PYG{n}{plot}\PYG{p}{(}\PYG{n}{lam}\PYG{p}{,}\PYG{n}{Tp}\PYG{p}{,}\PYG{l+s+s1}{\PYGZsq{}}\PYG{l+s+s1}{\PYGZhy{}\PYGZhy{}b}\PYG{l+s+s1}{\PYGZsq{}}\PYG{p}{,}\PYG{n}{label}\PYG{o}{=}\PYG{l+s+s1}{\PYGZsq{}}\PYG{l+s+s1}{\PYGZdl{}T\PYGZus{}}\PYG{l+s+s1}{\PYGZbs{}}\PYG{l+s+s1}{mathrm}\PYG{l+s+si}{\PYGZob{}TM\PYGZcb{}}\PYG{l+s+s1}{\PYGZdl{} (coh)}\PYG{l+s+s1}{\PYGZsq{}}\PYG{p}{)}
\PYG{n}{plt}\PYG{o}{.}\PYG{n}{plot}\PYG{p}{(}\PYG{n}{lam}\PYG{p}{,}\PYG{n}{Rp\PYGZus{}incoh}\PYG{p}{,}\PYG{l+s+s1}{\PYGZsq{}}\PYG{l+s+s1}{\PYGZhy{}r}\PYG{l+s+s1}{\PYGZsq{}}\PYG{p}{,}\PYG{n}{label}\PYG{o}{=}\PYG{l+s+s1}{\PYGZsq{}}\PYG{l+s+s1}{\PYGZdl{}R\PYGZus{}}\PYG{l+s+s1}{\PYGZbs{}}\PYG{l+s+s1}{mathrm}\PYG{l+s+si}{\PYGZob{}TM\PYGZcb{}}\PYG{l+s+s1}{\PYGZdl{} (incoh)}\PYG{l+s+s1}{\PYGZsq{}}\PYG{p}{)}
\PYG{n}{plt}\PYG{o}{.}\PYG{n}{plot}\PYG{p}{(}\PYG{n}{lam}\PYG{p}{,}\PYG{n}{Tp\PYGZus{}incoh}\PYG{p}{,}\PYG{l+s+s1}{\PYGZsq{}}\PYG{l+s+s1}{\PYGZhy{}b}\PYG{l+s+s1}{\PYGZsq{}}\PYG{p}{,}\PYG{n}{label}\PYG{o}{=}\PYG{l+s+s1}{\PYGZsq{}}\PYG{l+s+s1}{\PYGZdl{}T\PYGZus{}}\PYG{l+s+s1}{\PYGZbs{}}\PYG{l+s+s1}{mathrm}\PYG{l+s+si}{\PYGZob{}TM\PYGZcb{}}\PYG{l+s+s1}{\PYGZdl{} (incoh)}\PYG{l+s+s1}{\PYGZsq{}}\PYG{p}{)}
\PYG{n}{plt}\PYG{o}{.}\PYG{n}{title}\PYG{p}{(}\PYG{l+s+s1}{\PYGZsq{}}\PYG{l+s+s1}{arreglo 1.0/1.5/4.3}\PYG{l+s+s1}{\PYGZsq{}}\PYG{p}{)}
\PYG{n}{plt}\PYG{o}{.}\PYG{n}{xlabel}\PYG{p}{(}\PYG{l+s+s1}{\PYGZsq{}}\PYG{l+s+s1}{Longitud de onda (\PYGZdl{}}\PYG{l+s+s1}{\PYGZbs{}}\PYG{l+s+s1}{mu\PYGZdl{}m)}\PYG{l+s+s1}{\PYGZsq{}}\PYG{p}{)}
\PYG{n}{plt}\PYG{o}{.}\PYG{n}{ylabel}\PYG{p}{(}\PYG{l+s+s1}{\PYGZsq{}}\PYG{l+s+s1}{Refletividad / Transmisividad}\PYG{l+s+s1}{\PYGZsq{}}\PYG{p}{)}
\PYG{n}{plt}\PYG{o}{.}\PYG{n}{legend}\PYG{p}{(}\PYG{p}{)}
\PYG{n}{plt}\PYG{o}{.}\PYG{n}{show}\PYG{p}{(}\PYG{p}{)}
\end{sphinxVerbatim}

\end{sphinxuseclass}\end{sphinxVerbatimInput}
\begin{sphinxVerbatimOutput}

\begin{sphinxuseclass}{cell_output}
\noindent\sphinxincludegraphics{{5_TransporteRadiativo_49_0}.png}

\end{sphinxuseclass}\end{sphinxVerbatimOutput}

\end{sphinxuseclass}
\sphinxAtStartPar
En la figura, para el caso de luz coherente, notamos oscilaciones en \(R\) y \(T\) producto de los fenómenos de interferencia. En el caso de luz incoherente, el fenómeno de interferencia desaparace.


\subsection{Película de material con baja concentración de partículas (Ley de Beer\sphinxhyphen{}Lambert)}
\label{\detokenize{5_TransporteRadiativo/5_TransporteRadiativo:pelicula-de-material-con-baja-concentracion-de-particulas-ley-de-beer-lambert}}
\sphinxAtStartPar
Definimos el \sphinxstylestrong{camino libre medio de scattering}, como \(\Lambda_\mathrm{sca} = \left(f_v C_\mathrm{sca}/V_p\right)^{-1}\). Este parametro representa la distancia promedio que recorre la luz entre eventos de scattering.

\sphinxAtStartPar
Si \(\Lambda_\mathrm{sca}\) es mayor que el espesor del material, \(t_\mathrm{film}\), la probabilidad de que ocurran más de un evento scattering es despreciable, y decimos que el \sphinxstylestrong{scattering es de primer orden}. En este caso, el tercer término de la RTE es despreciable y la radiación solo pierde energía por absorción del material o extinción inducida por las partículas.

\sphinxAtStartPar
La solución a esta ecuación se conoce como la ley de Beer\sphinxhyphen{}Lambert. Mediante esta aproximación podemos deducir las componentes total y especular de la transmitancia:
\label{equation:5_TransporteRadiativo/5_TransporteRadiativo:8023a28d-9cc5-4e13-a69b-6171fe305c03}\begin{equation}
T_\mathrm{tot} = T_0 e^{-f_v\frac{C_\mathrm{abs}}{V_p}t_\mathrm{film}}\quad\quad\mathrm{and}\quad\quad 
T_\mathrm{spec} = T_0 e^{-f_v\frac{C_\mathrm{ext}}{V_p}t_\mathrm{film}}
\end{equation}
\sphinxAtStartPar
donde, \(T_0\) es la \sphinxstylestrong{transmisividad incoherente del material sin incrustaciones}.

\noindent{\hspace*{\fill}\sphinxincludegraphics[width=500\sphinxpxdimen]{{beer_lambert}.png}\hspace*{\fill}}



\sphinxAtStartPar
Podemos utilizar la ley de Beer\sphinxhyphen{}Lambert para analizar, de forma aproximada, el efecto del color del cielo durante el día y en la tarde.

\sphinxAtStartPar
El color del cielo está dado por la componente difusa. Así calculamos \(T_\mathrm{dif} = T_\mathrm{tot} - T_\mathrm{spec}\).

\sphinxAtStartPar
Consideremos una atmosfera compuesta de aire (\(N_h = 1.0\)) y una pequeña concentración (\(f_v = 1\times 10^{-6}~\%\)) de partículas de 10 nm de diámetro e índice de refracción \(N_p = 1.5\). El espesor de la atmosfera es \(t_\mathrm{atm} = 100~\mathrm{km}\)

\begin{sphinxuseclass}{cell}\begin{sphinxVerbatimInput}

\begin{sphinxuseclass}{cell_input}
\begin{sphinxVerbatim}[commandchars=\\\{\}]
\PYG{k+kn}{import} \PYG{n+nn}{empylib}\PYG{n+nn}{.}\PYG{n+nn}{rad\PYGZus{}transfer} \PYG{k}{as} \PYG{n+nn}{rt}
\PYG{k+kn}{import} \PYG{n+nn}{empylib}\PYG{n+nn}{.}\PYG{n+nn}{nklib} \PYG{k}{as} \PYG{n+nn}{nk}
\PYG{k+kn}{import} \PYG{n+nn}{numpy} \PYG{k}{as} \PYG{n+nn}{np}
\PYG{k+kn}{import} \PYG{n+nn}{matplotlib}\PYG{n+nn}{.}\PYG{n+nn}{pyplot} \PYG{k}{as} \PYG{n+nn}{plt}
\PYG{k+kn}{from} \PYG{n+nn}{empylib}\PYG{n+nn}{.}\PYG{n+nn}{ref\PYGZus{}spectra} \PYG{k+kn}{import} \PYG{n}{AM15}
\PYG{k+kn}{from} \PYG{n+nn}{empylib}\PYG{n+nn}{.}\PYG{n+nn}{ref\PYGZus{}spectra} \PYG{k+kn}{import} \PYG{n}{color\PYGZus{}system} \PYG{k}{as} \PYG{n}{cs}
\PYG{n}{cs} \PYG{o}{=} \PYG{n}{cs}\PYG{o}{.}\PYG{n}{hdtv}

\PYG{n}{lam1} \PYG{o}{=} \PYG{n}{np}\PYG{o}{.}\PYG{n}{linspace}\PYG{p}{(}\PYG{l+m+mf}{0.38}\PYG{p}{,}\PYG{l+m+mf}{0.78}\PYG{p}{,}\PYG{l+m+mi}{100}\PYG{p}{)} \PYG{c+c1}{\PYGZsh{} espectro de longitudes de onda}
\PYG{n}{cs}\PYG{o}{.}\PYG{n}{interp\PYGZus{}internals}\PYG{p}{(}\PYG{n}{lam1}\PYG{p}{)}
\PYG{k}{def} \PYG{n+nf}{plot\PYGZus{}atmosphere}\PYG{p}{(}\PYG{n}{theta\PYGZus{}sun}\PYG{p}{)}\PYG{p}{:}
    \PYG{n}{fig}\PYG{p}{,} \PYG{n}{ax} \PYG{o}{=} \PYG{n}{plt}\PYG{o}{.}\PYG{n}{subplots}\PYG{p}{(}\PYG{p}{)}          
    \PYG{n}{fig}\PYG{o}{.}\PYG{n}{set\PYGZus{}size\PYGZus{}inches}\PYG{p}{(}\PYG{l+m+mi}{8}\PYG{p}{,} \PYG{l+m+mi}{5}\PYG{p}{)}         \PYG{c+c1}{\PYGZsh{} Tamaño del gráfico}
    \PYG{n}{plt}\PYG{o}{.}\PYG{n}{rcParams}\PYG{p}{[}\PYG{l+s+s1}{\PYGZsq{}}\PYG{l+s+s1}{font.size}\PYG{l+s+s1}{\PYGZsq{}}\PYG{p}{]} \PYG{o}{=} \PYG{l+s+s1}{\PYGZsq{}}\PYG{l+s+s1}{14}\PYG{l+s+s1}{\PYGZsq{}}  \PYG{c+c1}{\PYGZsh{} tamaño de  fuente}
    
    \PYG{c+c1}{\PYGZsh{} parámetros de entrada}
    \PYG{n}{tatm} \PYG{o}{=} \PYG{l+m+mf}{100E6}                      \PYG{c+c1}{\PYGZsh{} espesor de la atmósfera 100 km}
    \PYG{n}{N} \PYG{o}{=} \PYG{p}{(}\PYG{l+m+mf}{1.0}\PYG{p}{,}\PYG{l+m+mf}{1.0}\PYG{p}{,}\PYG{l+m+mf}{1.0}\PYG{p}{)}                 \PYG{c+c1}{\PYGZsh{} indice de refracción superior, intermedio e inferior}
    \PYG{n}{fvp} \PYG{o}{=} \PYG{l+m+mf}{1E\PYGZhy{}8}                        \PYG{c+c1}{\PYGZsh{} fracción de volúmen de las partículas}
    \PYG{n}{Dp} \PYG{o}{=} \PYG{l+m+mf}{0.010}                        \PYG{c+c1}{\PYGZsh{} diámetro de las partículas}
    \PYG{n}{Np} \PYG{o}{=} \PYG{l+m+mf}{1.5}\PYG{o}{*}\PYG{n}{lam1}\PYG{o}{*}\PYG{o}{*}\PYG{l+m+mi}{0}                  \PYG{c+c1}{\PYGZsh{} índice de refracción de las partículas}
    
    \PYG{c+c1}{\PYGZsh{} transmitancia total y especular}
    \PYG{n}{theta} \PYG{o}{=} \PYG{n}{np}\PYG{o}{.}\PYG{n}{radians}\PYG{p}{(}\PYG{n}{theta\PYGZus{}sun}\PYG{p}{)}    \PYG{c+c1}{\PYGZsh{} posición del sol en radianes}
    \PYG{n}{Ttot}\PYG{p}{,} \PYG{n}{Tspec} \PYG{o}{=} \PYG{n}{rt}\PYG{o}{.}\PYG{n}{T\PYGZus{}beer\PYGZus{}lambert}\PYG{p}{(}\PYG{n}{lam1}\PYG{p}{,}\PYG{n}{theta}\PYG{p}{,}\PYG{n}{tatm}\PYG{p}{,}\PYG{n}{N}\PYG{p}{,}\PYG{n}{fvp}\PYG{p}{,}\PYG{n}{Dp}\PYG{p}{,}\PYG{n}{Np}\PYG{p}{)}\PYG{p}{[}\PYG{p}{:}\PYG{l+m+mi}{2}\PYG{p}{]}
    
    \PYG{n}{Tdif} \PYG{o}{=} \PYG{n}{Ttot} \PYG{o}{\PYGZhy{}} \PYG{n}{Tspec}              \PYG{c+c1}{\PYGZsh{} transmitancia difusa}
    \PYG{n}{ax}\PYG{o}{.}\PYG{n}{plot}\PYG{p}{(}\PYG{n}{lam1}\PYG{p}{,}\PYG{n}{Tdif}\PYG{p}{,}\PYG{l+s+s1}{\PYGZsq{}}\PYG{l+s+s1}{\PYGZhy{}k}\PYG{l+s+s1}{\PYGZsq{}}\PYG{p}{,}\PYG{n}{label} \PYG{o}{=} \PYG{l+s+s1}{\PYGZsq{}}\PYG{l+s+s1}{Tdif}\PYG{l+s+s1}{\PYGZsq{}}\PYG{p}{)}
    \PYG{n}{ax}\PYG{o}{.}\PYG{n}{set\PYGZus{}xlabel}\PYG{p}{(}\PYG{l+s+s1}{\PYGZsq{}}\PYG{l+s+s1}{Longitud de onda (\PYGZdl{}}\PYG{l+s+s1}{\PYGZbs{}}\PYG{l+s+s1}{mu\PYGZdl{}m)}\PYG{l+s+s1}{\PYGZsq{}}\PYG{p}{)}
    \PYG{n}{ax}\PYG{o}{.}\PYG{n}{set\PYGZus{}ylabel}\PYG{p}{(}\PYG{l+s+s1}{\PYGZsq{}}\PYG{l+s+s1}{Transmisividad}\PYG{l+s+s1}{\PYGZsq{}}\PYG{p}{)}
    \PYG{n}{ax}\PYG{o}{.}\PYG{n}{set\PYGZus{}title}\PYG{p}{(}\PYG{l+s+sa}{r}\PYG{l+s+s1}{\PYGZsq{}}\PYG{l+s+s1}{Posición del sol, \PYGZdl{}}\PYG{l+s+s1}{\PYGZbs{}}\PYG{l+s+s1}{theta\PYGZus{}}\PYG{l+s+s1}{\PYGZbs{}}\PYG{l+s+s1}{mathrm}\PYG{l+s+si}{\PYGZob{}sun\PYGZcb{}}\PYG{l+s+s1}{\PYGZdl{}=}\PYG{l+s+si}{\PYGZpc{}.1f}\PYG{l+s+s1}{°}\PYG{l+s+s1}{\PYGZsq{}}\PYG{o}{\PYGZpc{}} \PYG{p}{(}\PYG{n}{theta\PYGZus{}sun}\PYG{p}{)}\PYG{p}{)}
    \PYG{n}{ax}\PYG{o}{.}\PYG{n}{set\PYGZus{}ylim}\PYG{p}{(}\PYG{l+m+mi}{0}\PYG{p}{,}\PYG{l+m+mf}{1.05}\PYG{p}{)}
    
    \PYG{n}{Dcircle} \PYG{o}{=} \PYG{l+m+mf}{0.25}
    \PYG{n}{ax2} \PYG{o}{=} \PYG{n}{fig}\PYG{o}{.}\PYG{n}{add\PYGZus{}axes}\PYG{p}{(}\PYG{p}{[}\PYG{l+m+mf}{0.11}\PYG{p}{,}\PYG{l+m+mf}{0.15}\PYG{p}{,} \PYG{n}{Dcircle}\PYG{p}{,} \PYG{n}{Dcircle}\PYG{p}{]}\PYG{p}{)}
    \PYG{n}{Irad} \PYG{o}{=} \PYG{n}{Tdif}\PYG{o}{*}\PYG{n}{AM15}\PYG{p}{(}\PYG{n}{lam1}\PYG{p}{)}
    \PYG{n}{html\PYGZus{}rgb} \PYG{o}{=} \PYG{n}{cs}\PYG{o}{.}\PYG{n}{spec\PYGZus{}to\PYGZus{}rgb}\PYG{p}{(}\PYG{n}{Irad}\PYG{p}{,} \PYG{n}{out\PYGZus{}fmt}\PYG{o}{=}\PYG{l+s+s1}{\PYGZsq{}}\PYG{l+s+s1}{html}\PYG{l+s+s1}{\PYGZsq{}}\PYG{p}{)}
    \PYG{n}{Circle} \PYG{o}{=} \PYG{n}{plt}\PYG{o}{.}\PYG{n}{Circle}\PYG{p}{(}\PYG{p}{(}\PYG{l+m+mi}{0}\PYG{p}{,} \PYG{l+m+mi}{0}\PYG{p}{)}\PYG{p}{,} \PYG{n}{Dcircle}\PYG{p}{,} \PYG{n}{color}\PYG{o}{=}\PYG{n}{html\PYGZus{}rgb}\PYG{p}{)}
    \PYG{n}{ax2}\PYG{o}{.}\PYG{n}{add\PYGZus{}patch}\PYG{p}{(}\PYG{n}{Circle}\PYG{p}{)}
    \PYG{n}{ax2}\PYG{o}{.}\PYG{n}{set\PYGZus{}aspect}\PYG{p}{(}\PYG{l+s+s1}{\PYGZsq{}}\PYG{l+s+s1}{equal}\PYG{l+s+s1}{\PYGZsq{}}\PYG{p}{)}
    \PYG{n}{ax2}\PYG{o}{.}\PYG{n}{set\PYGZus{}xlim}\PYG{p}{(}\PYG{o}{\PYGZhy{}}\PYG{n}{Dcircle}\PYG{o}{*}\PYG{l+m+mf}{1.2}\PYG{p}{,}\PYG{n}{Dcircle}\PYG{o}{*}\PYG{l+m+mf}{1.2}\PYG{p}{)}
    \PYG{n}{ax2}\PYG{o}{.}\PYG{n}{set\PYGZus{}ylim}\PYG{p}{(}\PYG{o}{\PYGZhy{}}\PYG{n}{Dcircle}\PYG{o}{*}\PYG{l+m+mf}{1.2}\PYG{p}{,}\PYG{n}{Dcircle}\PYG{o}{*}\PYG{l+m+mf}{1.2}\PYG{p}{)}
    \PYG{n}{ax2}\PYG{o}{.}\PYG{n}{set\PYGZus{}xticks}\PYG{p}{(}\PYG{p}{[}\PYG{p}{]}\PYG{p}{)}
    \PYG{n}{ax2}\PYG{o}{.}\PYG{n}{set\PYGZus{}yticks}\PYG{p}{(}\PYG{p}{[}\PYG{p}{]}\PYG{p}{)}
    \PYG{n}{ax2}\PYG{o}{.}\PYG{n}{set\PYGZus{}facecolor}\PYG{p}{(}\PYG{l+s+s1}{\PYGZsq{}}\PYG{l+s+s1}{k}\PYG{l+s+s1}{\PYGZsq{}}\PYG{p}{)}
\end{sphinxVerbatim}

\end{sphinxuseclass}\end{sphinxVerbatimInput}

\end{sphinxuseclass}
\begin{sphinxuseclass}{cell}\begin{sphinxVerbatimInput}

\begin{sphinxuseclass}{cell_input}
\begin{sphinxVerbatim}[commandchars=\\\{\}]
\PYG{c+c1}{\PYGZsh{} from ipywidgets import interact}

\PYG{n+nd}{@interact}\PYG{p}{(}\PYG{n}{theta\PYGZus{}sun}\PYG{o}{=}\PYG{p}{(}\PYG{l+m+mi}{0}\PYG{p}{,}\PYG{l+m+mf}{89.99}\PYG{p}{,}\PYG{l+m+mf}{0.1}\PYG{p}{)}\PYG{p}{)}
\PYG{k}{def} \PYG{n+nf}{g}\PYG{p}{(}\PYG{n}{theta\PYGZus{}sun}\PYG{o}{=}\PYG{l+m+mi}{0}\PYG{p}{)}\PYG{p}{:}
    \PYG{k}{return} \PYG{n}{plot\PYGZus{}atmosphere}\PYG{p}{(}\PYG{n}{theta\PYGZus{}sun}\PYG{p}{)}
\end{sphinxVerbatim}

\end{sphinxuseclass}\end{sphinxVerbatimInput}
\begin{sphinxVerbatimOutput}

\begin{sphinxuseclass}{cell_output}
\begin{sphinxVerbatim}[commandchars=\\\{\}]
interactive(children=(FloatSlider(value=0.0, description=\PYGZsq{}theta\PYGZus{}sun\PYGZsq{}, max=89.99), Output()), \PYGZus{}dom\PYGZus{}classes=(\PYGZsq{}wi…
\end{sphinxVerbatim}

\end{sphinxuseclass}\end{sphinxVerbatimOutput}

\end{sphinxuseclass}

\subsection{Pelicula de material particulado (simulaciones de transferencia radiativa)}
\label{\detokenize{5_TransporteRadiativo/5_TransporteRadiativo:pelicula-de-material-particulado-simulaciones-de-transferencia-radiativa}}
\sphinxAtStartPar
Este caso corresponde a materiales con \(\Lambda_\mathrm{sca} > t_\mathrm{film}\). En este caso los eventos de scattering se producen más de una vez, y decimos que estámos en un régimen de \sphinxstylestrong{scattering múltiple}. Como resultado, los tres términos de la RTE son relevantes y debemos resolver la ecuación mediante simulación computacional.

\noindent{\hspace*{\fill}\sphinxincludegraphics[width=400\sphinxpxdimen]{{multiple_scattering}.png}\hspace*{\fill}}



\sphinxAtStartPar
Consideremos un material de sílice de espesor \(t_\mathrm{film} = 5~\mathrm{mm}\). Evaluaremos los colores de este material en transmisión y reflección para luz incidente normal a la superficie en función de la concentración y el diámetro de las partículas. Utilizamos la función \sphinxcode{\sphinxupquote{ad\_rad\_transfer}} de la librería \sphinxcode{\sphinxupquote{empylib.rad\_transfer}}

\begin{sphinxuseclass}{cell}\begin{sphinxVerbatimInput}

\begin{sphinxuseclass}{cell_input}
\begin{sphinxVerbatim}[commandchars=\\\{\}]
\PYG{k+kn}{import} \PYG{n+nn}{empylib}\PYG{n+nn}{.}\PYG{n+nn}{rad\PYGZus{}transfer} \PYG{k}{as} \PYG{n+nn}{rt}
\PYG{k+kn}{import} \PYG{n+nn}{empylib}\PYG{n+nn}{.}\PYG{n+nn}{nklib} \PYG{k}{as} \PYG{n+nn}{nk}
\PYG{k+kn}{import} \PYG{n+nn}{empylib}\PYG{n+nn}{.}\PYG{n+nn}{miescattering} \PYG{k}{as} \PYG{n+nn}{mie}
\PYG{k+kn}{import} \PYG{n+nn}{numpy} \PYG{k}{as} \PYG{n+nn}{np}
\PYG{k+kn}{import} \PYG{n+nn}{matplotlib}\PYG{n+nn}{.}\PYG{n+nn}{pyplot} \PYG{k}{as} \PYG{n+nn}{plt}

\PYG{n}{lam2} \PYG{o}{=} \PYG{n}{np}\PYG{o}{.}\PYG{n}{linspace}\PYG{p}{(}\PYG{l+m+mf}{0.3}\PYG{p}{,}\PYG{l+m+mf}{1.0}\PYG{p}{,}\PYG{l+m+mi}{100}\PYG{p}{)} \PYG{c+c1}{\PYGZsh{} espectro de longitudes de onda en micrometros}
\PYG{n}{Nlayers} \PYG{o}{=} \PYG{p}{(}\PYG{l+m+mf}{1.0}\PYG{p}{,}\PYG{l+m+mf}{1.5}\PYG{p}{,}\PYG{l+m+mf}{1.0}\PYG{p}{)}   \PYG{c+c1}{\PYGZsh{} indice de refracción superior, intermedio e inferior}
\PYG{n}{Np} \PYG{o}{=} \PYG{n}{nk}\PYG{o}{.}\PYG{n}{silver}\PYG{p}{(}\PYG{n}{lam2}\PYG{p}{)}                \PYG{c+c1}{\PYGZsh{} Índice de refracción de las partículas}
\PYG{n}{cs}\PYG{o}{.}\PYG{n}{interp\PYGZus{}internals}\PYG{p}{(}\PYG{n}{lam2}\PYG{p}{)}

\PYG{k}{def} \PYG{n+nf}{plot\PYGZus{}glass\PYGZus{}silver}\PYG{p}{(}\PYG{n}{fv}\PYG{p}{,}\PYG{n}{D}\PYG{p}{)}\PYG{p}{:}
    \PYG{c+c1}{\PYGZsh{} parámetros de entrada}
    
    \PYG{n}{theta} \PYG{o}{=} \PYG{n}{np}\PYG{o}{.}\PYG{n}{radians}\PYG{p}{(}\PYG{l+m+mi}{0}\PYG{p}{)}       \PYG{c+c1}{\PYGZsh{} 0 grados en radianes}
    \PYG{n}{tfilm} \PYG{o}{=} \PYG{l+m+mi}{5}                   \PYG{c+c1}{\PYGZsh{} espesor en milímetros}
    
    \PYG{n}{fv} \PYG{o}{=} \PYG{n}{fv}\PYG{o}{*}\PYG{l+m+mf}{1E\PYGZhy{}7}                \PYG{c+c1}{\PYGZsh{} fracción de volúmen de las partículas}
    \PYG{n}{D} \PYG{o}{=} \PYG{n}{D}\PYG{o}{*}\PYG{l+m+mf}{1E\PYGZhy{}3}                  \PYG{c+c1}{\PYGZsh{} diámetro de las partículas}
    
    \PYG{n}{qext}\PYG{p}{,} \PYG{n}{qsca} \PYG{o}{=} \PYG{n}{mie}\PYG{o}{.}\PYG{n}{scatter\PYGZus{}efficiency}\PYG{p}{(}\PYG{n}{lam2}\PYG{p}{,}\PYG{n}{Nlayers}\PYG{p}{[}\PYG{l+m+mi}{1}\PYG{p}{]}\PYG{p}{,}\PYG{n}{Np}\PYG{p}{,}\PYG{n}{D}\PYG{p}{)}\PYG{p}{[}\PYG{p}{:}\PYG{l+m+mi}{2}\PYG{p}{]}
    \PYG{n}{qabs} \PYG{o}{=} \PYG{n}{qext} \PYG{o}{\PYGZhy{}} \PYG{n}{qsca}
    \PYG{n}{Rtot}\PYG{p}{,} \PYG{n}{Ttot} \PYG{o}{=} \PYG{n}{rt}\PYG{o}{.}\PYG{n}{ad\PYGZus{}rad\PYGZus{}transfer}\PYG{p}{(}\PYG{n}{lam2}\PYG{p}{,}\PYG{n}{tfilm}\PYG{p}{,}\PYG{n}{Nlayers}\PYG{p}{,}\PYG{n}{fv}\PYG{p}{,}\PYG{n}{D}\PYG{p}{,}\PYG{n}{Np}\PYG{p}{)}

    \PYG{n}{fig}\PYG{p}{,} \PYG{n}{ax} \PYG{o}{=} \PYG{n}{plt}\PYG{o}{.}\PYG{n}{subplots}\PYG{p}{(}\PYG{l+m+mi}{1}\PYG{p}{,}\PYG{l+m+mi}{3}\PYG{p}{)}
    \PYG{n}{fig}\PYG{o}{.}\PYG{n}{set\PYGZus{}size\PYGZus{}inches}\PYG{p}{(}\PYG{l+m+mi}{20}\PYG{p}{,} \PYG{l+m+mi}{5}\PYG{p}{)}
    \PYG{n}{plt}\PYG{o}{.}\PYG{n}{rcParams}\PYG{p}{[}\PYG{l+s+s1}{\PYGZsq{}}\PYG{l+s+s1}{font.size}\PYG{l+s+s1}{\PYGZsq{}}\PYG{p}{]} \PYG{o}{=} \PYG{l+s+s1}{\PYGZsq{}}\PYG{l+s+s1}{16}\PYG{l+s+s1}{\PYGZsq{}}
    
    \PYG{n}{ax}\PYG{p}{[}\PYG{l+m+mi}{0}\PYG{p}{]}\PYG{o}{.}\PYG{n}{plot}\PYG{p}{(}\PYG{n}{lam2}\PYG{p}{,}\PYG{n}{qsca}\PYG{p}{,}\PYG{l+s+s1}{\PYGZsq{}}\PYG{l+s+s1}{\PYGZhy{}r}\PYG{l+s+s1}{\PYGZsq{}}\PYG{p}{,}\PYG{n}{label}\PYG{o}{=}\PYG{l+s+s1}{\PYGZsq{}}\PYG{l+s+s1}{\PYGZdl{}C\PYGZus{}}\PYG{l+s+s1}{\PYGZbs{}}\PYG{l+s+s1}{mathrm}\PYG{l+s+si}{\PYGZob{}sca\PYGZcb{}}\PYG{l+s+s1}{ A\PYGZus{}c\PYGZdl{}}\PYG{l+s+s1}{\PYGZsq{}}\PYG{p}{)}
    \PYG{n}{ax}\PYG{p}{[}\PYG{l+m+mi}{0}\PYG{p}{]}\PYG{o}{.}\PYG{n}{plot}\PYG{p}{(}\PYG{n}{lam2}\PYG{p}{,}\PYG{n}{qabs}\PYG{p}{,}\PYG{l+s+s1}{\PYGZsq{}}\PYG{l+s+s1}{\PYGZhy{}b}\PYG{l+s+s1}{\PYGZsq{}}\PYG{p}{,}\PYG{n}{label}\PYG{o}{=}\PYG{l+s+s1}{\PYGZsq{}}\PYG{l+s+s1}{\PYGZdl{}C\PYGZus{}}\PYG{l+s+s1}{\PYGZbs{}}\PYG{l+s+s1}{mathrm}\PYG{l+s+si}{\PYGZob{}abs\PYGZcb{}}\PYG{l+s+s1}{ A\PYGZus{}c\PYGZdl{}}\PYG{l+s+s1}{\PYGZsq{}}\PYG{p}{)}
    \PYG{n}{ax}\PYG{p}{[}\PYG{l+m+mi}{0}\PYG{p}{]}\PYG{o}{.}\PYG{n}{set\PYGZus{}xlabel}\PYG{p}{(}\PYG{l+s+s1}{\PYGZsq{}}\PYG{l+s+s1}{Longitud de onda (\PYGZdl{}}\PYG{l+s+s1}{\PYGZbs{}}\PYG{l+s+s1}{mu\PYGZdl{}m)}\PYG{l+s+s1}{\PYGZsq{}}\PYG{p}{)}
    \PYG{n}{ax}\PYG{p}{[}\PYG{l+m+mi}{0}\PYG{p}{]}\PYG{o}{.}\PYG{n}{set\PYGZus{}ylabel}\PYG{p}{(}\PYG{l+s+s1}{\PYGZsq{}}\PYG{l+s+s1}{Eficiencia transversal}\PYG{l+s+s1}{\PYGZsq{}}\PYG{p}{)}
    \PYG{n}{ax}\PYG{p}{[}\PYG{l+m+mi}{0}\PYG{p}{]}\PYG{o}{.}\PYG{n}{set\PYGZus{}title}\PYG{p}{(}\PYG{l+s+s1}{\PYGZsq{}}\PYG{l+s+s1}{Partícula de plata (D=}\PYG{l+s+si}{\PYGZpc{}.0f}\PYG{l+s+s1}{ nm)}\PYG{l+s+s1}{\PYGZsq{}} \PYG{o}{\PYGZpc{}} \PYG{p}{(}\PYG{n}{D}\PYG{o}{*}\PYG{l+m+mf}{1E3}\PYG{p}{)}\PYG{p}{)}
    \PYG{n}{ax}\PYG{p}{[}\PYG{l+m+mi}{0}\PYG{p}{]}\PYG{o}{.}\PYG{n}{legend}\PYG{p}{(}\PYG{p}{)}
    \PYG{n}{ax}\PYG{p}{[}\PYG{l+m+mi}{0}\PYG{p}{]}\PYG{o}{.}\PYG{n}{set\PYGZus{}ylim}\PYG{p}{(}\PYG{l+m+mi}{0}\PYG{p}{,}\PYG{l+m+mi}{10}\PYG{p}{)}
    
    \PYG{n}{ax}\PYG{p}{[}\PYG{l+m+mi}{1}\PYG{p}{]}\PYG{o}{.}\PYG{n}{plot}\PYG{p}{(}\PYG{n}{lam2}\PYG{p}{,}\PYG{n}{Rtot}\PYG{p}{,}\PYG{l+s+s1}{\PYGZsq{}}\PYG{l+s+s1}{\PYGZhy{}r}\PYG{l+s+s1}{\PYGZsq{}}\PYG{p}{,}\PYG{n}{label} \PYG{o}{=} \PYG{l+s+s1}{\PYGZsq{}}\PYG{l+s+s1}{Rtot}\PYG{l+s+s1}{\PYGZsq{}}\PYG{p}{)}
    \PYG{n}{ax}\PYG{p}{[}\PYG{l+m+mi}{1}\PYG{p}{]}\PYG{o}{.}\PYG{n}{plot}\PYG{p}{(}\PYG{n}{lam2}\PYG{p}{,}\PYG{n}{Ttot}\PYG{p}{,}\PYG{l+s+s1}{\PYGZsq{}}\PYG{l+s+s1}{\PYGZhy{}b}\PYG{l+s+s1}{\PYGZsq{}}\PYG{p}{,}\PYG{n}{label} \PYG{o}{=} \PYG{l+s+s1}{\PYGZsq{}}\PYG{l+s+s1}{Ttot}\PYG{l+s+s1}{\PYGZsq{}}\PYG{p}{)}
    \PYG{n}{ax}\PYG{p}{[}\PYG{l+m+mi}{1}\PYG{p}{]}\PYG{o}{.}\PYG{n}{set\PYGZus{}xlabel}\PYG{p}{(}\PYG{l+s+s1}{\PYGZsq{}}\PYG{l+s+s1}{Longitud de onda (\PYGZdl{}}\PYG{l+s+s1}{\PYGZbs{}}\PYG{l+s+s1}{mu\PYGZdl{}m)}\PYG{l+s+s1}{\PYGZsq{}}\PYG{p}{)}
    \PYG{n}{ax}\PYG{p}{[}\PYG{l+m+mi}{1}\PYG{p}{]}\PYG{o}{.}\PYG{n}{set\PYGZus{}ylabel}\PYG{p}{(}\PYG{l+s+s1}{\PYGZsq{}}\PYG{l+s+s1}{Transmisividad}\PYG{l+s+s1}{\PYGZsq{}}\PYG{p}{)}
    \PYG{n}{ax}\PYG{p}{[}\PYG{l+m+mi}{1}\PYG{p}{]}\PYG{o}{.}\PYG{n}{set\PYGZus{}title}\PYG{p}{(}\PYG{l+s+sa}{r}\PYG{l+s+s1}{\PYGZsq{}}\PYG{l+s+s1}{Sílice con plata (fv = }\PYG{l+s+si}{\PYGZpc{}.3e}\PYG{l+s+s1}{ }\PYG{l+s+si}{\PYGZpc{}\PYGZpc{}}\PYG{l+s+s1}{)}\PYG{l+s+s1}{\PYGZsq{}} \PYG{o}{\PYGZpc{}} \PYG{p}{(}\PYG{n}{fv}\PYG{o}{*}\PYG{l+m+mi}{100}\PYG{p}{)} \PYG{p}{)}
    \PYG{n}{ax}\PYG{p}{[}\PYG{l+m+mi}{1}\PYG{p}{]}\PYG{o}{.}\PYG{n}{legend}\PYG{p}{(}\PYG{p}{)}
    \PYG{n}{ax}\PYG{p}{[}\PYG{l+m+mi}{1}\PYG{p}{]}\PYG{o}{.}\PYG{n}{set\PYGZus{}ylim}\PYG{p}{(}\PYG{l+m+mi}{0}\PYG{p}{,}\PYG{l+m+mi}{1}\PYG{p}{)}

    \PYG{n}{Dcircle} \PYG{o}{=} \PYG{l+m+mf}{0.20}
    \PYG{n}{html\PYGZus{}rgb} \PYG{o}{=} \PYG{n}{cs}\PYG{o}{.}\PYG{n}{spec\PYGZus{}to\PYGZus{}rgb}\PYG{p}{(}\PYG{n}{Ttot}\PYG{o}{*}\PYG{n}{AM15}\PYG{p}{(}\PYG{n}{lam2}\PYG{p}{)}\PYG{p}{,} \PYG{n}{out\PYGZus{}fmt}\PYG{o}{=}\PYG{l+s+s1}{\PYGZsq{}}\PYG{l+s+s1}{html}\PYG{l+s+s1}{\PYGZsq{}}\PYG{p}{)}
    \PYG{n}{Circle} \PYG{o}{=} \PYG{n}{plt}\PYG{o}{.}\PYG{n}{Circle}\PYG{p}{(}\PYG{p}{(}\PYG{l+m+mi}{0}\PYG{p}{,} \PYG{l+m+mi}{0}\PYG{p}{)}\PYG{p}{,} \PYG{n}{Dcircle}\PYG{p}{,} \PYG{n}{color}\PYG{o}{=}\PYG{n}{html\PYGZus{}rgb}\PYG{p}{)}
    \PYG{n}{ax}\PYG{p}{[}\PYG{l+m+mi}{2}\PYG{p}{]}\PYG{o}{.}\PYG{n}{add\PYGZus{}patch}\PYG{p}{(}\PYG{n}{Circle}\PYG{p}{)}
    \PYG{n}{ax}\PYG{p}{[}\PYG{l+m+mi}{2}\PYG{p}{]}\PYG{o}{.}\PYG{n}{annotate}\PYG{p}{(}\PYG{l+s+s1}{\PYGZsq{}}\PYG{l+s+s1}{Luz Trasera}\PYG{l+s+s1}{\PYGZsq{}}\PYG{p}{,} \PYG{n}{xy}\PYG{o}{=}\PYG{p}{(}\PYG{l+m+mi}{0}\PYG{p}{,} \PYG{l+m+mi}{0}\PYG{p}{)}\PYG{p}{,} \PYG{n}{va}\PYG{o}{=}\PYG{l+s+s1}{\PYGZsq{}}\PYG{l+s+s1}{center}\PYG{l+s+s1}{\PYGZsq{}}\PYG{p}{,} \PYG{n}{ha}\PYG{o}{=}\PYG{l+s+s1}{\PYGZsq{}}\PYG{l+s+s1}{center}\PYG{l+s+s1}{\PYGZsq{}}\PYG{p}{)}

    \PYG{n}{html\PYGZus{}rgb} \PYG{o}{=} \PYG{n}{cs}\PYG{o}{.}\PYG{n}{spec\PYGZus{}to\PYGZus{}rgb}\PYG{p}{(}\PYG{n}{Rtot}\PYG{o}{*}\PYG{n}{AM15}\PYG{p}{(}\PYG{n}{lam2}\PYG{p}{)}\PYG{p}{,} \PYG{n}{out\PYGZus{}fmt}\PYG{o}{=}\PYG{l+s+s1}{\PYGZsq{}}\PYG{l+s+s1}{html}\PYG{l+s+s1}{\PYGZsq{}}\PYG{p}{)}
    \PYG{n}{Circle} \PYG{o}{=} \PYG{n}{plt}\PYG{o}{.}\PYG{n}{Circle}\PYG{p}{(}\PYG{p}{(}\PYG{n}{Dcircle}\PYG{o}{*}\PYG{l+m+mf}{1.2}\PYG{o}{*}\PYG{l+m+mi}{2}\PYG{p}{,} \PYG{l+m+mi}{0}\PYG{p}{)}\PYG{p}{,} \PYG{n}{Dcircle}\PYG{p}{,} \PYG{n}{color}\PYG{o}{=}\PYG{n}{html\PYGZus{}rgb}\PYG{p}{)}
    \PYG{n}{ax}\PYG{p}{[}\PYG{l+m+mi}{2}\PYG{p}{]}\PYG{o}{.}\PYG{n}{add\PYGZus{}patch}\PYG{p}{(}\PYG{n}{Circle}\PYG{p}{)}
    \PYG{n}{ax}\PYG{p}{[}\PYG{l+m+mi}{2}\PYG{p}{]}\PYG{o}{.}\PYG{n}{annotate}\PYG{p}{(}\PYG{l+s+s1}{\PYGZsq{}}\PYG{l+s+s1}{Luz Frontal}\PYG{l+s+s1}{\PYGZsq{}}\PYG{p}{,} \PYG{n}{xy}\PYG{o}{=}\PYG{p}{(}\PYG{n}{Dcircle}\PYG{o}{*}\PYG{l+m+mf}{1.2}\PYG{o}{*}\PYG{l+m+mi}{2}\PYG{p}{,} \PYG{l+m+mi}{0}\PYG{p}{)}\PYG{p}{,} \PYG{n}{va}\PYG{o}{=}\PYG{l+s+s1}{\PYGZsq{}}\PYG{l+s+s1}{center}\PYG{l+s+s1}{\PYGZsq{}}\PYG{p}{,} \PYG{n}{ha}\PYG{o}{=}\PYG{l+s+s1}{\PYGZsq{}}\PYG{l+s+s1}{center}\PYG{l+s+s1}{\PYGZsq{}}\PYG{p}{)}

    \PYG{n}{ax}\PYG{p}{[}\PYG{l+m+mi}{2}\PYG{p}{]}\PYG{o}{.}\PYG{n}{set\PYGZus{}aspect}\PYG{p}{(}\PYG{l+s+s1}{\PYGZsq{}}\PYG{l+s+s1}{equal}\PYG{l+s+s1}{\PYGZsq{}}\PYG{p}{)}
    \PYG{n}{ax}\PYG{p}{[}\PYG{l+m+mi}{2}\PYG{p}{]}\PYG{o}{.}\PYG{n}{set\PYGZus{}xlim}\PYG{p}{(}\PYG{o}{\PYGZhy{}}\PYG{n}{Dcircle}\PYG{o}{*}\PYG{l+m+mf}{1.2}\PYG{p}{,}\PYG{n}{Dcircle}\PYG{o}{*}\PYG{l+m+mf}{1.2}\PYG{o}{*}\PYG{l+m+mi}{3}\PYG{p}{)}
    \PYG{n}{ax}\PYG{p}{[}\PYG{l+m+mi}{2}\PYG{p}{]}\PYG{o}{.}\PYG{n}{set\PYGZus{}ylim}\PYG{p}{(}\PYG{o}{\PYGZhy{}}\PYG{n}{Dcircle}\PYG{o}{*}\PYG{l+m+mf}{1.2}\PYG{p}{,}\PYG{n}{Dcircle}\PYG{o}{*}\PYG{l+m+mf}{1.2}\PYG{p}{)}
    \PYG{n}{ax}\PYG{p}{[}\PYG{l+m+mi}{2}\PYG{p}{]}\PYG{o}{.}\PYG{n}{set\PYGZus{}xticks}\PYG{p}{(}\PYG{p}{[}\PYG{p}{]}\PYG{p}{)}
    \PYG{n}{ax}\PYG{p}{[}\PYG{l+m+mi}{2}\PYG{p}{]}\PYG{o}{.}\PYG{n}{set\PYGZus{}yticks}\PYG{p}{(}\PYG{p}{[}\PYG{p}{]}\PYG{p}{)}
    \PYG{n}{ax}\PYG{p}{[}\PYG{l+m+mi}{2}\PYG{p}{]}\PYG{o}{.}\PYG{n}{set\PYGZus{}facecolor}\PYG{p}{(}\PYG{l+s+s1}{\PYGZsq{}}\PYG{l+s+s1}{k}\PYG{l+s+s1}{\PYGZsq{}}\PYG{p}{)}
    \PYG{n}{plt}\PYG{o}{.}\PYG{n}{subplots\PYGZus{}adjust}\PYG{p}{(}\PYG{n}{wspace}\PYG{o}{=}\PYG{l+m+mf}{0.3}\PYG{p}{)}
\end{sphinxVerbatim}

\end{sphinxuseclass}\end{sphinxVerbatimInput}

\end{sphinxuseclass}
\begin{sphinxuseclass}{cell}\begin{sphinxVerbatimInput}

\begin{sphinxuseclass}{cell_input}
\begin{sphinxVerbatim}[commandchars=\\\{\}]
\PYG{k+kn}{from} \PYG{n+nn}{ipywidgets} \PYG{k+kn}{import} \PYG{n}{interact}

\PYG{n+nd}{@interact}\PYG{p}{(}\PYG{n}{fv}\PYG{o}{=}\PYG{p}{(}\PYG{l+m+mi}{1}\PYG{p}{,}\PYG{l+m+mi}{100}\PYG{p}{,}\PYG{l+m+mi}{1}\PYG{p}{)}\PYG{p}{,} \PYG{n}{D} \PYG{o}{=} \PYG{p}{(}\PYG{l+m+mi}{10}\PYG{p}{,}\PYG{l+m+mi}{200}\PYG{p}{,}\PYG{l+m+mi}{1}\PYG{p}{)}\PYG{p}{)}
\PYG{k}{def} \PYG{n+nf}{g}\PYG{p}{(}\PYG{n}{fv}\PYG{o}{=}\PYG{l+m+mi}{20}\PYG{p}{,} \PYG{n}{D} \PYG{o}{=} \PYG{l+m+mi}{70}\PYG{p}{)}\PYG{p}{:}
    \PYG{k}{return} \PYG{n}{plot\PYGZus{}glass\PYGZus{}silver}\PYG{p}{(}\PYG{n}{fv}\PYG{p}{,}\PYG{n}{D}\PYG{p}{)}
\end{sphinxVerbatim}

\end{sphinxuseclass}\end{sphinxVerbatimInput}
\begin{sphinxVerbatimOutput}

\begin{sphinxuseclass}{cell_output}
\begin{sphinxVerbatim}[commandchars=\\\{\}]
interactive(children=(IntSlider(value=20, description=\PYGZsq{}fv\PYGZsq{}, min=1), IntSlider(value=70, description=\PYGZsq{}D\PYGZsq{}, max=2…
\end{sphinxVerbatim}

\end{sphinxuseclass}\end{sphinxVerbatimOutput}

\end{sphinxuseclass}
\sphinxAtStartPar
Cuando la concentración de partículas es densa, el medio se vuelve opaco. Este régimen se denomina \sphinxstylestrong{scattering difuso} y permite explicar, entre otras cosas, el color de las nubes o la pintura blanca

\noindent{\hspace*{\fill}\sphinxincludegraphics[width=400\sphinxpxdimen]{{diffuse_scattering}.png}\hspace*{\fill}}



\sphinxAtStartPar
Un ejemplo interesante corresponde a la leche. En términos simples, la leche es una emulsión formada por pequeñas partículas de grasa dispersas en un medio acuoso.

\noindent{\hspace*{\fill}\sphinxincludegraphics[width=400\sphinxpxdimen]{{milk_microscope}.png}\hspace*{\fill}}



\sphinxAtStartPar
Fuente: \sphinxhref{https://www.ncbi.nlm.nih.gov/pmc/articles/PMC6836175/}{Braun K., Hanewald A. and Vilgis T. Foods 8(10): 483(2019)}

\sphinxAtStartPar
Como aproximación, consideremos un medio de espesor \(1\) cm, compuesto por agua \(N_h = 1.3\) y pequeñas partículas esféricas de aceite \(N_p = 1.5\). La emulsión considera un 60\% de partículas de aceite por volumen.

\begin{sphinxuseclass}{cell}\begin{sphinxVerbatimInput}

\begin{sphinxuseclass}{cell_input}
\begin{sphinxVerbatim}[commandchars=\\\{\}]
\PYG{o}{\PYGZpc{}\PYGZpc{}capture} showplot
\PYG{c+c1}{\PYGZsh{} import empylib.nklib as nk}
\PYG{k+kn}{import} \PYG{n+nn}{numpy} \PYG{k}{as} \PYG{n+nn}{np}
\PYG{k+kn}{import} \PYG{n+nn}{empylib}\PYG{n+nn}{.}\PYG{n+nn}{rad\PYGZus{}transfer} \PYG{k}{as} \PYG{n+nn}{rt}

\PYG{c+c1}{\PYGZsh{} Solo modificar estos parámetros}
\PYG{c+c1}{\PYGZsh{}\PYGZhy{}\PYGZhy{}\PYGZhy{}\PYGZhy{}\PYGZhy{}\PYGZhy{}\PYGZhy{}\PYGZhy{}\PYGZhy{}\PYGZhy{}\PYGZhy{}\PYGZhy{}\PYGZhy{}\PYGZhy{}\PYGZhy{}\PYGZhy{}\PYGZhy{}\PYGZhy{}\PYGZhy{}\PYGZhy{}\PYGZhy{}\PYGZhy{}\PYGZhy{}\PYGZhy{}\PYGZhy{}\PYGZhy{}\PYGZhy{}\PYGZhy{}\PYGZhy{}\PYGZhy{}\PYGZhy{}\PYGZhy{}\PYGZhy{}\PYGZhy{}\PYGZhy{}\PYGZhy{}\PYGZhy{}\PYGZhy{}\PYGZhy{}\PYGZhy{}\PYGZhy{}\PYGZhy{}\PYGZhy{}\PYGZhy{}\PYGZhy{}\PYGZhy{}\PYGZhy{}\PYGZhy{}\PYGZhy{}\PYGZhy{}\PYGZhy{}\PYGZhy{}\PYGZhy{}\PYGZhy{}\PYGZhy{}\PYGZhy{}\PYGZhy{}\PYGZhy{}\PYGZhy{}\PYGZhy{}\PYGZhy{}\PYGZhy{}\PYGZhy{}}
\PYG{n}{lam3} \PYG{o}{=} \PYG{n}{np}\PYG{o}{.}\PYG{n}{linspace}\PYG{p}{(}\PYG{l+m+mf}{0.3}\PYG{p}{,}\PYG{l+m+mf}{1.0}\PYG{p}{,}\PYG{l+m+mi}{100}\PYG{p}{)}   \PYG{c+c1}{\PYGZsh{} espectro de longitudes de onda}
\PYG{n}{tfilm} \PYG{o}{=} \PYG{l+m+mi}{10}                        \PYG{c+c1}{\PYGZsh{} espesor en milímetros}
\PYG{n}{fv} \PYG{o}{=} \PYG{l+m+mf}{0.60}                          \PYG{c+c1}{\PYGZsh{} fracción de volúmen de los poros}
\PYG{n}{D} \PYG{o}{=} \PYG{l+m+mf}{1.0}                           \PYG{c+c1}{\PYGZsh{} diámetro de los poros (micrones)}
\PYG{n}{Nh2o} \PYG{o}{=} \PYG{l+m+mf}{1.3}                        \PYG{c+c1}{\PYGZsh{} Índice de refracción del agua}
\PYG{n}{Noil} \PYG{o}{=} \PYG{l+m+mf}{1.5}                        \PYG{c+c1}{\PYGZsh{} índice de refracción partículas de aceite}
\PYG{c+c1}{\PYGZsh{}\PYGZhy{}\PYGZhy{}\PYGZhy{}\PYGZhy{}\PYGZhy{}\PYGZhy{}\PYGZhy{}\PYGZhy{}\PYGZhy{}\PYGZhy{}\PYGZhy{}\PYGZhy{}\PYGZhy{}\PYGZhy{}\PYGZhy{}\PYGZhy{}\PYGZhy{}\PYGZhy{}\PYGZhy{}\PYGZhy{}\PYGZhy{}\PYGZhy{}\PYGZhy{}\PYGZhy{}\PYGZhy{}\PYGZhy{}\PYGZhy{}\PYGZhy{}\PYGZhy{}\PYGZhy{}\PYGZhy{}\PYGZhy{}\PYGZhy{}\PYGZhy{}\PYGZhy{}\PYGZhy{}\PYGZhy{}\PYGZhy{}\PYGZhy{}\PYGZhy{}\PYGZhy{}\PYGZhy{}\PYGZhy{}\PYGZhy{}\PYGZhy{}\PYGZhy{}\PYGZhy{}\PYGZhy{}\PYGZhy{}\PYGZhy{}\PYGZhy{}\PYGZhy{}\PYGZhy{}\PYGZhy{}\PYGZhy{}\PYGZhy{}\PYGZhy{}\PYGZhy{}\PYGZhy{}\PYGZhy{}\PYGZhy{}\PYGZhy{}\PYGZhy{}}
\PYG{n}{Rtot}\PYG{p}{,} \PYG{n}{Ttot} \PYG{o}{=} \PYG{n}{rt}\PYG{o}{.}\PYG{n}{ad\PYGZus{}rad\PYGZus{}transfer}\PYG{p}{(}\PYG{n}{lam3}\PYG{p}{,}\PYG{n}{tfilm}\PYG{p}{,}\PYG{p}{(}\PYG{l+m+mf}{1.0}\PYG{p}{,}\PYG{n}{Nh2o}\PYG{p}{,}\PYG{l+m+mf}{1.0}\PYG{p}{)}\PYG{p}{,}\PYG{n}{fv}\PYG{p}{,}\PYG{n}{D}\PYG{p}{,}\PYG{n}{Noil}\PYG{p}{)}

\PYG{n}{fig}\PYG{p}{,} \PYG{n}{ax} \PYG{o}{=} \PYG{n}{plt}\PYG{o}{.}\PYG{n}{subplots}\PYG{p}{(}\PYG{p}{)}
\PYG{n}{fig}\PYG{o}{.}\PYG{n}{set\PYGZus{}size\PYGZus{}inches}\PYG{p}{(}\PYG{l+m+mi}{7}\PYG{p}{,} \PYG{l+m+mi}{5}\PYG{p}{)}
\PYG{n}{plt}\PYG{o}{.}\PYG{n}{rcParams}\PYG{p}{[}\PYG{l+s+s1}{\PYGZsq{}}\PYG{l+s+s1}{font.size}\PYG{l+s+s1}{\PYGZsq{}}\PYG{p}{]} \PYG{o}{=} \PYG{l+s+s1}{\PYGZsq{}}\PYG{l+s+s1}{16}\PYG{l+s+s1}{\PYGZsq{}}
\PYG{n}{ax}\PYG{o}{.}\PYG{n}{plot}\PYG{p}{(}\PYG{n}{lam3}\PYG{p}{,}\PYG{n}{Rtot}\PYG{p}{,}\PYG{l+s+s1}{\PYGZsq{}}\PYG{l+s+s1}{\PYGZhy{}r}\PYG{l+s+s1}{\PYGZsq{}}\PYG{p}{,}\PYG{n}{label}\PYG{o}{=}\PYG{l+s+s1}{\PYGZsq{}}\PYG{l+s+s1}{R}\PYG{l+s+s1}{\PYGZsq{}}\PYG{p}{)}
\PYG{n}{ax}\PYG{o}{.}\PYG{n}{plot}\PYG{p}{(}\PYG{n}{lam3}\PYG{p}{,}\PYG{n}{Ttot}\PYG{p}{,}\PYG{l+s+s1}{\PYGZsq{}}\PYG{l+s+s1}{\PYGZhy{}b}\PYG{l+s+s1}{\PYGZsq{}}\PYG{p}{,}\PYG{n}{label}\PYG{o}{=}\PYG{l+s+s1}{\PYGZsq{}}\PYG{l+s+s1}{T}\PYG{l+s+s1}{\PYGZsq{}}\PYG{p}{)}
\PYG{n}{ax}\PYG{o}{.}\PYG{n}{set\PYGZus{}xlabel}\PYG{p}{(}\PYG{l+s+s1}{\PYGZsq{}}\PYG{l+s+s1}{Longitud de onda (\PYGZdl{}}\PYG{l+s+s1}{\PYGZbs{}}\PYG{l+s+s1}{mu\PYGZdl{}m)}\PYG{l+s+s1}{\PYGZsq{}}\PYG{p}{)}
\PYG{n}{ax}\PYG{o}{.}\PYG{n}{set\PYGZus{}ylabel}\PYG{p}{(}\PYG{l+s+s1}{\PYGZsq{}}\PYG{l+s+s1}{Reflectividad / Transmisividad}\PYG{l+s+s1}{\PYGZsq{}}\PYG{p}{)}
\PYG{n}{ax}\PYG{o}{.}\PYG{n}{set\PYGZus{}title}\PYG{p}{(}\PYG{l+s+sa}{r}\PYG{l+s+s1}{\PYGZsq{}}\PYG{l+s+s1}{Leche (fv = }\PYG{l+s+si}{\PYGZpc{}.0f}\PYG{l+s+s1}{ }\PYG{l+s+si}{\PYGZpc{}\PYGZpc{}}\PYG{l+s+s1}{)}\PYG{l+s+s1}{\PYGZsq{}} \PYG{o}{\PYGZpc{}} \PYG{p}{(}\PYG{n}{fv}\PYG{o}{*}\PYG{l+m+mi}{100}\PYG{p}{)}\PYG{p}{)}
\PYG{n}{ax}\PYG{o}{.}\PYG{n}{legend}\PYG{p}{(}\PYG{p}{)}
\PYG{n}{ax}\PYG{o}{.}\PYG{n}{set\PYGZus{}ylim}\PYG{p}{(}\PYG{l+m+mi}{0}\PYG{p}{,}\PYG{l+m+mf}{1.02}\PYG{p}{)}
\PYG{n}{plt}\PYG{o}{.}\PYG{n}{show}
\end{sphinxVerbatim}

\end{sphinxuseclass}\end{sphinxVerbatimInput}

\end{sphinxuseclass}
\begin{sphinxuseclass}{cell}\begin{sphinxVerbatimInput}

\begin{sphinxuseclass}{cell_input}
\begin{sphinxVerbatim}[commandchars=\\\{\}]
\PYG{n}{showplot}\PYG{p}{(}\PYG{p}{)}
\end{sphinxVerbatim}

\end{sphinxuseclass}\end{sphinxVerbatimInput}
\begin{sphinxVerbatimOutput}

\begin{sphinxuseclass}{cell_output}
\begin{sphinxVerbatim}[commandchars=\\\{\}]
\PYGZlt{}function matplotlib.pyplot.show(close=None, block=None)\PYGZgt{}
\end{sphinxVerbatim}

\noindent\sphinxincludegraphics{{5_TransporteRadiativo_68_1}.png}

\end{sphinxuseclass}\end{sphinxVerbatimOutput}

\end{sphinxuseclass}
\sphinxAtStartPar
Como vemos la alta concentración de partículas hace que la refletividad y transmisividad se vuelvan casi uniformes para todas las longitudes de onda. El espectro, así, toma un color blanco frente a una fuente de luz blanca.


\section{Referencias}
\label{\detokenize{5_TransporteRadiativo/5_TransporteRadiativo:referencias}}\begin{itemize}
\item {} 
\sphinxAtStartPar
Chen G. \sphinxstylestrong{Chapter 5 \sphinxhyphen{} Energy Transfer by Waves} in \sphinxstyleemphasis{Nanoscale energy transport and conversion}, 1st Ed, Oxford University Press, 2005

\end{itemize}

\sphinxstepscope

\sphinxAtStartPar
MEC501 \sphinxhyphen{} Manejo y Conversión de Energía Solar Térmica


\chapter{Radiación Térmica}
\label{\detokenize{6_RadiacionTermica/6_RadiacionTermica:radiacion-termica}}\label{\detokenize{6_RadiacionTermica/6_RadiacionTermica::doc}}
\sphinxAtStartPar

Profesor: Francisco Ramírez Cuevas
Fecha: 30 de Septiembre 2022


\section{Introducción a la Transferencia de Calor}
\label{\detokenize{6_RadiacionTermica/6_RadiacionTermica:introduccion-a-la-transferencia-de-calor}}
\sphinxAtStartPar
A nivel molecular, los átomos que compomen la materia siempre está vibrando. La magnitud de estas vibraciones está caracterizada estadísticamente por la temperatura:

\noindent{\hspace*{\fill}\sphinxincludegraphics[width=400\sphinxpxdimen]{{temperature_brownian_motion}.gif}\hspace*{\fill}}

\sphinxAtStartPar
Consideremos un sólido extendido con una diferencia de temperatura, \(\Delta T\), entre sus extremos, tal que el lado izquierdo tiene una mayor temperatura que el lado derecho

\noindent{\hspace*{\fill}\sphinxincludegraphics[width=300\sphinxpxdimen]{{heat_transfer}.gif}\hspace*{\fill}}

\sphinxAtStartPar
Debido a la diferencia de temperatura, la vibración molecular en el lado izquierdo es mayor. Esta energía cinética es transmitida a través del material hacia el lado derecho.

\sphinxAtStartPar
Definimos como \sphinxstylestrong{calor}, \(Q\), a la energía térmica intercambiada entre dos medios cuya diferencia de temperatura es \(\Delta T\). A mayor \(\Delta T\), mayor es el intercambio de calor, matematicamente:
\begin{equation*}
Q \propto \Delta T,\quad\mathrm{J}
\end{equation*}
\sphinxAtStartPar
La \sphinxstylestrong{taza de transferencia de calor}:
\begin{equation*}
\dot{Q} = \frac{dQ}{dt},\quad\mathrm{W}
\end{equation*}
\sphinxAtStartPar
corresponde al calor tranferido por unidad de tiempo.

\sphinxAtStartPar
Por último, definimos como \sphinxstylestrong{flujo de calor}:
\begin{equation*}
q'' = \dot{Q}/A,\quad\frac{\mathrm{W}}{\mathrm{m}^2}
\end{equation*}
\sphinxAtStartPar
a la taza de transferencia de calor por unidad de área.

\sphinxAtStartPar
Existen tres mecanismo de transferencia de calor:
\begin{itemize}
\item {} 
\sphinxAtStartPar
Conducción de calor

\item {} 
\sphinxAtStartPar
Convección de calor

\item {} 
\sphinxAtStartPar
Radiación

\end{itemize}

\noindent{\hspace*{\fill}\sphinxincludegraphics[width=500\sphinxpxdimen]{{heat_transfer_mechanism}.png}\hspace*{\fill}}


\subsection{Conducción de Calor}
\label{\detokenize{6_RadiacionTermica/6_RadiacionTermica:conduccion-de-calor}}
\sphinxAtStartPar
\sphinxstylestrong{Definimos como \sphinxstyleemphasis{conducción de calor} al calor transferido a través de un material en reposo}. El mecanismo generalmente se asocia a \sphinxstylestrong{sólidos}, donde el calor es transferido a travéz de la red atómica del material. Sin embargo, la definición también incluye \sphinxstylestrong{líquidos y gases en reposo.} En este caso, las moléculas se mueven eleatoriamente, de manera tal que la velocidad neta del fluido es cero.

\sphinxAtStartPar
Matemáticamente, para un material de espesor \(t\) y diferencia de temperatura \(\Delta T\), la \sphinxstylestrong{taza de transferencia de calor por conducción} a través de una superficie \(A\), es:

\noindent{\hspace*{\fill}\sphinxincludegraphics[width=700\sphinxpxdimen]{{heat_conduction_formula}.png}\hspace*{\fill}}

\sphinxAtStartPar
La conductividad térmica, \(k_c\), es una propiedad del material que varía según la temperatura.

\noindent{\hspace*{\fill}\sphinxincludegraphics[width=800\sphinxpxdimen]{{thermal_conductivity}.png}\hspace*{\fill}}

\sphinxAtStartPar
En su forma diferencial, \(\dot{Q}_\mathrm{cond}= - k\nabla T\), y para el caso unidimensional:
\label{equation:6_RadiacionTermica/6_RadiacionTermica:0b28a269-51c2-4af5-9ac8-62e47a2035ab}\begin{equation}
\dot{Q}_\mathrm{cond} = - kA\frac{dT}{dx},\quad\mathrm{W}
\end{equation}
\sphinxAtStartPar
A partir de esta fórmula podemos deducir expresiones para taza de transferencia de calor por conducción según la geometría:

\noindent{\hspace*{\fill}\sphinxincludegraphics[width=600\sphinxpxdimen]{{heat_conduction_resistance}.png}\hspace*{\fill}}

\sphinxAtStartPar
Notar que, como fórmula general, podemos expresar la taza de conducción de calor en la forma:
\begin{equation*}
\dot{Q}_\mathrm{cond} = \frac{T_H - T_C}{R_\mathrm{cond}},\quad\mathrm{W}
\end{equation*}
\sphinxAtStartPar
donde, \(R_\mathrm{cond}\) (K/W) es la \sphinxstylestrong{resistencia térmica} asociada al mecanismo de conducción.


\subsection{Convección de Calor}
\label{\detokenize{6_RadiacionTermica/6_RadiacionTermica:conveccion-de-calor}}
\sphinxAtStartPar
\sphinxstylestrong{Definimos como \sphinxstyleemphasis{convección de calor} al calor transferido a través de fluidos en movimiento.} El movimiento de un fluido puede ocurrir naturalmente, debido a los efectos de flotación a raíz de los cambio de densidad con la temperatura; o de forma inducida, como por ejemplo mediante un ventilador.

\sphinxAtStartPar
A partir de esto, clasificamos la transferencia de calor por convección, respectivamente, como:
\begin{itemize}
\item {} 
\sphinxAtStartPar
\sphinxstylestrong{convección natural}

\item {} 
\sphinxAtStartPar
\sphinxstylestrong{convección forzada}.

\end{itemize}

\noindent{\hspace*{\fill}\sphinxincludegraphics[width=300\sphinxpxdimen]{{heat_convection_mechanism}.png}\hspace*{\fill}}

\sphinxAtStartPar
La convección de calor esta asociada al contacto de fluidos con una superficie, \(A\). Así, independiente del mecanismo de convección de calor (natural o forzada), expresamos la \sphinxstylestrong{taza de transferencia de calor por convección} como:

\noindent{\hspace*{\fill}\sphinxincludegraphics[width=700\sphinxpxdimen]{{heat_convection_formula}.png}\hspace*{\fill}}

\sphinxAtStartPar
Notar que la taza de transferencia de calor por convección puede ser expresada en la forma:
\begin{equation*}
\dot{Q}_\mathrm{conv} = \frac{T_\infty - T}{R_\mathrm{conv}},\quad\mathrm{W}
\end{equation*}
\sphinxAtStartPar
donde \(R_\mathrm{conv}=1/hA\) es la resistencia térmica asociada a la convección de calor.

\sphinxAtStartPar
A diferencia de la conducción de calor, el coeficiente convectivo, \(h\), \sphinxstylestrong{no es una propiedad del fluido}. Esto porque no solo depende de las propiedades del fluido (densidad, viscocidad y conductividad térmica, entre otras), sino que además depende de condiciones externas, como la velocidad del flujo, la diferencia de temperaturas, y la geometría del cuerpó sometido a conveccción de calor.

\sphinxAtStartPar
El coeficiente convectivo se determina a partir de relaciones expresadas en términos del número de Nusselt, \(\mathrm{Nu} = \frac{hL_c}{k_f}\), donde \(L_c\) es una longitud característica y \(k_f\) es la conductividad térmica del fluido. En la mayoría de los casos, las relaciones para el número de Nusselt para cada caso se determinan experimentalmente.

\sphinxAtStartPar
Comúnmente, los valores para el número de Nusselt se encuentran dentro de los siguientes rangos:
\begin{itemize}
\item {} 
\sphinxAtStartPar
Convección forzada, \(\mathrm{Nu} \sim 5 - 1000 \)

\item {} 
\sphinxAtStartPar
Convección natural, \(\mathrm{Nu} \sim 0 - 100\)

\end{itemize}


\subsection{Transferencia de calor en estado estacionario}
\label{\detokenize{6_RadiacionTermica/6_RadiacionTermica:transferencia-de-calor-en-estado-estacionario}}
\sphinxAtStartPar
En estado estacionario, el flujo de calor es constante. En este caso, podemos simplificar el análisis de transferencia de calor por convección y conducción utilizando resistencias térmicas.

\noindent{\hspace*{\fill}\sphinxincludegraphics[width=700\sphinxpxdimen]{{stationary_heat_transfer}.png}\hspace*{\fill}}


\section{Fundamentos de la radiación térmica}
\label{\detokenize{6_RadiacionTermica/6_RadiacionTermica:fundamentos-de-la-radiacion-termica}}
\sphinxAtStartPar
Las vibraciones a nivel molecular también inducen polarización en la materia. Esto es similar al fenómeno de polarización inducida por ondas electromagnéticas estudiada en la unidad 3. Estos dipolos inducidos térmicamente, oscilan constantemente generando campos electromagnéticos que se propagan en dirección radial.

\noindent{\hspace*{\fill}\sphinxincludegraphics[width=700\sphinxpxdimen]{{radiating_dipole}.png}\hspace*{\fill}}

\sphinxAtStartPar
En la siguiente animación podemos ver el proceso de emisión de ondas electromagnéticas por un dipolo oscilatorio. El mapa de colores representa la intensidad del campo magnético, es decir \(|\vec{H}|\), donde rojo y azul corresponden, respectivamente, a los valorse máximos y mínimos.

\noindent{\hspace*{\fill}\sphinxincludegraphics[width=300\sphinxpxdimen]{{HW_vertical_noground}.gif}\hspace*{\fill}}


\subsection{Poder de emisión}
\label{\detokenize{6_RadiacionTermica/6_RadiacionTermica:poder-de-emision}}
\sphinxAtStartPar
Un cuerpo a temperatura \(T\) emite ondas electromagnéticas en todas las direcciones y en un espectro de longitudes de onda. En general, la distribución angular (\(\Omega\)) y espectral (\(\lambda\)) de la radiación emitida depende de las propiedades ópticas de la superficie y la temperatura del material.

\sphinxAtStartPar
Para caracterizar la intensidad de la radiación emitida por una superficie a tempertura \(T\), usamos la \sphinxstylestrong{intensidad específica o radiancia espectral}, \(I_\lambda(\lambda,\Omega,T)\).

\sphinxAtStartPar
La taza de calor total emitido por una superficie \(dA\) de un cuerpo negro en función de \(\lambda\) y \(\Omega\), \(d\dot{Q}_\mathrm{rad}\), está dada por:
\label{equation:6_RadiacionTermica/6_RadiacionTermica:519accb8-21bd-4400-9bca-bbc19146c510}\begin{equation}
d\dot{Q}_\mathrm{rad} = I_{\lambda}(\lambda,T,\theta,\phi) \sin\theta \cos\theta dA d\Omega d\lambda
\end{equation}
\sphinxAtStartPar
El término \(\cos\theta dA\) corresponde a la proyección de \(dA\) en la dirección \(\Omega\)

\noindent{\hspace*{\fill}\sphinxincludegraphics[width=300\sphinxpxdimen]{{specific_intensity1}.png}\hspace*{\fill}}

\sphinxAtStartPar
Definimos como \sphinxstylestrong{poder de emisión direccional espectral} a la relación:
\label{equation:6_RadiacionTermica/6_RadiacionTermica:aa3e888f-5d50-4fa7-9be8-b8c979a26493}\begin{equation}
E_{\lambda,\Omega}(T) = \frac{d\dot{Q}_\mathrm{rad}}{dAd\Omega d\lambda}=I_{\lambda}(\lambda,\Omega)\cos\theta ,\quad\quad\frac{\mathrm{W}}{\mathrm{m}^2\cdot\mu\mathrm{m}\cdot\mathrm{sr}}
\end{equation}
\sphinxAtStartPar
A diferencia de la intensidad específica, el poder de emisión considera la radiación effectiva emitida por una superficie.

\noindent{\hspace*{\fill}\sphinxincludegraphics[width=500\sphinxpxdimen]{{emissive_power}.png}\hspace*{\fill}}

\sphinxAtStartPar
A partir de este término podemos derivar:
\begin{itemize}
\item {} 
\sphinxAtStartPar
\sphinxstylestrong{Poder de emisión hemisférica espectral},

\end{itemize}
\begin{align*}
E_{\lambda}(T) = \frac{d\dot{Q}}{dA d\lambda} &= \int_0^{2\pi}\int_0^{\pi/2}I_{\lambda}(\lambda,\Omega)\cos\theta~\sin\theta  d\theta d\phi
  \\ 
  &=\int_\mathrm{hemi} I_{\lambda}(\lambda,\Omega)\cos\theta~d\Omega
  ,\quad\quad\frac{\mathrm{W}}{\mathrm{m}^2\cdot\mu\mathrm{m}}
\end{align*}\begin{itemize}
\item {} 
\sphinxAtStartPar
\sphinxstylestrong{Poder de emisión direccional total},

\end{itemize}
\begin{equation*}
E_\Omega(\lambda,T) = \frac{d\dot{Q}}{dAd\Omega}=\cos\theta \int_0^\infty~ I_{\lambda}(\lambda,\Omega)~d\lambda  ,\quad\quad\frac{\mathrm{W}}{\mathrm{m}^2 \cdot\mathrm{sr}}
\end{equation*}\begin{itemize}
\item {} 
\sphinxAtStartPar
\sphinxstylestrong{Poder de emisión hemisfética total},

\end{itemize}
\begin{equation*}
E(T) = \frac{d\dot{Q}}{dA}=\int_0^\infty\int_\mathrm{hemi}I_{\lambda}(\lambda,\Omega)\cos\theta~d\Omega d\lambda ,\quad\quad\frac{\mathrm{W}}{\mathrm{m}^2}
\end{equation*}

\subsection{Distribución de Planck}
\label{\detokenize{6_RadiacionTermica/6_RadiacionTermica:distribucion-de-planck}}
\sphinxAtStartPar
Max Planck en 1901 determinó que la \sphinxstylestrong{máxima radiancia espectral o intensidad específica} (flujo de energía por unidad de longitud de onda y ángulo sólido) emitida por un cuerpo a temperatura \(T\), en un medio con índice de refracción \(n_1\), está dada por:
\label{equation:6_RadiacionTermica/6_RadiacionTermica:d382d0ba-7493-4c0c-8dfa-373d48b82072}\begin{equation}
I_{\mathrm{bb},\lambda}(\lambda,T,\Omega) = \frac{C_1}{n_1\lambda^5\left[\exp\left(C_2/\lambda T\right) - 1\right]},\quad\quad\frac{\mathrm{W}}{\mathrm{m}^2\cdot\mu\mathrm{m}\cdot\mathrm{sr}}
\end{equation}
\sphinxAtStartPar
donde
\begin{align*}
C_1 &= 2hc_0^2 = 1.19104238\times 10^8 ~\mathrm{W}\cdot\mu\mathrm{m}^4/\mathrm{m}^2 \\
C_2 &= hc_0/k_\mathrm{B} = 1.43878\times10^{4}~\mu\mathrm{m}\cdot\mathrm{K}
\end{align*}
\sphinxAtStartPar
\(k_\mathrm{B} = 1.381\times 10^{-23}\) J/K \(=8.617\times 10^{-5}\) eV/K, es la constante de Boltzmann. La unidad “sr” correponde a un esteroradian.

\sphinxAtStartPar
Esta es la \sphinxstylestrong{distribución de Planck}. Representa la radiancia espectral emitida por un cuerpo idealizado, denominado \sphinxstylestrong{cuerpo negro}. Un cuerpo negro, así, representa un emisor perfecto, capaz de emitir la máxima radiacion posible a una temperatura \(T\).

\sphinxAtStartPar
El poder de emisión hemisférico espectral de la superficie de un cuerpo negro, \(E_\mathrm{bb}(\lambda,T)\), se obtiene integrando la radiancia espectral por ángulo sólido en el límite de una hemiesfera:
\label{equation:6_RadiacionTermica/6_RadiacionTermica:876bc3cb-254b-4556-a824-304d8c8bef5f}\begin{equation}
\int_\mathrm{hemi} I_{\mathrm{bb},\lambda}(\lambda,T,\Omega)\cos\theta d\Omega = \pi I_{\mathrm{bb},\lambda}(\lambda,T) = E_{\mathrm{bb},\lambda}(\lambda,T),\quad\quad\frac{\mathrm{W}}{\mathrm{m}^2\cdot\mu\mathrm{m}}
\end{equation}
\sphinxAtStartPar
A partir de la integral de \(E_\mathrm{bb}(\lambda,T)\) en el espectro de longitudes de onda, obtenemos el poder de emisión hemisferico total de un cuerpo negro:
\label{equation:6_RadiacionTermica/6_RadiacionTermica:c2169311-dcad-4504-a95c-4356f4be9f68}\begin{equation}
\int_0^\infty E_{\mathrm{bb},\lambda}(\lambda,T) d\lambda = \sigma T^4,\quad\quad\frac{\mathrm{W}}{\mathrm{m}^2}
\end{equation}
\sphinxAtStartPar
donde \(\sigma = 5.670\times10^{-8}\) W/m\(^2\cdot\)K\(^4\), es la \sphinxstyleemphasis{constante de Stefan\sphinxhyphen{}Boltzmann.} Esta fórmula se conoce como la \sphinxstylestrong{ley de Stefan\sphinxhyphen{}Boltzmann}

\sphinxAtStartPar
En la siguiente figura, se ilustra \(E_{\mathrm{bb},\lambda}(\lambda,T)\) función de la temperatura y longitud de onda.

\noindent{\hspace*{\fill}\sphinxincludegraphics[width=700\sphinxpxdimen]{{blackbody_rad}.png}\hspace*{\fill}}

\sphinxAtStartPar
A medida que \(T\) aumenta, notamos que el máximo de la curva se desplaza hacia el azul. La longitud de onda correspondiente a este máximo, \(\lambda_\mathrm{peak}\), está definida por la \sphinxstylestrong{ley de desplazamiento de Wien:}
\label{equation:6_RadiacionTermica/6_RadiacionTermica:e3a96c87-4563-43e3-8bba-0f97e12f7615}\begin{equation}
\lambda_\mathrm{peak}T = 2897.8\quad\mu\mathrm{m}\cdot\mathrm{K}
\end{equation}
\sphinxAtStartPar
Esta relación permite entender el cambio de color de la fuente emisora con la tempertura.

\sphinxAtStartPar
Recordemos, sin embargo,  que \sphinxstylestrong{el color de un material no solo se define por la emisión de radiación, sino también por la forma en la que interactúa con la luz incidente}. Como revisamos en las unidades anteriores, esta interacción está condionada por las propiedades radiativas.


\subsection{Propiedades Radiativas}
\label{\detokenize{6_RadiacionTermica/6_RadiacionTermica:propiedades-radiativas}}
\sphinxAtStartPar
Definimos como \sphinxstylestrong{emisividad direccional espectral, \(\epsilon_{\lambda,\Omega}\),} a la \sphinxstyleemphasis{razón entre la radiación emitida por una superficie, \(I_\lambda(\lambda,T,\Omega)\), y la radiación emitida por un cuerpo negro, ambas a temperatura \(T\)}:
\label{equation:6_RadiacionTermica/6_RadiacionTermica:4c3b1832-62b1-43b2-a1d6-6967fea6c512}\begin{equation}
\epsilon_{\lambda,\Omega} = \frac{I_\lambda(\lambda,T,\Omega)}{I_{\mathrm{bb},\lambda}(\lambda,T,\Omega)}
\end{equation}
\sphinxAtStartPar
De esta forma, \(\epsilon\) es una propiedad adimensional de superfice que varía entre \(0 \le \epsilon \le 1\).

\sphinxAtStartPar
Definimos como \sphinxstylestrong{absortividad direccional espectral, \(A_{\lambda,\Omega}\),} a la \sphinxstyleemphasis{porción de radiación incidente que es absorbida por una superficie}. Igualmente, \(0 \le \alpha \le 1\).

\sphinxAtStartPar
A través de la \sphinxstylestrong{ley de Kirchhoff}, podemos establecer una relación entre la absortancia y emisividad espectral direccional:
\label{equation:6_RadiacionTermica/6_RadiacionTermica:ab60c71a-9d21-46ad-b21c-f7d018f721df}\begin{equation}
\epsilon_{\lambda,\Omega} = A_{\lambda,\Omega}
\end{equation}
\sphinxAtStartPar
En otras palabras, las propiedades de un material como receptor o emisor de radiación, son iguales. Este concepto, denominado \sphinxstyleemphasis{reciprocidad}, es consecuencia de las ecuaciones de Maxwell y es la base fundamental para el diseño de antenas y radares.

\sphinxAtStartPar
Por conservación de energía:
\label{equation:6_RadiacionTermica/6_RadiacionTermica:ddc2a57f-cd8b-4c2d-a883-02d3e42e90d3}\begin{equation}
A_{\lambda,\Omega} + R_{\lambda,\Omega} + T_{\lambda,\Omega} = 1
\end{equation}
\sphinxAtStartPar
donde \(R_{\lambda,\Omega}\) y \(T_{\lambda,\Omega}\) son, respectivamente, la reflectividad y transmisividad espectral direccional del material

\sphinxAtStartPar
Debido a la naturaleza de la radiación térmica, la polarización de las ondas electromagnéticas es aleatoria. Así, \(R_{\lambda,\Omega}\) y \(T_{\lambda,\Omega}\) se calculan como:
\label{equation:6_RadiacionTermica/6_RadiacionTermica:d42a9bb2-3965-4f1a-a000-bbaccd0f05b5}\begin{equation}
R_{\lambda,\Omega} = \frac{R_{\lambda,\Omega}^\mathrm{TM}+R_{\lambda,\Omega}^\mathrm{TE}}{2}\quad\quad
T_{\lambda,\Omega} = \frac{T_{\lambda,\Omega}^\mathrm{TM}+T_{\lambda,\Omega}^\mathrm{TE}}{2}
\end{equation}
\sphinxAtStartPar
A partir de estas relaciones podemos determinar \(\epsilon_{\lambda,\Omega}\).

\sphinxAtStartPar
Cabe mencionar que en textos de radiometría y transferencia de calor, la reflectividad y transmisividad se denominan, respectivamente, \sphinxstylestrong{reflectancia (\(\rho\))}, \sphinxstylestrong{transmitancia (\(\tau\)).} Igualmente la absortividad se denomina \sphinxstylestrong{absortancia (\(\alpha\)).} Ambos términos son equivalentes.

\sphinxAtStartPar
En este curso, seguiremos utilizando los términos y notación de óptica, es decir \(R_{\lambda,\Omega}\), \(T_{\lambda,\Omega}\) y \(A_{\lambda,\Omega}\), para evitar confusiones.

\sphinxAtStartPar
Como ejemplo, analicemos el poder de emisión espectral direccional, \(E_{\lambda,\Omega}(T)\) y la emisividad \(\epsilon(\lambda,\Omega)\) de una capa de vidrio en función de la temperatura (\(T\)), espesor (\(d\)) y dirección (\(\theta\)). En este caso:
\begin{align*}
E_{\lambda,\Omega}(T) &= \epsilon(\lambda,\Omega)I_{\mathrm{bb},\lambda}(\lambda,\Omega,T)\cos\theta \\[10pt]
 &= \left[1 - R_{\lambda,\Omega} - T_{\lambda,\Omega}\right]I_{\mathrm{bb},\lambda}(\lambda,\Omega,T)\cos\theta
\end{align*}
\sphinxAtStartPar
Antes, analicemos el índice de refracción del vidrio (sílicice, SiO\(_2\)), en el espectro \(\lambda\in[0.3,15]\) \(\mu\)m.

\begin{sphinxuseclass}{cell}
\begin{sphinxuseclass}{tag_hide-input}
\end{sphinxuseclass}
\end{sphinxuseclass}
\begin{sphinxuseclass}{cell}\begin{sphinxVerbatimInput}

\begin{sphinxuseclass}{cell_input}
\begin{sphinxVerbatim}[commandchars=\\\{\}]
\PYG{n}{showplot}\PYG{p}{(}\PYG{p}{)}
\end{sphinxVerbatim}

\end{sphinxuseclass}\end{sphinxVerbatimInput}
\begin{sphinxVerbatimOutput}

\begin{sphinxuseclass}{cell_output}
\noindent\sphinxincludegraphics{{6_RadiacionTermica_63_0}.png}

\end{sphinxuseclass}\end{sphinxVerbatimOutput}

\end{sphinxuseclass}
\sphinxAtStartPar
Comparamos \(E_{\lambda,\Omega}(T)\) con el poder de emisión espectral direccional del cuerpo negro, \(E_{\mathrm{bb},\lambda,\Omega} = I_{\mathrm{bb},\lambda}\cos\theta\)

\begin{sphinxuseclass}{cell}
\begin{sphinxuseclass}{tag_hide-input}
\end{sphinxuseclass}
\end{sphinxuseclass}
\begin{sphinxuseclass}{cell}\begin{sphinxVerbatimInput}

\begin{sphinxuseclass}{cell_input}
\begin{sphinxVerbatim}[commandchars=\\\{\}]
 \PYG{k+kn}{from} \PYG{n+nn}{ipywidgets} \PYG{k+kn}{import} \PYG{n}{interact}

\PYG{n+nd}{@interact}\PYG{p}{(}\PYG{n}{T}\PYG{o}{=}\PYG{p}{(}\PYG{l+m+mi}{300}\PYG{p}{,}\PYG{l+m+mi}{1000}\PYG{p}{,}\PYG{l+m+mi}{10}\PYG{p}{)}\PYG{p}{,} \PYG{n}{d}\PYG{o}{=}\PYG{p}{(}\PYG{l+m+mi}{0}\PYG{p}{,}\PYG{l+m+mi}{10}\PYG{p}{,}\PYG{l+m+mf}{0.01}\PYG{p}{)}\PYG{p}{,} \PYG{n}{lam0}\PYG{o}{=}\PYG{p}{(}\PYG{l+m+mi}{5}\PYG{p}{,}\PYG{l+m+mi}{10}\PYG{p}{,}\PYG{l+m+mf}{0.01}\PYG{p}{)}\PYG{p}{,} \PYG{n}{theta0}\PYG{o}{=}\PYG{p}{(}\PYG{l+m+mi}{0}\PYG{p}{,}\PYG{l+m+mi}{90}\PYG{p}{,}\PYG{l+m+mi}{1}\PYG{p}{)}\PYG{p}{)}
\PYG{k}{def} \PYG{n+nf}{g}\PYG{p}{(}\PYG{n}{T}\PYG{o}{=}\PYG{l+m+mi}{300}\PYG{p}{,}\PYG{n}{d}\PYG{o}{=}\PYG{l+m+mi}{1}\PYG{p}{,} \PYG{n}{lam0}\PYG{o}{=}\PYG{l+m+mi}{10}\PYG{p}{,} \PYG{n}{theta0}\PYG{o}{=}\PYG{l+m+mi}{0}\PYG{p}{)}\PYG{p}{:}
    \PYG{k}{return} \PYG{n}{plot\PYGZus{}emisivity\PYGZus{}glass}\PYG{p}{(}\PYG{n}{T}\PYG{p}{,}\PYG{n}{d}\PYG{p}{,}\PYG{n}{lam0}\PYG{p}{,}\PYG{n}{theta0}\PYG{p}{)}
\end{sphinxVerbatim}

\end{sphinxuseclass}\end{sphinxVerbatimInput}
\begin{sphinxVerbatimOutput}

\begin{sphinxuseclass}{cell_output}
\begin{sphinxVerbatim}[commandchars=\\\{\}]
interactive(children=(IntSlider(value=300, description=\PYGZsq{}T\PYGZsq{}, max=1000, min=300, step=10), FloatSlider(value=1.0…
\end{sphinxVerbatim}

\end{sphinxuseclass}\end{sphinxVerbatimOutput}

\end{sphinxuseclass}
\sphinxAtStartPar
Al igual que con el poder de emisión, la emisividad puede también ser definida de forma hemisférica (integración por ángulo sólido) o total (integración por longitudes de onda)

\sphinxAtStartPar
\sphinxstylestrong{Emisividad direccional total}
\label{equation:6_RadiacionTermica/6_RadiacionTermica:56d96298-1633-45cd-a604-fa48e7e5279c}\begin{equation}
\epsilon_\Omega(\Omega,T) = \frac{E_\Omega(T)}{E_{\Omega,bb}(T)} =\frac{\int_0^\infty \epsilon_{\lambda,\Omega} I_{\mathrm{bb},\lambda} (T)\cos\theta~d\lambda}{\int_0^\infty I_{\mathrm{bb},\lambda} (T)\cos\theta~d\lambda} = \frac{\pi}{\sigma T^4}\int_0^\infty \epsilon_{\lambda,\Omega} I_{\mathrm{bb},\lambda} (T)~d\lambda
\end{equation}
\sphinxAtStartPar
\sphinxstylestrong{Emisividad hemisférica espectral}
\label{equation:6_RadiacionTermica/6_RadiacionTermica:8cb7ba4c-f6fb-42be-9785-6825f28b58c3}\begin{equation}
\epsilon_\lambda(\lambda) = \frac{E_\lambda(T)}{E_{\lambda,bb}(T)} = \frac{\int_\mathrm{hemi}\epsilon_{\lambda,\Omega}I_{\mathrm{bb},\lambda}(T)\cos\theta~d\Omega}{\pi I_{\mathrm{bb},\lambda}(T)}  = \frac{1}{\pi}\int_\mathrm{hemi}\epsilon_{\lambda,\Omega}\cos\theta~d\Omega
\end{equation}
\sphinxAtStartPar
\sphinxstylestrong{Emisividad hemisférica total}
\label{equation:6_RadiacionTermica/6_RadiacionTermica:1dacb1e1-3c11-4880-91f7-b6aea703c4d9}\begin{equation}
\epsilon(T) = \frac{E(T)}{\sigma T^4} =\frac{1}{\sigma T^4}\int\int_0^\infty \epsilon_{\lambda,\Omega} I_{\mathrm{bb},\lambda} (T)\cos\theta~d\lambda~d\Omega
\end{equation}
\sphinxAtStartPar
Notar que las valores totatles (integración en el espectro), implícitamente dependen de \(T\) debido al factor \(\sigma T^4\).


\subsection{Materiales idealizados}
\label{\detokenize{6_RadiacionTermica/6_RadiacionTermica:materiales-idealizados}}\begin{itemize}
\item {} 
\sphinxAtStartPar
Decimos que una superficie es \sphinxstylestrong{difusa} cuando sus propiedades radiativas no dependen de \(\Omega\). Es práctica común suponer que las superficies son emisores difusos, con una emisividad igual al valor de la dirección normal (\(\theta = 0\)).

\end{itemize}
\begin{itemize}
\item {} 
\sphinxAtStartPar
Decimos que una superficie es \sphinxstylestrong{gris} cuando la emisividad es independiente de \(\lambda\).

\end{itemize}
\begin{itemize}
\item {} 
\sphinxAtStartPar
Decimos que una superficie es \sphinxstylestrong{opaca} cuando \(T_{\lambda,\Omega} = 0\). En este caso,
tenemos \(A_{\lambda,\Omega} + R_{\lambda,\Omega} = 1\)

\end{itemize}
\begin{itemize}
\item {} 
\sphinxAtStartPar
En el caso de \sphinxstylestrong{gases}, \(R_{\lambda,\Omega} \approx 0\), así \(A_{\lambda,\Omega} + T_{\lambda,\Omega} = 1\)

\end{itemize}


\subsection{Irradiancia (G) y Radiosidad (J)}
\label{\detokenize{6_RadiacionTermica/6_RadiacionTermica:irradiancia-g-y-radiosidad-j}}
\sphinxAtStartPar
Definimos como \sphinxstylestrong{irradiación espectral direccional, \(G_{\lambda,\Omega}\)} a la radiancia espectral incidente en una superficie.

\noindent{\hspace*{\fill}\sphinxincludegraphics[width=300\sphinxpxdimen]{{irradiance}.png}\hspace*{\fill}}

\sphinxAtStartPar
Matématicamente:
\label{equation:6_RadiacionTermica/6_RadiacionTermica:1aa8ee01-5f19-4b6b-a74b-924bdfbd97e2}\begin{equation}
G_{\lambda,\Omega} = I_i(\lambda,\Omega) \cos\theta\quad\frac{\mathrm{W}}{\mathrm{m}^2}
\end{equation}
\sphinxAtStartPar
donde \(I_i\) es la radiancia espectral incidente sobre una superficie \(dA\)

\sphinxAtStartPar
Definimos como \sphinxstylestrong{radiosidad, \(J_{\lambda,\Omega}\)} a la combinación de radiación emitida y reflejada por una superficie

\noindent{\hspace*{\fill}\sphinxincludegraphics[width=250\sphinxpxdimen]{{radiosity}.png}\hspace*{\fill}}

\sphinxAtStartPar
Matemáticamente:
\label{equation:6_RadiacionTermica/6_RadiacionTermica:0ed26ef9-1321-47a9-91a8-3a658d5dde18}\begin{equation}
J_{\lambda,\Omega} = R_{\lambda,\Omega}G_{\lambda,\Omega}+E_{\lambda,\Omega}
\end{equation}
\sphinxAtStartPar
Al igual que con el poder de emisión, la irradiancia (\(G_{\lambda,\Omega}\)) y la radiosidad (\(J_{\lambda,\Omega}\)) pueden ser definidas de forma hemisférica (integrando por ángulo sólido) o total (integrando por longitud de onda).


\section{Referencias}
\label{\detokenize{6_RadiacionTermica/6_RadiacionTermica:referencias}}\begin{itemize}
\item {} 
\sphinxAtStartPar
Çengel Y. A y Ghanjar A. J. \sphinxstylestrong{Capítulo 12 \sphinxhyphen{} Fundamentos de la radiación térmica} en \sphinxstyleemphasis{Transferencia de calor y masa}, 4ta Ed, McGraw Hill, 2011

\end{itemize}

\begin{sphinxuseclass}{cell}\begin{sphinxVerbatimInput}

\begin{sphinxuseclass}{cell_input}
\begin{sphinxVerbatim}[commandchars=\\\{\}]
\PYG{k+kn}{from} \PYG{n+nn}{IPython}\PYG{n+nn}{.}\PYG{n+nn}{display} \PYG{k+kn}{import} \PYG{n}{YouTubeVideo}
\PYG{n}{YouTubeVideo}\PYG{p}{(}\PYG{l+s+s1}{\PYGZsq{}}\PYG{l+s+s1}{FDmYCI\PYGZus{}xYlA}\PYG{l+s+s1}{\PYGZsq{}}\PYG{p}{,} \PYG{n}{width}\PYG{o}{=}\PYG{l+m+mi}{600}\PYG{p}{,} \PYG{n}{height}\PYG{o}{=}\PYG{l+m+mi}{400}\PYG{p}{,}  \PYG{n}{playsinline}\PYG{o}{=}\PYG{l+m+mi}{0}\PYG{p}{)}
\end{sphinxVerbatim}

\end{sphinxuseclass}\end{sphinxVerbatimInput}
\begin{sphinxVerbatimOutput}

\begin{sphinxuseclass}{cell_output}
\noindent\sphinxincludegraphics{{6_RadiacionTermica_85_0}.jpg}

\end{sphinxuseclass}\end{sphinxVerbatimOutput}

\end{sphinxuseclass}






\renewcommand{\indexname}{Index}
\printindex
\end{document}